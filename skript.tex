\documentclass{article}

\usepackage{amsmath,amsfonts,amssymb,amsthm}
\usepackage{enumitem}
\usepackage{a4wide}
\usepackage[document]{ragged2e}
\usepackage{tikz}
\usepackage[utf8]{inputenc}
\usepackage[T1]{fontenc}
\usepackage[german]{babel}
\usepackage{hyphenat}
\usepackage{listings}
\usepackage{caption}
\usepackage{makeidx}
\usepackage{wasysym}
\usepackage{color}
\usepackage{graphicx}
\usepackage{accents}
\usepackage{stmaryrd}
\usepackage{pgfplots}


\usetikzlibrary{positioning, calc}
\usetikzlibrary{decorations.pathreplacing,angles,quotes}
\usetikzlibrary{decorations.pathmorphing,snakes}
\usetikzlibrary{shapes.misc}
\usetikzlibrary{intersections,through,backgrounds}
\usetikzlibrary{matrix}


\theoremstyle{plain}
\newtheorem{theorem}{Theorem}
\newtheorem{lemma}[theorem]{Lemma}
\newtheorem*{lemma*}{Lemma}
\newtheorem{cor}[theorem]{Korollar}
\newtheorem*{cor*}{Korollar}
\newtheorem{prop}[theorem]{Proposition}
\setlength{\parskip}{1em}

\theoremstyle{definition}
\newtheorem{definition}[theorem]{Definition}
\newtheorem*{definition*}{Definition}
\newtheorem{notation}[theorem]{Notation}
\newtheorem{bemerkung}[theorem]{Bemerkung}
\newtheorem*{bemerkung*}{Bemerkung}
\newtheorem{bsp}[theorem]{Beispiel}
\newtheorem*{remark*}{remark}
\newtheorem{remark}{remark}

\numberwithin{equation}{section}

\newcommand{\norm}[1] {
\left|\left| #1 \right|\right|
}

\newcommand{\hnorm}[1] {
\left|\left|\left| #1 \right|\right|\right|
}

\newcommand{\colvec}[1]{
\begin{pmatrix}#1\end{pmatrix}
}

\newcommand{\vect}[1]{
\begin{pmatrix}#1\end{pmatrix}
}

\newcommand{\skprod}[2]{
\left \langle #1,#2 \right \rangle
}
\newcommand{\abs}[1] {
\left| #1 \right|
}

\newcommand{\br}[1] {
\left( #1 \right)
}

\newcommand{\R}[0] {
\mathbb R
}

\newcommand{\Z}[0] {
    \mathbb Z
}

\newcommand{\N}[0] {
    \mathbb N
}

\newcommand{\F}[0]{
    \mathcal F
}

\newcommand{\srmatrix}[1] {
\left( \begin{smallmatrix} #1 \end{smallmatrix} \right)
}

\newcommand{\mtitle}[1] {
    \begin{center}
        \large{\textbf{#1}}
    \end{center}
}

\newcommand{\Index}[1]{#1\index{#1}}

\newcommand{\mim}[1] {
\underline{\textbf{#1\index{#1}}}
}

\newcommand{\C}[0]{
    \cdot
}

\newcommand{\pa}[1] {
    \par{\textbf{#1}}
}

\newcommand{\tvmark}[3] {
    \draw[] (#1,0.1+#2) -- (#1,-0.1+#2) node[anchor =north] {\small #3};
}
\newcommand{\thmark}[3] {
    \draw[] (#1+0.1,#2) -- (#1-0.1,#2) node[left] {\small #3};
}

\newcommand{\tvnmark}[3] {
    \draw[] (#1,#2-0.1) node[below] {\small #3};
}

\newcommand{\tlbar}[5]{
    \draw[thick] (#1,#2-0.1) node[below] {\small #4} -- (#1,#3) node[above] {#5};
    \draw[thick] (#1-0.1,#3) -- (#1+0.1,#3);
}

\newcommand{\filter}[1] {
\begin{tabular}{|c|c|c|}
    \hline
    #1\\
    \hline
\end{tabular}
}

\newcommand{\x}[0] {
  \boldsymbol{x}
}
\newcommand{\y}[0] {
    \boldsymbol{y}
}

\newcommand{\mat}[1] {
\begin{pmatrix} #1 \end{pmatrix}
}

\definecolor{dkgreen}{rgb}{0,0.6,0}
\definecolor{gray}{rgb}{0.5,0.5,0.5}
\definecolor{mauve}{rgb}{0.58,0,0.82}

\lstset{frame=tb,
language=Matlab,
aboveskip=10mm,
belowskip=10mm,
showstringspaces=false,
columns=flexible,
basicstyle={\small\ttfamily},
numbers=left,
identifierstyle=\color{black},
numberstyle=\small\color{gray},
keywordstyle=\color{blue},
commentstyle=\color{dkgreen},
stringstyle=\color{mauve},
breaklines=true,
breakatwhitespace=true,
tabsize=3}

%Font
%\usepackage{tgadventor}
\renewcommand*\rmdefault{cmss}
\makeindex

\begin{document}
\title{Mathematische Bildverarbeitung}
%\author{Jonas Sattler}
\date{}
\maketitle

\tableofcontents

\newpage

\section{Überblick}
    \subsection{Techniken der Bildverarbeitung}
        \begin{enumerate}[label=\textbullet]
            \item Kontrastverbesserung
            \item Entrauschen
            \item Kantendetektion
            \item Schärfen
            \item Inpainting
            \item Segmentierung (Einzlene Objekte detektieren)
            \item Registrierung (Bilder des selben Objektes in Einklang bringen)
        \end{enumerate}

    \subsection{Unser Fokus}
        \begin{enumerate}[label=\textbullet]
            \item Mathematische Beschreibung
        \end{enumerate}

    \subsection{Verwandte Vorlesungen}
        \begin{enumerate}[label=\textbullet]
            \item 3D computervision
            \item Digitale Bildanalyse
            \item Mustererkennung und Datenkompression
            \item Medical imaging
        \end{enumerate}

    \subsection{Literatur}
        \begin{enumerate}[label=\textbullet]
            \item Bredies, Lorenz : Mathematische Bildverarbeitung
            \item Aubert, Kornprobst : Mathematical Problems in Image Processing
            \item Modersitzki : Numerical Methods for Image Registration
            \item Alt : Lineare Funktionalanalysis
        \end{enumerate}

\section{Was ist ein Bild?}
    \subsection{Definition}
        \begin{minipage}[t]{0.45\linewidth}
            \mtitle{\underline{Digitale/diskrete Sicht}}
            \begin{center}
                \begin{tikzpicture}
                    \fill[black!30!white] (1,2) rectangle (1.5,2.5);
                    \draw[step=0.5,thin,draw=black] (0,0) grid (4,4);
                    \draw[thick] (0,0) rectangle (4,4);
                \end{tikzpicture}
                \captionof{figure}{Diskretes Bild}
                Darstellung als Matrix.
            \end{center}
        \end{minipage}
        \hfill\vrule\hfill
        \begin{minipage}[t]{0.45\linewidth}
            \mtitle{\underline{Kontinuierlich/analoge Sicht}}
            \begin{center}
                \begin{tikzpicture}
                    \draw[thick,->] (0,0) -- (3,0) node[anchor=west] {X};
                    \draw[thick,->] (0,0) -- (0,3) node[anchor=south] {Y};
                    \draw[thick] (0.2,0.2) rectangle (2.8,2.8);
                    \draw (1.5,2) node[] {\tiny{\textbullet}} node[anchor=west] {\small$(x,y)$};
                    \draw[] (1.5,0.1) -- node[anchor=north] {\small x} (1.5,-0.1);
                    \draw[] (0.1,2) -- node[anchor=east] {\small y} (-0.1,2);
                    \draw[|-|] (0.2,-0.4) node[anchor=north] {\small a} -- (2.8,-0.4) node[anchor=north] {\small b};
                    \draw[|-|] (-0.4,0.2) node[anchor=east] {\small c} -- (-0.4,2.8) node[anchor=east] {\small d};
                \end{tikzpicture}
                \captionof{figure}{Kontinuierliches Bild}
                Darstelllung als Funktion in zwei Veränderlichen
            \end{center}
        \end{minipage}

        \ \\

        \begin{minipage}[t]{0.47\linewidth}
            \pa{Werkzeuge:} Lineare Algebra
            \pa{Vorteile:} Endlicher Speicher
            \pa{Nachteile:} Probleme bei zoomen und drehen
        \end{minipage}
        \hfill\vrule\hfill
        \begin{minipage}[t]{0.47\linewidth}
            \pa{Werkzeuge:} Analysis
            \pa{Vorteile:} Mehr Freiheit (z.b. Kante=Linie entlang einer Unstetigkeit)
            \pa{Nachteile:} Unendlicher Speicher
        \end{minipage}

        \begin{definition*}
            Ein \mim{Bild} ist eine Funktion $u: \Omega \to F$, wobei $\Omega \subset \mathbb Z^d$ (im diskreten Fall) oder $\Omega \subset \mathbb R^d$ (im kontinuierlichen Fall).
            \begin{enumerate}
                \item[$d=2$:] Typisches 2D Bild
                \item[$d=3$:] 3D-Bild bzw. "Körper" \ \underline{oder} Video: 2D Ort + Zeit
            \end{enumerate}
            F ist der \mim{Farbraum}, Beispiele:
            \begin{enumerate}[label=\textbullet]
                \item F$=[0,1]$ oder F=$\{0,1,..., 255\}$, Graustufen
                \item F$=\{0,1\}$ schwarz/weiß
                \item F$=[0,1]^3$ oder F$=\{0,1,...,255\}^3$ Farbbilder
            \end{enumerate}
        \end{definition*}
    \subsection{Umwandlung}

        \pa{Kontinuierlich $\to$ Diskret:}\\
            \begin{minipage}[t]{0.49\linewidth}
                \
                \begin{center}
                    \begin{tikzpicture}
                        \draw (-0.5,2) node {$\Omega=$};
                        \draw[step=0.5,thin,draw=black,dotted] (0.01,0.01) grid (3.99,3.99);
                        \draw[thick] (0,0) rectangle (4,4);
                        \draw[thick] (2,2) rectangle (2.5,2.5);
                        \draw[] (2.1,2.1) -- (3,-0.3) node[anchor = south west, yshift = -7] {\small Box $B_i$};
                    \end{tikzpicture}
                \end{center}
            \end{minipage}
            \hfill\vrule\hfill
            \begin{minipage}[t]{0.49\linewidth}
                \
                \begin{center}
                    \begin{enumerate}[label=\textbullet]
                        \item $\Omega$ in Gitter zerlegen
                        \item Jede Box durch nur einen Farbwert approximieren
                        \item Etwa durch den Funktionswert im Mittelpunkt der Box
                        \item oder durch den Mittlewert in der Box:\\ $\displaystyle \frac{1}{\abs{B_i}} \C \int_{B_i}u(x)dx$
                    \end{enumerate}
                \end{center}
            \end{minipage}

                \ \\
            \pa{Diskret $\to$ Kontinuierlich:}\\
                \begin{minipage}[t]{0.49\linewidth}
                    \
                    \begin{center}
                        \begin{tikzpicture}
                            \draw (-1,2) node {$\Omega=$};
                            \draw[->] (-0.25,-0.25) -- (-0.25,4.5);
                            \draw[->] (-0.25,-0.25) -- (4.5,-0.25);
                            \draw (-0.35,4) -- (-0.15,4);
                            \draw (-0.35,0) -- (-0.15,0);
                            \draw (0,-0.35) -- (0,-0.15);
                            \draw (4,-0.35) -- (4,-0.15);
                            \draw[step=0.5,thin,draw=black] (0,0) grid (4,4);
                            \foreach \x in {0.25,0.75,..., 3.75}{
                                \foreach \y in {0.25,0.75,..., 3.75}{
                                    \draw (\x,\y) node {\tiny \textbullet};
                            }
                            }
                            \draw[thick] (0,0) rectangle (4,4);
                            \draw[thick] (2,2) rectangle (2.5,2.5);
                            \draw[] (2.1,2.1) -- (3,-0.7) node[anchor = south west, yshift = -7] {\small Box $B_i$};
                        \end{tikzpicture}
                    \end{center}
                \end{minipage}
                \hfill\vrule\hfill
                \begin{minipage}[t]{0.49\linewidth}
                    \
                    \begin{center}
                        \begin{enumerate}
                            \item[1.] Idee: Jeder Punkt der Box $B_i$ erhält den Funktionswert von $B_i$ aus als diskretem Pixel $\Rightarrow$ \mim{Nearest neighbour interpolation}.
                            \item[2.] Idee: Mittelpunkt von Box $B_i$ erhält den Wert von Pixel $B_i$ sonst wird interpoliert.\\
                            Grauwert $g:=$ Gewichtetes Mittel aus Grauwerten $a,b,c,d$.\\
                        \end{enumerate}

                            \begin{center}
                                \begin{tikzpicture}[scale=1.5]
                                    \draw[step = 1] (0.75,0.75) grid (3.25,3.25);
                                    \foreach \x in {1.5,2.5}{
                                        \foreach \y in {1.5,2.5}{
                                            \draw (\x,\y) node {\tiny \textbullet};
                                    }
                                    }
                                    \draw (1.5,2.5) node[anchor=north west] {a};
                                    \draw (2.5,2.5) node[anchor=south west] {b};
                                    \draw (1.5,1.5) node[anchor=north west] {c};
                                    \draw (2.5,1.5) node[anchor=north west] {d};

                                    \draw (1.5,1.5) rectangle (2.5,2.5);
                                    \draw (1.45,2.15) -- (1.55,2.15);
                                    \draw (1.8,2.45) -- (1.8,2.55);
                                    \draw[decorate,decoration={brace,amplitude=2pt}] (1.5,2.6) -- node[anchor=south] {\small $\alpha$} (1.8,2.6);
                                    \draw[decorate,decoration={brace,amplitude=2pt}] (1.4,2.15) --node[anchor=east] {\small $\beta$} (1.4,2.5);
                                    \draw (1.8,2.15) node {\tiny \textbullet} node[anchor=west, xshift=-2] {\small p};
                                    \draw[decorate,decoration={brace,amplitude=4pt}] (1.5,2.9) -- node[anchor= south west,yshift = 2] {\small 1} (2.5,2.9);
                                    \draw[decorate,decoration={brace,amplitude=4pt}] (1.1,1.5) -- node[anchor= south east, xshift=-2] {\small 1} (1.1,2.5);
                                \end{tikzpicture}
                            \end{center}

                            \small{$g=(1-\alpha) \C (1-\beta) \C a + \alpha \C (1- \beta) \C b + (1-\alpha) \C \beta \C c + \alpha \C \beta \C d$}\\
                            Dieses wird \mim{Bilinear interpolation} genannt.
                    \end{center}
                \end{minipage}

    \subsection{Beispiel Rotation}
        \begin{center}
            \begin{tikzpicture}%TODO redo
                \draw (0,0.1) -- node[anchor=center] {\tiny \textbullet} node[anchor=north] {\tiny $0$} (3,0.1);
                \draw  (4,0.1) node  {$\overset{\text{\footnotesize um $\alpha$ drehen}}{\Rightarrow}$};
                \draw (6.5,0.1) -- ++(30:1.5);
                \draw (6.5,0.1) node[anchor=center] {\tiny \textbullet} node[anchor=north] {\tiny $0$} -- ++(210:1.5);
                \draw[dotted] (5,0.1) -- (8,0.1);
                \draw (7,0.1) arc[radius=0.5,start angle=0,end angle=30]  node[anchor = north west] {\tiny $\alpha$};
            \end{tikzpicture}
        \end{center}

        \pa{1. Fall, kontinuierliches Bild}\\
            Sei $u$ das alte Bild und $v$ das neue Bild, dann ist die Drehung gegeben durch eine \mim{Drehmatrix}:

            \[D_\varphi \in \R^{d \times d}, D_\varphi=\srmatrix{cos(\varphi) \ -sin(\varphi) \\ sin(\varphi) \ cos(\varphi)}\]

            Damit folgt, dass $D(u)= D_\varphi \Omega$ und $v(x)=u(\underbrace{D_\varphi^{-1}x}_{\in \Omega}) = u(D_{-\varphi}x)$. ($D(u)$ ist die \mim{Domain} von $u$)

        \pa{2. Fall, diskretes Bild}\\
            Dieses ist problematisch, denn I.A. $x \in \Z^d$, aber $D_\varphi x \not \in \Z^d$.

            \begin{center}
                \begin{tikzpicture}
                    \foreach \x in {0,1,2,3}{
                            \draw (0.5*\x,0) node {\tiny \textbullet};
                    }
                    \draw[decorate,decoration={brace,amplitude=4pt,mirror}] (-0.1,-0.1) -- node[anchor= north] {\small $\subset \Z^d$} (1.6,-0.1);
                    \draw[->] (1.75,0) -- (2,0);
                    \draw[step=0.5, shift={(2.25,0.25)}] (0,0) grid (2,-0.5);
                    \foreach \x in {0,1,2,3}{
                        \draw[shift={(2.5,0)}] (0.5*\x,0) node {\tiny \textbullet};
                    }
                    \draw[decorate,decoration={brace,amplitude=4pt,mirror}] (2.15,-0.4) -- node[anchor= north] {\small $\subset \R^d$} (4.35,-0.4);
                    \draw[->] (4.5,0) -- node[above] {\small $D_\varphi$} (4.75,0);
                    \draw[step=0.5, shift={(5,0.25)},rotate around = {45:(1,-0.25)}] (0,0) grid (2,-0.5);
                    \foreach \x in {0,1,2,3}{
                        \draw[shift={(5.25,0)},rotate around = {45:(0.75,0)}] (0.5*\x,0) node {\tiny \textbullet};
                    }
                    \draw[decorate,decoration={brace,amplitude=4pt,mirror}] (4.9,-1) -- node[anchor= north] {\small $\subset \R^d$} (7.1,-1);
                    \draw[->] (7.25,0) -- node[above] {\small rastern} (7.5,0);
                    \draw[->] (4.5,0) -- node[above] {\small $D_\varphi$} (4.75,0);
                    \draw[step=0.5, shift={(8,0.25)},rotate around = {45:(1,-0.25)}] (0,0) grid (2,-0.5);
                    \foreach \x in {0,1,2,3}{
                        \draw[shift={(8.25,0)},rotate around = {45:(0.75,0)}] (0.5*\x,0) node {\tiny \textbullet};
                    }
                    \draw[dotted,step=0.5] (7.9,1) grid (10.1,-1);
                    \draw[decorate,decoration={brace,amplitude=4pt,mirror}] (7.9,-1.2) -- node[anchor= north] {\small $\subset \Z^d$} (10.1,-1.2);
                \end{tikzpicture}
            \end{center}

            Weiterhin ist $v(x)=u(D_\varphi^{-1} x)$, wobei der konkrete Wert durch Interpolation bestimmt wird.

            \begin{center}
                \begin{tikzpicture}
                    %TODO Images
                    \draw[line width = 0.5cm] (0,0) -- (4,0);
                    \draw[->] (4.3,-0.4) -- node[sloped,below] {Bilinear} (7,-2);
                    \draw[->] (4.3,0.4) -- node[sloped,above] {Nearest neighbour} (7,2);
                \end{tikzpicture}
            \end{center}


\section{Histogramme und deren Anwendungen}

    \subsection{Histogramme}
        Sei $u:\Omega \to F$ ein diskretes Bild, dann heißt die Abbildung
        \[H_u : F \to \N_0\]
        \[F \ni k \mapsto \# \{x \in \Omega | u(x) = k\}\]
        \mim{Histogramm} des Bildes $u$. Dieses gibt an, wie often die Farbe $k$ im Bild vorhanden ist.\\
        Damit gilt auch:
        \[\sum_{k \in F}H_u(k)=\abs{\Omega} \text{, also die Anzahl der Pixel}\]
        \begin{bemerkung*}
            Manchmal betrachtet man die relative Häufigkeit $\displaystyle \tilde H_u(k) = \frac{H_u(k)}{\abs{\Omega}}$.
        \end{bemerkung*}

        \pa{\large Beispiel:}

        \begin{center}
            \begin{tikzpicture}
                \draw (0,0) node {$\Omega=$};
                \draw[fill = black] (1.05,0.95) rectangle (1.95,0.05);
                \draw[fill = black] (2.05,0.95) rectangle (2.95,0.05);
                \draw[fill = black!30] (1.05,-0.05) rectangle (1.95,-0.95);
                \draw[fill = white] (2.05,-0.05) rectangle (2.95,-0.95);
                \draw (1,1) grid (3,-1);
                \draw[->] (3.25,0) -- (3.5,0);
                \draw[->] (4,-1.25) -- (8,-1.25) node[right] {$F$};
                \draw[->] (4,-1.25) -- (4,1.5);
                \tvmark{5}{-1.25}{black}
                \tvmark{6}{-1.25}{grey}
                \tvmark{7}{-1.25}{white}
                \thmark{4}{-0.25}{1}
                \thmark{4}{0.75}{2}
                \draw (5,0.75) node {\textbullet};
                \draw (6,-0.25) node {\textbullet};
                \draw (7,-0.25) node {\textbullet};
            \end{tikzpicture}
        \end{center}

        Für kontinuierliche Bilder wird  das allgemeinere Konzept von einem \mim{Maß} benötigt:
        \begin{center}
            \begin{tikzpicture}
                \draw (0,0) node[left] {$A \subset F, \ \mathcal H_u := \abs{u^{-1}(A)}$};
                \draw[->,text width = 4.5cm] (0,0.1) -- (0.75,0.75) node[right] {\small \underline{Diskretes Bild:}\\ Anzahl der Elemente in $u^{-1}(A)$};
                \draw[->,text width = 4.5cm] (0,-0.1) -- (0.75,-0.75) node[right] {\small \underline{Kontinuierliches Bild:}\\ Volumen von  $u^{-1}(A)$};
            \end{tikzpicture}
        \end{center}

        Zusammenhang zum vorherigen: $\displaystyle \mathcal H_u(A) = \sum_{k \in A} H_u(k)$.
        Man sagt dann, dass $U_u$ eine \mim{Dichte} zum Maß $\mathcal H_u$ sei. Diese kann auch im kontinuierlichen existieren:

        \begin{center}
            \begin{tikzpicture}
               \draw[->] (0,0) -- (8,0) node[right] {$F$};
               \draw[->] (0,0) -- (0,4);
               \draw[thick] plot [smooth, tension = 0.8] coordinates {(0,1) (2,2) (4,0.75) (5.5,2.5) (7,0)};
            \end{tikzpicture}
        \end{center}

    \subsection{Anwendung: Kontrastverbesserung}
        \pa{Problem \& Idee:} Falls das Bild nur einen kleinen Teil von $F$ nutzt, kann der Kontrast verbessert werden, indem man das Bild auf ganz $F$ verteilt.
        \begin{center}
            \begin{tikzpicture}
               \draw[->] (0,0) node[yshift=-4, xshift = -4] {0} -- (4.2,0) node[right] {$F$};
               \draw[->] (0,0) -- (0,2);
               \draw[thick] plot [smooth, tension = 0.8] coordinates {(1.5,0) (1.75,0.75) (2,0.5) (2.25,1.25) (2.5,0)};
               \tvmark{1.5}{0}{$k_{min}$}
               \tvmark{2.5}{0}{$k_{max}$}
               \draw[->,thick] (4.25,1) -- (4.75,1);
               \draw[->] (5,0) node[yshift=-4, xshift = -4] {0} -- (9.2,0) node[right] {$F$};
               \draw[->] (5,0) -- (5,2);
               \draw[thick,shift={(5,0)}] plot [smooth, tension = 0.8] coordinates {(0,0) (1,0.75) (2,0.5) (3,1.25) (4,0)};
               \tvmark{9}{0}{$N$}
            \end{tikzpicture}
        \end{center}

        \begin{center}
            \begin{tikzpicture}
                \draw[->] (0,0) node[below] {$\Omega$} to[bend left] node[above] {\small originales Bild} (1.8,0);
                \draw (2,0) node[below] {$F$};
                \draw[->] (2.2,0) to[bend left] node[above] {\small Zauberei $T$} (4,0) node[below] {$F$};
                \draw[->] (0,-0.5) to[bend right] node[below] {\small Bild mit besserem Kontrast} (4,-0.5);
                \draw[text width = 2.3cm] (5,0) node[right] {\small $v = T \circ u$\\ $v(k)=T(u(k)) $};
            \end{tikzpicture}
        \end{center}

        \pa{1. Idee, Kontrastdehnung:}\\
        $T$ ''lineare'' Abbildung, so dass $T(k_{min})=0$ und $T(k_{max})=N$:
        \[T(k) = \frac{k - k_{min}}{k_{max} - k_{min}} N \text{, Kontinuierlicher Farbraum}\]
        \[T(k) = \left[\frac{k - k_{min}}{k_{max} - k_{min}} N \right] \text{, Diskreter Farbraum}\]

        \underline{Beispiel:}

        \begin{center}
            \begin{tikzpicture}
                \draw[->] (0,0) node[yshift=-4, xshift = -4] {0} -- (6,0) node[right] {$F$};
                \draw[->] (0,0) -- (0,3);
                \tlbar{0.75}{0}{1}{20}{4}
                \tlbar{1.5}{0}{1.25}{40}{6}
                \tvnmark{2.25}{0}{...}
                \tlbar{3}{0}{1.75}{99}{10}
                \tlbar{3.4}{-0.3}{2.75}{100}{20}
                \tlbar{3.8}{0}{2.25}{101}{15}
                \tlbar{4.2}{-0.3}{1.125}{102}{5}
                \tvnmark{4.8}{0}{...}
                \tvmark{5.5}{0}{255}
                \draw[->, thick, text width = 2cm] (6.5,1.5) -- node[above] {\footnotesize Lineare Transformation $T$} (7.5,1.5);
                \draw[->] (8,0) node[yshift=-4, xshift = -4] {0} -- (13,0) node[right] {$F$};
                \draw[->] (8,0) -- (8,3);
                \tlbar{8.25}{0}{1}{3}{4}
                \tvnmark{8.75}{0}{...}
                \tlbar{9.25}{0}{1.75}{64}{6}
                \tvnmark{9.75}{0}{...}
                \tlbar{10.25}{0}{1.75}{243}{10}
                \tlbar{10.75}{-0.3}{2.75}{246}{20}
                \tlbar{11.25}{0}{2.25}{249}{15}
                \tlbar{11.75}{-0.3}{1.125}{252}{5}
                \tvmark{12.4}{0}{255}
                \draw[] (1.5,2.25) node[] {$\abs{\Omega}=60$};
                \draw (10.5,-0.8) node[below] {\footnotesize $k_{min} =19, \ k_{max}=103$};
            \end{tikzpicture}
        \end{center}

        \pa{2. Idee nicht-lineare Kontrastdehnung}\\
            Diesesmal setzen wir $\displaystyle T(k) =\left[ \frac{N}{\abs{\Omega}} \sum_{l=0}^{k}H_u(l) \right]$ für einen diskreten Farbraum und erhalten:

            \begin{center}
                \begin{tikzpicture}
                    \draw[->] (0,0) node[yshift=-4, xshift = -4] {0} -- (8,0) node[right] {$F$};
                    \draw[->] (0,0) -- (0,3);
                    \draw[decorate,decoration={brace,amplitude=2pt,mirror}] (0,-0.2) -- node[below] {\small $\frac{4}{60} N$} (1,-0.2);
                    \tlbar{1}{0}{1}{}{4}
                    \draw[decorate,decoration={brace,amplitude=2pt,mirror}] (1,-0.2) -- node[below] {\small $\frac{6}{60} N$} (2.5,-0.2);
                    \tlbar{2.5}{0}{1.25}{}{6}
                    \draw[decorate,decoration={brace,amplitude=2pt,mirror}] (2.5,-0.2) -- node[below] {\small $\frac{10}{60} N$} (4.5,-0.2);
                    \tlbar{4.5}{0}{1.75}{}{10}
                    \draw[decorate,decoration={brace,amplitude=2pt,mirror}] (4.5,-0.2) -- node[below] {\small $\frac{20}{60} N$} (7,-0.2);
                    \tlbar{7}{0}{2.5}{}{20}
                    \tvnmark{7.5}{0}{...}
                \end{tikzpicture}
            \end{center}
        $T$ lässt sich auch alternativ ausdrücken durch:
        \[ T(k)=\left[ \mathcal H_u(\{0,...,k\}) \right] \]
        Und somit folgt dass für den kontinuierlichen Fall $T$ durch
        \[ T(k) = \frac{N}{\abs{\Omega}} \mathcal H_u((0,k))\]
        definiert werden kann. Allgemein heißt der Prozess \mim{Histogramm - equalization}.
    \subsection{Anwendung: SW-Konvertierung}
        Aufgabe: Graustufenbild $\to$ SW-Bild.\\
        Nützlich etwa bei Objekterkennung/Segmentierung.

        Idee: Das Histogramm an einem gewissen \mim{Schwellenwert} $t$ spalten:

        \begin{center}
            \begin{tikzpicture}
                \fill[black!80](0.1,0) rectangle (3.5,3);
                \fill [white] (0.05,0) -- plot [smooth, tension = 0.7] coordinates {(1,0) (2,1.7) (3,1) (4.5,2) (7,0)} -- (7.5,3.5) -- (0.1,3.5)  -- cycle;
                \draw[->] (0,0) node[below] {dunkel} -- (8,0) node[below] {hell} node[right] {$F$};
                \draw[->] (0,0) -- (0,4);
                \draw[thick] plot [smooth, tension = 0.7] coordinates {(1,0) (2,1.7) (3,1) (4.5,2) (7,0)};
                \draw[densely dotted,thick] (3.5,-0.2) node[below] {$t$} -- (3.5,3);
            \end{tikzpicture}
        \end{center}
        Also setze nun für $t \in F$:
        \[\text{schwarz} = \left\{k \in F | k \leq t \right\} \]
        \[\text{weiß} = \left\{k \in F | k >t \right\}\]
        Graustufenbild $u$ $\longrightarrow$ schwarz/weiß Bild $\tilde u$:

        \[\tilde u(x) = \begin{cases}
            0, \ u(x) \in \text{schwarz}\\
            1, \ u(x) \in \text{weiß}
        \end{cases} \Rightarrow \ \tilde F = \{0,1\}\]

        \pa{Methoden um diesen Schwellenwert zu wählen:}

        \begin{enumerate}
            \item \mim{Shape based Methods}:\\
            Falls das Histogramm von $u$ \mim{bimodal} ist, also die Form:
            \begin{center}
                \begin{tikzpicture}
                    \draw[->] (0,0) -- (4,0) node[right] {$F$};
                    \draw[->] (0,0) -- (0,2);
                    \draw[thick] plot [smooth, tension = 0.6] coordinates {(0.5,0) (1.3,1) (2,0.3) (2.5,1.5) (3,0)};
                    \draw (1.3,-0.1) node[below] {\small $k_{max 1}$} -- (1.3,1);
                    \draw (2,-0.35) node[below] {\small $k_{min}$} -- (2,0.3);
                    \draw (2.5,-0.1) node[below] {\small $k_{max 2}$} -- (2.5,1.5);
                \end{tikzpicture}
            \end{center}
            hat, dann wähle:
            \[t:=k_{min}\]
            \[\text{oder} \ t:= \frac{k_{max 1} + k_{max 2}}{2}\]
            \item \mim{Otsu's Verfahren} (1979):\\
            Vorher einige Definitionen.\\
            Die \mim{Masse}:
            \[m_{\text{schwarz}} := \sum_{k \in \text{schwarz}} H_u(k)\]
            \[m_{\text{weiß}} := \sum_{k \in \text{weiß}} H_u(k)\]
            Der \mim{Mittlewert}:
            \[\mu_{\text{schwarz}} := \frac{\displaystyle \sum_{k_ \in \text{schwarz}} k \C H_u(k) }{\displaystyle \sum_{k_ \in \text{schwarz}}H_u(k)} = \frac{\displaystyle \sum_{k_ \in \text{schwarz}} k \C H_u(k) }{m_{\text{schwarz}}}\]
            \[\mu_{\text{weiß}} := \frac{\displaystyle \sum_{k_ \in \text{weiß}} k \C H_u(k) }{\displaystyle \sum_{k_ \in \text{weiß}}H_u(k)} = \frac{\displaystyle \sum_{k_ \in \text{weiß}} k \C H_u(k) }{m_{\text{weiß}}}\]
            Die \mim{Varianz}:
            \[\sigma^2_{\text{schwarz}} = \sum_{k \in \text{schwarz}} (k - \mu_{\text{schwarz}})^2 \C H_u(k)\]
            \[\sigma^2_{\text{weiß}} = \sum_{k \in \text{weiß}} (k - \mu_{\text{weiß}})^2 \C H_u(k)\]
            Nun lautet Otsu's Methode: $\sigma^2_{\text{schwarz}} + \sigma^2_{\text{weiß}} \overset{t}{\rightarrow} \text{min}$.
            \item \mim{Median}:\\
            Wähle $t$ so dass $m_{\text{schwarz}} = m_{\text{weiß}}$.
            \item \mim{Isodata Algorithmus} (1970s):\\
            Wähle t so, dass $\displaystyle t = \frac{\mu_{\text{schwarz}} - \mu_{\text{weiß}}}{2} =: f(t)$.\\
            Diese Gleichung ist bereits eine \mim{Fixpunktgleichung} und eine Lösung kann, etwa mit einer \mim{Fixpunktiteration} approximiert werden, das heißt $t_{n+1} := f(t_n)$.
            \begin{center}
                \begin{tikzpicture}%TODO fixpunkt iteration in Grafik?
                    \draw[->] (0,0) node[below] {0} -- (0,5) node[above] {y};
                    \draw[->] (0,0) -- (5,0)node[right] {x};
                    \draw (-0.1,4.5) node[left] {N} -- (4.5,4.5);
                    \draw (4.5,-0.1) node[below] {N} -- (4.5,4.5);
                    \draw[name path = P1] (-0.1,-0.1) -- node[below,sloped, pos = 0.7] {$y=x$} (4.6,4.6);
                    \coordinate (A) at (0.5,0.5);
                    \draw[thick, name path = P2] plot [smooth, tension = 0.8] coordinates {(0,0.5) (0.5,0.7) (1,1.1) (2, 1.7) (2.8, 3.1) (4.5,4.25)};
                    \draw (3,3.7) node[rotate = 35] {$y = f(x)$};
                    \fill[name intersections={of=P1 and P2}]
                    (intersection-1) circle (1.5pt) node {}
                    (intersection-2) circle (1.5pt) node {}
                    (intersection-3) circle (1.5pt) node {};
                \end{tikzpicture}
            \end{center}
        \end{enumerate}

        \pa{Matlab code}:\\
        \begin{lstlisting}
u=imread('liftingbody.png');
t=greythresh(u);%uses Otsu's method
v=im2bn(u,t);
imshow(v);
        \end{lstlisting}

        Einige dieser Verfahren können auch erweitert werden, so dass ein Graustufenbild nicht nur in zwei, sondern in $M$ Farben zerlegt werden kann. Im allgemeinen werden dann $M-1$ thresholds benötigt.

        \begin{enumerate}
            \item \pa{Shape based}:\\
            \begin{center}
                \begin{tikzpicture}
                    \draw (0,0) -- (0,2);
                    \draw (0,0) -- (6,0);
                    \draw[thick, name path = P2] plot [smooth, tension = 0.7] coordinates {(0.5,0) (1,1) (1.5,0.2) (2, 0.9) (2.5, 0.5) (3.25,1.5) (4,0.5) (4.5,1.2) (5,0)};
                    \draw (1,-0.1) node[below] {$t_1$} -- (1,1);
                    \draw (2,-0.1) node[below] {$t_2$} -- (2,0.9);
                    \draw (3.25,-0.1) node[below] {$t_3$} -- (3.25,1.5);
                    \draw (4.5,-0.1) node[below] {$t_4$} -- (4.5,1.2);
                \end{tikzpicture}
            \end{center}
            \ \\
            \item \pa{Otsu's Verfahren}:\\
            Farbklassen:
            \[ F_1 = \{ k : k \leq t_1\}\]
            \[ F_2 = \{ k : t_1 < k \leq t_2\}\]
            \[\vdots\]
            \[ F_M = \{ k : t_{M-1} < k \}\]
            Und wie zuvor: $\sigma_1^2 + ... + \sigma_M^2 \to $ min
            \item \pa{Median}:\\
            Zerteile $F$ in M Quantile gleicher Masse.
            \item \pa{Isodata}:\\
            Hierzu existiert keine Bekannte Verallgemeinerung auf $M$ Farbklassen.
        \end{enumerate}

        \pa{Matlab code}:\\
        \begin{lstlisting}
u=imread('Circles Bright Dark.png');
t=multithresh(u,M-1);
v=imquantize(u,t);
w=label2rgb(u,t);
imshow(w);
        \end{lstlisting}

\section{Einfache \Index{Morphographische Operationen}}
        S/W Bild:
        \begin{center}
            \begin{tikzpicture}
                \draw (0,0) node[left] {$A=$} node[right] {\includegraphics[scale = 0.2]{Bild1.png}};
            \end{tikzpicture}
        \end{center}
        \mim{Strukturelement}:
        \begin{center}
            \begin{tikzpicture}[scale = 2]
                    \draw[->] (0,-1) -- (0,0) node {\tiny \textbullet} -- (0,1);
                    \draw[->] (-1,0) -- (0,0) -- (1,0);
                    \fill[black!30, opacity = 0.6] (0,0) circle (0.5);
                    \draw (0,0) circle (0.5);
                    \draw (-1.1,0) node[left] {$B=$};
            \end{tikzpicture}
        \end{center}

    \subsection{Verknüpfungen von A und B}
        \[A+B := \{a + b : a \in A, b \in B\}\]
        Diese wird \mim{dilation} genannt.\\
        Anschaulich wird an jeden schwarzen Punkt des Bildes $A$ das Struktur element $B$ gelegt.
        \begin{center}
            \begin{tikzpicture}
                \draw (0,0) node[left] {$A+B=$} node[right] {\includegraphics[scale = 0.2]{Bild1dil.png}};
            \end{tikzpicture}
        \end{center}
        Bild erzeugt in Matlab durch:\\
        \begin{lstlisting}
I=imread('Bild1.png');
se=strel('disk',40,8);
I2=imcomplement(imdilate(imcomplement(I),se));%Es wird das Komplement des Bildes gebildet, damit das Strukturelement auf den schwarzen bereich angewendet wird
imshow(I2);
        \end{lstlisting}

        \[A-B := \{a : a + B \subset A\}\]
        Diese wird \mim{erosion} genannt.\\
        Anschaulich werden die schwarzen Bereiche des Bildes gesucht, in die das Strukturelement hinein passt.
        \begin{center}
            \begin{tikzpicture}
                \draw (0,0) node[left] {$A-B=$} node[right] {\includegraphics[scale = 0.2]{Bild1erode.png}};
            \end{tikzpicture}
        \end{center}

        Bild erzeugt in Matlab durch:\\
        \begin{lstlisting}
I=imread('Bild1.png');
se=strel('disk',20,8);
I2=imcomplement(imerode(imcomplement(I),se));
imshow(I2);
        \end{lstlisting}

        Es ist schnell zu erkennen das $A \neq (A+B)-B$, deshalb wird eine neue Operation eingeführt:
        \[A \bullet B := (A+B)-B\]
        Dieses wird \mim{schließen} genannt und wird etwa genutzt um Löcher, z.b. Rauschen, in einem Bild zu entfernen. Im Beispiel Bild ist zu sehen, dass das obere Fenster nicht mehr vorhanden ist.
        \begin{center}
            \begin{tikzpicture}
                \draw (0,0) node[left] {$A \bullet B=$} node[right] {\includegraphics[scale = 0.2]{Bild1close.png}};
            \end{tikzpicture}
        \end{center}

        Bild erzeugt in Matlab durch:\\
        \begin{lstlisting}
I=imread('Bild1.png');
se=strel('disk',20,8);
I2=imcomplement(imdilate(imcomplement(I),se));
I3=imcomplement(imerode(imcomplement(I2),se));
imshow(I3);
        \end{lstlisting}

        Es existiert auch die Umgekehrt Operation:
        \[ A \circ B := (A-B)+B\]
        Diese wird \mim{öffnen} genannt.

        Diesmal mit einem neuen Beispiel:

        \begin{center}
            \begin{tikzpicture}
                \draw (0,0) node[left] {$A=$} node[right] {\includegraphics[scale = 0.2]{Bild2.png}};
            \end{tikzpicture}
        \end{center}

        \begin{center}
            \begin{tikzpicture}
                \draw[->] (0,-1) -- (0,0) node {\tiny \textbullet} -- (0,1);
                \draw[->] (-1,0) -- (0,0) -- (1,0);
                \fill[black!60, opacity = 0.6] (-0.05,-0.7) rectangle (0.05,0.7);
                \draw (-0.05,-0.7) rectangle (0.05,0.7);
                \draw (-1.1,0) node[left] {$B=$};
            \end{tikzpicture}
        \end{center}

        \begin{center}
            \begin{tikzpicture}
                \draw (0,0) node[left] {$A \circ B =$} node[right] {\includegraphics[scale = 0.2]{Bild2open.png}};
            \end{tikzpicture}
        \end{center}

        Bild erzeugt in Matlab durch:\\
        \begin{lstlisting}
I=imread('Bild2.png');
se=strel('line',10,90);
I2=imcomplement(imerode(imcomplement(I),se));
I3=imcomplement(imerode(imcomplement(I2),se));
imshow(I3);
        \end{lstlisting}

\section{Entrauschen: Filter \& Co.}
    \subsection{Rauschen}
        \mim{Rauschen}: Ungewollte Störungen in einem Bild
        \begin{enumerate}[label = \textbullet]
            \item punktweise
            \item zufällig
            \item unabhängig
            \item additiv (bei multiplikativem Rauschen $log$ anwenden)
        \end{enumerate}

        Notation:
        \begin{center}
            \begin{tikzpicture}
                \draw (0,0) rectangle (2,2);
                \draw (1,0) node[below] {\small Sauberes Bild};
                \draw (1,1) node[] {\LARGE $f_0$};
                \draw[->] (2.1,1) -- node[above] {\small $+$ Rauschen} (3.9,1);
                \draw (4,0) rectangle (6,2);
                \draw (5,0) node[below] {\small Gestörtes Bild};
                \draw (5,1) node[] {\LARGE $f$};
                \draw[->] (6.1,1) -- node[above] {\small Entrauschen} (7.9,1);
                \draw (8,0) rectangle (10,2);
                \draw (9,0) node[below] {\small Resultat};
                \draw (9,1) node[] {\LARGE $u$};
            \end{tikzpicture}
        \end{center}

        Wie gut das entrauschte Bild $u$ das saubere Bild $f_0$ beschreibt wird durch Normen gemessen.
	    \begin{align*}
        &\norm{f-f_0}, \text{Rauschen}\\
        &\norm{u-f_0}, \text{\mim{Absoluter Fehler}}\\
        &\frac{\norm{u-f_o}}{\norm{f-f_0}}, \text{\mim{Relativer Fehler} im Vergleich zum Rauschen}\\
        &\frac{\norm{u-f_o}}{\norm{f_0}}, \text{Relativer Fehler im Vergleich zum Signal}
        \end{align*}

        Typischerweise ist die gewählte Norm:
        \[\norm{f} = \norm{f}_2 = \sqrt{\int_{\Omega} \abs{f(x)}^2 dx}\]
        oder im diskreten:
        \[\norm{f}_2=\sqrt{\sum_{x \in \Omega} \abs{f(x)}^2}\]

        Eng verwandt ist die \mim{Signal to noise ratio} (SNR):
        \[log(\underbrace{\frac{\norm{f_0}_2}{\norm{u-f_0}_2}}_{\in \ [1,\infty)}) \in [0,+\infty), \text{ wobei $0$ schlecht und $+\infty$ gut ist.}\]
    \subsection{Glättungsfilter}
        Grundidee: (zur Vereinfachung in 1D)
        \begin{center}
            \begin{tikzpicture}
                \draw (0,0) node[left] {$f_0$:};
                \draw[thick] plot [smooth, tension = 0.7] coordinates {(0,0) (1.75,1.5) (3.5,0.5) (6,2) (9,0) (11,0.5)};
                \draw[->,thick] (-0.3,-0.5) -- node[left] {Rauschen} (-0.3,-2);
                \draw (0,-2.5) node[left] {$f$:};
                \draw[thick, name path = P2,shift = {(0,-2.5)}] plot [smooth, tension = 0.7] coordinates {(0,0) (1.75,1.5) (3.5,0.5) (6,2) (9,0) (11,0.5)};
                \draw[name path = P1, draw = none] (0,-1.1) -- (3,-1.1);
                \draw[name intersections={of = P1 and P2},red,thick]
                (intersection-1) edge[bend right] ++(0.3,0.5) ++(0.3,0.5) edge[bend right] (intersection-2);
                \draw[name path = P3,draw = none] (3,-2.3) -- (5,-1.2);
                \draw[name intersections={of = P3 and P2},red,thick]
                (intersection-1) edge[bend left] ++(0.3,-0.3) ++(0.3,-0.3) edge[bend left] (intersection-2);
                \draw[] (2,-0.9) -- ++(0.5,-0.1) node[right] {\small Störungen} (3.5,-2.1) -- ++(-0.5,0.9);
                \draw (0,-5) node[left] {$f$:};
                \draw[thick, name path = P4,shift = {(0,-5)}] plot [smooth, tension = 0.7] coordinates {(0,0) (1.75,1.5) (3.5,0.5) (6,2) (9,0) (11,0.5)};
                \draw[name path = P5, shift = {(0,-2.5)}, draw = none] (0,-1.1) -- (3,-1.1);
                \draw[name intersections={of = P4 and P5},red,thick]
                (intersection-1) edge[bend right] node[black,pos=0] {\small \textbullet} ++(0.3,0.5) ++(0.3,0.5) edge[bend right] node[black,pos=1] {\small \textbullet}  node[black,pos=0] {\small \textbullet} (intersection-2);
                \draw[name path = P6, shift = {(0,-2.5)}, draw = none] (3,-2.3) -- (5,-1.2);
                \draw[name intersections={of = P4 and P6},red,thick]
                (intersection-1) edge[bend left] node[black,pos=0] {\small \textbullet} ++(0.3,-0.3) ++(0.3,-0.3) edge[bend left] node[black,pos=1] {\small \textbullet}  node[black,pos=0] {\small \textbullet} (intersection-2);
                \draw[decorate,decoration={brace,amplitude=2pt,mirror}] (1.4,-3.7) -- (2.2,-3.7);
                \draw[thick] (1.8,-3.8) -- (1.8,-4.3) node[below] {\small \framebox{Mittelwert}};
                \draw[->,thick,double] (1.6,-5) -- (0.4,-5);
                \draw[->,thick,double] (2,-5) -- (3.2,-5);
                \draw[->,thick] (1.8,-5.1) -- (1.8,-5.7);
                \draw (0,-7.5) node[left] {$u$:};
                \draw[->,thick] (-0.3,-5.5) -- node[left] {Entrauschen} (-0.3,-7);
                \draw[thick, name path = P7,shift = {(0,-7.5)}] plot [smooth, tension = 0.7] coordinates {(0,0) (1.75,1.5) (3.5,0.5) (6,2) (9,0) (11,0.5)};
                \draw[name path = P8, shift = {(0,-5)}, draw = none] (0,-1.2) -- (3,-1.2);
                \draw[name intersections={of = P7 and P8},red,thick]
                plot [smooth,tension=0.7] coordinates {(intersection-1) ($(intersection-1) + (0.5,0.35)$) (intersection-2)};
                \draw[name path = P9, shift = {(0,-5)}, draw = none] (3,-2.25) -- (5,-1.15);
                \draw[name intersections={of = P7 and P9},red,thick]
                plot [smooth,tension=0.7] coordinates {(intersection-1) ($(intersection-1) + (0.45,0.05)$) (intersection-2)};
            \end{tikzpicture}
        \end{center}

        \begin{equation} \label{eq:5.1}
            u(k):=\alpha \cdot f(k-1) + \beta \cdot f(k) + \gamma \cdot f(k+1)
        \end{equation}
        wobei:
        \begin{equation} \label{eq:5.2}
            \alpha + \beta + \gamma = 1
        \end{equation}

        Schematisch bedeutet \eqref{eq:5.1}:

        \begin{center}
            \begin{tikzpicture}
                \draw (0,-0.5) node[left] {\large f:};
                \draw[thick] (0.5,0) grid (6.5,-1);
                \draw (0.5,-0.5) node {\LARGE ...};
                \draw (6.5,-0.5) node {\LARGE ...};
                \draw (1.5,0) node[above] {\dots};
                \draw (2.5,0) node[above] {k-1};
                \draw (3.5,0) node[above] {k};
                \draw (4.5,0) node[above] {k+1};
                \draw (5.5,0) node[above] {\dots};
                \draw (0.5,-2.5) node {\LARGE ...};
                \draw (6.5,-2.5) node {\LARGE ...};

                \draw[->,red] (1.6,-1.1) -- node[left] {\tiny $\alpha$} (2.4,-1.9);
                \draw[->,red] (2.6,-1.1) -- node[left] {\tiny $\alpha$} (3.4,-1.9);
                \draw[->,red] (3.6,-1.1) -- node[left] {\tiny $\alpha$} (4.4,-1.9);
                \draw[->,black!40!green] (2.5,-1.1) -- node[left, pos = 0.3] {\tiny $\beta$} (2.5,-1.9);
                \draw[->,black!40!green] (3.5,-1.1) -- node[left, pos = 0.3] {\tiny $\beta$} (3.5,-1.9);
                \draw[->,black!40!green] (4.5,-1.1) -- node[left, pos = 0.3] {\tiny $\beta$} (4.5,-1.9);
                \draw[->,black!40!blue] (3.4,-1.1) -- node[right, pos = 0.75] {\tiny $\gamma$} (2.6,-1.9);
                \draw[->,black!40!blue] (4.4,-1.1) -- node[right, pos = 0.75] {\tiny $\gamma$} (3.6,-1.9);
                \draw[->,black!40!blue] (5.4,-1.1) -- node[right, pos = 0.75] {\tiny $\gamma$} (4.6,-1.9);

                \draw (0,-2.5) node[left] {\large u:};
                \draw[thick] (0.5,-2) grid (6.5,-3);
            \end{tikzpicture}
        \end{center}

        \begin{center}
            \begin{tikzpicture}
                \draw (-0.6,-0.5) node {\LARGE ...};
                \draw (1.6,-0.5) node {\LARGE ...};
                \draw (-1.6,-2.5) node {\LARGE ...};
                \draw (2.6,-2.5) node {\LARGE ...};
                \draw (0.5,0) node[above] {k};
                \draw[thick] (-0.6,0) grid (1.6,-1);
                \draw[->,red] (0.4,-1.1) -- node[left] {$\alpha$} (-0.5,-1.9);
                \draw[->,black!40!green] (0.5,-1.1) -- node[left] {$\beta$} (0.5,-1.9);
                \draw[->,black!40!blue] (0.6,-1.1) -- node[right] {$\gamma$} (1.5,-1.9);
                \draw[thick] (-1.6,-2) grid (2.6,-3);
            \end{tikzpicture}
        \end{center}

        Durch \eqref{eq:5.1} ist eine Abbildung $f \mapsto u$ gegeben, wir schreiben kurz:
        \[u = m \boxast f, \ \text{dieses wird \mim{Korrelation} genannt.}\]
        mit:
        \begin{equation}\label{eq:5.3}
            \framebox{$\displaystyle (m \boxast f)(k) = \sum_{i \in supp(m)} m(i) f(k+i)$}
        \end{equation}
        und:
        \begin{center}
            \begin{tikzpicture}
                \draw (0,0.5) node[left] {m=};
                \draw[thick] (0.5,0) grid (4.5,1);
                \draw (0.5,0.5) node {\large \dots};
                \draw (1.5,0.5) node {\large $\alpha$};
                \draw (2.5,0.5) node {\large $\beta$};
                \draw (3.5,0.5) node {\large $\gamma$};
                \draw (4.5,0.5) node {\large \dots};
                \draw (0.5,1) node[above] {\large \dots};
                \draw (1.5,1) node[above] {\large -1};
                \draw (2.5,1) node[above] {\large 0};
                \draw (3.5,1) node[above] {\large 1};
                \draw (4.5,1) node[above] {\large \dots};
                \draw (5,0.5) node[right] {gennant \mim{Maske}.};
            \end{tikzpicture}
        \end{center}

        Setzt man nun $j:= k + i$ in \eqref{eq:5.1}, so ist $i=j-k$, d.h.
        \begin{equation}\label{eq:5.4}
            \framebox{$\displaystyle (m \boxast f)(k) = \sum_{i \in supp(m)} m(j-k) f(j)$}
        \end{equation}

        Um die Abbildung auf den Rand anzuwenden wird das Bild gespiegel, in 1D:
        \begin{center}
            \begin{tikzpicture}
                \draw[step = 0.5] (0,0) grid (2.2,-0.5);
                \draw (2.5,-0.25) node {\large ...};
                \draw[step = 0.5] (2.8,0) grid (5,-0.5);
                \draw[thick] (0,0.1) -- (0,-0.6);
                \draw[thick] (5,0.1) -- (5,-0.6);
                \draw[step = 0.5, dotted] (0,0) grid (-1.2,-0.5);
                \draw[step = 0.5, dotted] (5,0) grid (6.2,-0.5);
                \draw[] (0.25,0.1) edge[bend right = 60, ->] (-0.25,0.1);
                \draw[] (0.75,0.1) edge[bend right = 60, ->] node[above] {\small spiegeln} (-0.75,0.1);
                \draw[] (4.75,0.1) edge[bend left = 60, ->] (5.25,0.1);
                \draw[] (4.25,0.1) edge[bend left = 60, ->] node[above] {\small spiegeln} (5.75,0.1);
                \draw[step = 0.5] (0,-1) grid (2.2,-1.5);
                \draw (2.5,-1.25) node {\large ...};
                \draw[step = 0.5] (2.8,-1) grid (5,-1.5);
                \draw[->] (0.25,-0.55) -- (0.25,-0.95);
                \draw[->] (-0.25,-0.55) -- (0.15,-0.95);
                \draw[->] (0.75,-0.55) -- (0.35,-0.95);
                \draw[->] (4.75,-0.55) -- (4.75,-0.95);
                \draw[->] (4.25,-0.55) -- (4.65,-0.95);
                \draw[->] (5.25,-0.55) -- (4.85,-0.95);
            \end{tikzpicture}
        \end{center}

        in 2D:

        \begin{center}
            \begin{tikzpicture}
                \draw[step =2, thick] (0,0) grid (6,6);
                \draw (3,3) node {\Huge P};
                \draw (1,1) node[rotate = 180] {\Huge P};
                \draw (5,1) node[rotate = 180] {\Huge P};
                \draw (1,5) node[rotate = 180] {\Huge P};
                \draw (5,5) node[rotate = 180] {\Huge P};
                \draw (3,5) node[yscale=-1,xscale=1] {\Huge P};
                \draw (5,3) node[rotate = 180,yscale=-1,xscale=1] {\Huge P};
                \draw (1,3) node[rotate = 180,yscale=-1,xscale=1] {\Huge P};
                \draw (3,1) node[yscale=-1,xscale=1] {\Huge P};
            \end{tikzpicture}
        \end{center}

        Formel \eqref{eq:5.4} erinnert an die Formel der \mim{Faltung}:
        \begin{equation}\label{eq:5.5}
            \framebox{$\displaystyle (g * f)(k) = \sum_{j \in \Z} g(\underbrace{k-j}_{\text{Anders als \eqref{eq:5.4}}}) \cdot f(j)$}
        \end{equation}

        Setzt man also $g(i) := m(-i) =: \tilde m(i)$, was einer Spieglung der Maske entspricht, dann ist
        \[m \boxast f = g * f = \tilde m * f\]

        Eigenschaften der Faltung:
        \begin{enumerate}[label=\framebox{\arabic *}]
            \item $(f * g) * h = f * (g* h)$, Assoziativität
            \item $f*g=g*f$, Kommutativität
            \item $\tilde f * \tilde g = \widetilde{f * g}$, Kompatibilität mit Spiegelung
        \end{enumerate}
        Eigenschaften der Korrelation:
        \begin{enumerate}[label=\framebox{\arabic *'}]
            \item $f \boxast (g \boxast h) = \tilde f * ( \tilde g* h) \overset{\framebox{\small 1}}{=} ( \tilde f * \tilde g) * h \overset{\framebox{\small 3}}{=} (\widetilde{f * g}) * h = (f * g) \boxast h \neq (f \boxast g) \boxast h$, nicht assoziativ!
            \item $f \boxast g = \tilde f * g \overset{\framebox{\small 2}}{=} g * \tilde f = \tilde{\tilde g} * \tilde f \overset{\framebox{\small 3}}{=} \widetilde{(\tilde g * f)} = \widetilde{g \boxast f} \neq g \boxast f$, nicht kommutativ!
            \item $\tilde f \boxast \tilde g = \tilde{\tilde f} * \tilde g \overset{\framebox{\small 3}}{=} \widetilde{(\tilde f * g)} = \widetilde{f \boxast g}$, Kompatibilität mit Spiegelung
        \end{enumerate}

	$\boxast$ und $*$ definiert man auf: $\ell^1(\Z^d):=\{f=(f_i)_{i \in \Z^d} : \underbrace{\sum_{i \in \Z^d}\abs{f_i}}_{:=\norm{f}_1} < \infty\}$

	Man kann zeigen (Übung): $f,g \in \ell^1 \Rightarrow f * g \in \ell^1$ und $\norm{f * g}_1 \leq \norm{f}_1 \C \norm{g}_1$.
    Wobei oft die Gleichheit gilt.

    Alles gilt auch in der Kontinuierlichen Version:
    \[L^1(\R^d) := \{f:\R^d \to \R | \underbrace{\int_{\R^d}\abs{f} dx}_{:=\norm{f}_1} < \infty\}\]
    \[f,g \in L^1(\R^d): (g*f)(x)=\int_{\R^d}g(x-y)f(y) dy, \ y,x \in \R^d\]

    Beispiel für den kontinueirlichen Fall:
    \begin{center}
        \begin{tikzpicture}
            \draw[dotted] (0,1) node[left] {$\frac{1}{2a}$} -- (8,1);
            \draw (0,0) node[left] {$g:$} -- (2.5,0) -- (2.5,1) -- (5.5,1) -- (5.5,0) -- (8,0);
            \draw (2.5,0) node[below] {\small $-a$};
            \draw[dotted] (4,0) node[below] {\small $0$} -- (4,1.5);
            \draw (5.5,0) node[below] {\small $-a$};
        \end{tikzpicture}
    \end{center}
    Hierbei gilt $\displaystyle \int_{\R} g(x) dx = 1$

    \begin{center}
        \begin{tikzpicture}
            \draw[thick, name path = P2] node [left] {$f:$}plot [smooth, tension = 0.45] coordinates {(0,0) (1.2,1.5) (2.1,0.3) (3,2) (4.3,1) (5.3,1.6) (6.6,0.5) (7.5,1) (8,0)};
        \end{tikzpicture}
    \end{center}

    $g \boxast f = $ \mim{gleitendes Mittel}.\\

        \begin{tikzpicture}
            \draw[dotted] (0,1) node[left] {$\frac{1}{2a}$} -- (8,1);
            \draw (0,0) node[left] {$g \boxast g = \tilde g * g = g * g=$} -- (1,0) -- (4,1) -- (7,0) -- (8,0);
            \draw (1,0) node[below] {\small $-2a$};
            \draw[dotted] (4,0) node[below] {\small $0$} -- (4,1.5);
            \draw (7,0) node[below] {\small $-2a$};
        \end{tikzpicture}

    Weitere Eigenschaften der Faltung:\\
    Für alle $f,g \in L^1$ or $\ell ^1$
    \begin{equation*}
        \left.\begin{aligned}
            (g_1 + g_2) * f = (g_1 * f) + (g_2 * g)\\
            (\alpha g) * f = \alpha (g * f)
        \end{aligned}\right\}=\text{Linearität}
    \end{equation*}
    Somit ist:
    \[g \mapsto f * g\]
    ein linearer Operator.

    Formt $\ell^1$ bzw. $L^1$ eine Algebra mit neutralem Element $\delta$?

    $\ell^1$?:
    \begin{center}
        \begin{tikzpicture}
            \draw (0,-0.5) node[left] {\large $\delta$:};
            \draw[thick] (0.5,0) grid (6.5,-1);
            \draw (0.5,-0.5) node {\LARGE ...};
            \draw (6.5,-0.5) node {\LARGE ...};
            \draw (1.5,-0.5) node {\LARGE 0};
            \draw (2.5,-0.5) node {\LARGE 0};
            \draw (3.5,-0.5) node {\LARGE 1};
            \draw (4.5,-0.5) node {\LARGE 0};
            \draw (5.5,-0.5) node {\LARGE 0};
            \draw[->] (3.5,-1.4) node[below] {Pos $0$} -- (3.5,-1.1);
        \end{tikzpicture}
    \end{center}
    Ja!

    $L^1$?:
    Für ein solches Element muss gelten:\\
    $\forall f \in L^1 : d * f = f$\\
    $\forall x \in \R :\displaystyle \int_{\R^d} \underbrace{\delta(x-y)}_{=0 \forall x \neq y} f(y) dy = f(x)$

    Diese Funktion wird \mim{Dirac-Impuls} genannt ist aber kein Element von $L^1$.

    \pa{Nun zu Masken in 2D:}

    \begin{equation*}
        u = m \boxast f \text{ mit } m= \raisebox{-0.665cm}{\begin{tikzpicture}
            \draw[step = 0.5] (0,0) grid (0.5,1.5);
            \draw[step = 0.5] (-0.5,1) grid (1,0.5);
            \draw (0.25,1.25) node {$\alpha$};
            \draw (0.25,0.75) node {$\gamma$};
            \draw (0.25,0.25) node {$\epsilon$};
            \draw (-0.25,0.75) node {$\beta$};
            \draw (0.75,0.75) node {$\delta$};
        \end{tikzpicture}}
    \end{equation*}
    wobei $\alpha + \beta +\gamma +\delta + \epsilon = 1$\\
    Kurzschreibweise: $u_{ij}:=u(x)$ wobei $x = \begin{pmatrix}i\\j\end{pmatrix} \in \Z^2$, analog für $f_{ij}$.

    \[\Rightarrow u_{ij} = \alpha f_{i-1,j} + \beta f_{i,j-i} + \gamma f_{ij} + \delta f_{i,j+1} + \epsilon f_{i+1,j}\]

    \begin{equation*}
        u = m \boxast f = \tilde m * f \text{ mit } \tilde m = \raisebox{-0.665cm}{\begin{tikzpicture}
            \draw[step = 0.5] (0,0) grid (0.5,1.5);
            \draw[step = 0.5] (-0.5,1) grid (1,0.5);
            \draw (0.25,1.25) node {$\epsilon$};
            \draw (0.25,0.75) node {$\gamma$};
            \draw (0.25,0.25) node {$\alpha$};
            \draw (-0.25,0.75) node {$\delta$};
            \draw (0.75,0.75) node {$\beta$};
        \end{tikzpicture}}
    \end{equation*}

    \pa{Symmetrischer Fall:}

    \begin{equation*}
    \tilde m = \raisebox{-0.665cm}{\begin{tikzpicture}
            \draw[step = 0.5] (0,0) grid (0.5,1.5);
            \draw[step = 0.5] (-0.5,1) grid (1,0.5);
            \draw (0.25,1.25) node {$\alpha$};
            \draw (0.25,0.75) node {$\gamma$};
            \draw (0.25,0.25) node {$\alpha$};
            \draw (-0.25,0.75) node {$\alpha$};
            \draw (0.75,0.75) node {$\alpha$};
        \end{tikzpicture}} \text{ mit } \gamma = 1 - 4 \alpha
    \end{equation*}

    \begin{equation}
        u_{ij} = (1 - 4 \alpha)f_{ij} + \alpha(f_{i-1,j} + f_{i,j-1} + f_{i,j+1} + f_{i+1,j})
    \end{equation}

    \begin{equation*}
            \text{Erinnerung: } \raisebox{-1.2cm}{\begin{tikzpicture}[scale=0.9]
                \draw (0,0) rectangle (2,2);
                \draw (1,0) node[below] {\small Sauberes Bild};
                \draw (1,1) node[] {\LARGE $f_0$};
                \draw[->] (2.1,1) -- node[above] {\small $+$ Rauschen} (3.9,1);
                \draw (4,0) rectangle (6,2);
                \draw (5,0) node[below] {\small Gestörtes Bild};
                \draw (5,1) node[] {\LARGE $f$};
                \draw[->] (6.1,1) -- node[above] {\small Entrauschen} (7.9,1);
                \draw (8,0) rectangle (10,2);
                \draw (9,0) node[below] {\small Resultat};
                \draw (9,1) node[] {\LARGE $u$};
            \end{tikzpicture}}
    \end{equation*}

    Annahme: $f_{ij} = f_{ij} +r_{ij}$ mit $r_{ij} \sim N(0,\sigma^2)$ iid.

    z.z.: $Var(u_{ij}) \leq Var(f_{ij})$\\

    \begin{equation*}
        Var(f_{ij}) = E(\underbrace{f_{ij} - \overbrace{E f_{ij}}^{f^0_{ij}}}_{r_{ij}})^2 = \sigma^2
    \end{equation*}

    \begin{align*}
        Var(u_{ij}) &= E(u_{ij} - E u_{ij})^2 = E((1 - 4 \alpha) (\underbrace{f_{ij} - f^0_{ij}}_{r_{ij}}) + \alpha(\underbrace{(f_{i-1,j} - f^0_{i-1,j})}_{r_{i-1,j}} + ... + \underbrace{(f_{i+1,j} - f^0_{i+1,j})}_{r_{i+1,j}}))^2\\
        &= E((1 - 4 \alpha)^2 r_{ij}^2 + \alpha^2(r_{i-1,j}^2 + r_{i,j-1}^2 +r_{i,j+1}^2 + r_{i+1,j}^2) + 2 (1 - 4 \alpha) \alpha r_{ij} r_{i-1,j}...)\\%TODO fix this mess
        &= (1 - 4 \alpha)^2 \underbrace{E r_{i,j}^2}_{\sigma^2} + \alpha^2(E r_{i-1,j}^2 + ... + E r_{i+1,j}^2) + 2 (1 - 4 \alpha) \alpha \underbrace{E(r_{ij}r_{i-1,j})}_{\underbrace{E r_{ij} E r_{i-1,j}}_{0}} + \underbrace{...}_{0})\\
        &=(1 - 4 \alpha)^2 \sigma^2 + \alpha^2 4 \sigma^2 = (1 - 8 \alpha + 16 \alpha ^2 + 4 \alpha^2) \sigma^2
    \end{align*}

    Da $0 \leq \alpha$ und $ 0 \leq 1 - 4 \alpha \Rightarrow 0 \leq \alpha \leq \frac{1}{4}$:

    \begin{equation*}
        (1 - 8 \alpha + 16 \alpha ^2 + 4 \alpha^2) \sigma^2 = \underbrace{1 + \underbrace{20 \alpha}_{\geq 0} (\underbrace{\alpha - \frac{2}{5}}_{< 0})}_{\leq 1}
    \end{equation*}

    $\Rightarrow Var(u_{ij}) \leq Var(f_{ij})$ für $\alpha \in [0,\frac{1}{4}]$\\

    Dabei gilt: $Var(u_{ij}) \overset{\alpha}{\to} d \text{min} \iff 1 - 8 \alpha + 20 \alpha^2 \overset{\alpha}{\to} \text{min} \iff -8 + 40 \alpha = 0 \iff \alpha = \frac{1}{5}$

    \begin{equation*}
        \Rightarrow \text{bester Filter} : \ \raisebox{-0.9cm}{\begin{tikzpicture}[scale = 1.3]
                \draw[step = 0.5] (0,0) grid (0.5,1.5);
                \draw[step = 0.5] (-0.5,1) grid (1,0.5);
                \draw (0.25,1.25) node {$\frac{1}{5}$};
                \draw (0.25,0.75) node {$\frac{1}{5}$};
                \draw (0.25,0.25) node {$\frac{1}{5}$};
                \draw (-0.25,0.75) node {$\frac{1}{5}$};
                \draw (0.75,0.75) node {$\frac{1}{5}$};
            \end{tikzpicture}}
        \end{equation*}

    \subsection{Frequenzraum-filter}
        Ansatz: Rauschen = hochfrequente Anteile des Signals.\\
        Diese können mittels der \mim{Fouriertransformation} $\mathcal F$ gezielt entfernt werden.

        \begin{equation*}
            \begin{tikzpicture}[scale = 0.9]
                \draw (0,0) rectangle (2,2);
                \draw (1,0) node[below] {\small Sauberes Bild};
                \draw (1,1) node[] {\LARGE $f_0$};
                \draw[->] (2.1,1) -- node[above] {$+r$} (2.9,1);
                \draw (3,0) rectangle (5,2);
                \draw (4,0) node[below] {\small Gestörtes Bild};
                \draw (4,1) node[] {\LARGE $f$};
                \draw[->] (5.1,1) -- node[above] {$\mathcal F$} (5.9,1);
                \draw (6,-0.5) rectangle (9,2.5);
                \draw (6,1) -- (9,1);
                \draw[thick, name path = P2]plot [smooth, tension = 0.8] coordinates {(6,1) (6.1,1.3) (6.4,1.1) (6.6,1.6) (7,1.1) (7.3,2.0) (7.5,1.1) (7.7,1.2) (7.8,1.9) (8.1,1.5) (8.5,1.9) (9,1)};
                \draw[->] (9.1,1) -- node[above] {\small Abschneiden} (10.9,1);
                \draw[->] (14.1,1) -- node[above] {$\mathcal F^{-1}$} (14.9,1);
                \draw[decorate,decoration={brace,amplitude=2pt,mirror}] (6,0.6) -- node[below] {\scriptsize \begin{tabular}{c} niedrige \\ Frequenzen \end{tabular}} (7.5,0.6);
                \draw[decorate,decoration={brace,amplitude=2pt,mirror}] (7.5,0.6) -- node[below] {\scriptsize \begin{tabular}{c} hohe \\ Frequenzen \end{tabular}} (9,0.6);
                \draw[thick, name path = P2,shift = {(5,0)}]plot [smooth, tension = 0.8] coordinates {(6,1) (6.1,1.3) (6.4,1.1) (6.6,1.6) (7,1.1) (7.3,2.0) (7.5,1.1) (7.7,1.2) (7.8,1.9) (8.1,1.5) (8.5,1.9) (9,1)};
                \draw[color = white,fill] (12.7,1) rectangle (14,2.5);
                \draw (11,1) -- (14,1);
                \draw (11,-0.5) rectangle (14,2.5);
                \draw (15,0) rectangle (17,2);
                \draw[thick] (7.7,0.8) -- (7.7,2.3);
                \draw[thick] (12.7,0.8) -- (12.7,2.3);
                \draw (16,1) node[] {\LARGE $u$};
                \draw[decorate,decoration={brace,amplitude=2pt,mirror}] (5.5,-1) -- node[below] {im Frequenzbereich} (14.5,-1);
                \draw[decorate,decoration={brace,amplitude=2pt,mirror}] (3,-1.5) -- node[below] {\mim{Frequenzraumfilter} (Tiefpass)} (17,-1.5);
            \end{tikzpicture}
    \end{equation*}

    Ein wichtiges Instrument ist hierbei die Fouriertransformation:
    \[\mathcal F : f \mapsto \hat f\]
    \begin{equation}
        \boxed{\hat f(z) = \frac{1}{(2 \pi)^\frac{d}{2}} \int_{\R^d} f(x) e^{-i \skprod{z}{x}} dx}
    \end{equation}

    Wobei  $z \in \R^d, f \in L^1(\R^d)$.

    Falls auch $\hat f \in L^1(\R^d)$ ist ,dann lässt sich $f$ wie folgt mittels der inversen Fouriertransformation aus $\hat f$ rekonstruieren:

    \[\mathcal F^{-1} : \hat f \mapsto f\]
    \begin{equation}
        \boxed{\hat f(z) = \frac{1}{(2 \pi)^\frac{d}{2}} \int_{\R^d} f(x) e^{i \skprod{z}{x}} dx}
    \end{equation}
    Wobei $x \in \R^d$.\\

    Man hat also $\mathcal F^{-1} \mathcal F f$, d.h.

    \[f(x) = \frac{1}{(2 \pi)^\frac{d}{2}} \int_{\R^d} \left(\frac{1}{(2 \pi)^\frac{d}{2}} \int_{\R^d} f(y) e^{-i \skprod{z}{y}} dy\right) e^{i \skprod{z}{x}} dz\]

    Sei nun $e_z(x) := e^{i \skprod{z}{x}}, \ x \in \R^d$ mit Parameter
$z = \srmatrix{z_1 \\ \vdots \\ z_d}$.\\
    Also $e_z(x) = e^{i\skprod{\srmatrix{z_1\\z_2}}{\srmatrix{x_1\\x_2}}} = e^{i (z_1 x_1 +z_2 x_2)}$\\
    Beispiele in $2D$:\\
    (Hier stellen die Linien, Punkte mit konstantem wert dar)

    \begin{minipage}[t]{0.49\linewidth}
        \begin{center}
            $z = \srmatrix{1 \\ 0}, \  e_z(x)=e^{ix_1}$:\\
            \begin{tikzpicture}
                \draw[->] (-1.1,0) -- (4,0) node[right] {$x_1$};
                \draw[->] (0,-1.1) -- (0,4) node[above] {$x_2$};
                \foreach \i in {-0.7, 0.3, 1.3, 2.3, 3.3}{
                    \draw[thick] (\i,-0.6) -- (\i,4.1);
                }
            \end{tikzpicture}

            $z = \srmatrix{0 \\ 1}, \  e_z(x)=e^{ix_2}$:\\
            \begin{tikzpicture}
                \draw[->] (-1.1,0) -- (4,0) node[right] {$x_1$};
                \draw[->] (0,-1.1) -- (0,4) node[above] {$x_2$};
                \foreach \i in {-0.7, 0.3, 1.3, 2.3, 3.3}{
                    \draw[thick] (-0.6,\i) -- (4.1,\i);
                }
            \end{tikzpicture}

            $z = \srmatrix{1 \\ 1}, \  e_z(x)=e^{i(x_1 + x_2}$:\\
            \begin{tikzpicture}
                \draw[->] (-1.1,0) -- (4,0) node[right] {$x_1$};
                \draw[->] (0,-1.1) -- (0,4) node[above] {$x_2$};
                \foreach \i in {-2.6, -1.6, -0.6, 0.4, 1.4}{
                    \draw[thick, rotate = -45,shift={(0,\i)}] (-2,2) -- (2,2);
                }
            \end{tikzpicture}
        \end{center}
    \end{minipage}
    \hfill
    \begin{minipage}[t]{0.49\linewidth}
        $z = \srmatrix{2 \\ 0}, \ e_z(x)=e^{i2x_1}$:\\
        \begin{tikzpicture}
            \draw[->] (-1.1,0) -- (4,0) node[right] {$x_1$};
            \draw[->] (0,-1.1) -- (0,4) node[above] {$x_2$};
            \foreach \i in {-0.7, -0.2, 0.3, 0.8, 1.3, 1.8, 2.3, 2.8, 3.3}{
                \draw[thick] (\i,-0.6) -- (\i,4.1);
            }
        \end{tikzpicture}

        $z = \srmatrix{0 \\ 2}, \  e_z(x)=e^{i2x_2}$:\\
        \begin{tikzpicture}
            \draw[->] (-1.1,0) -- (4,0) node[right] {$x_1$};
            \draw[->] (0,-1.1) -- (0,4) node[above] {$x_2$};
            \foreach \i in {-0.7, -0.2, 0.3, 0.8, 1.3, 1.8, 2.3, 2.8, 3.3}{
                \draw[thick] (-0.6,\i) -- (4.1,\i);
            }
        \end{tikzpicture}

        $z = \srmatrix{-2 \\ 1}, \  e_z(x)=e^{i(-2x_1 + x_2)}$:\\
        \begin{tikzpicture}
            \draw[->] (-1.1,0) -- (4,0) node[right] {$x_1$};
            \draw[->] (0,-1.1) -- (0,4) node[above] {$x_2$};
            \foreach \i in {-2.8, -1.8, -0.8, 0.2, 1.2}{
                \draw[thick, rotate = 112.5,shift={(0.85*\i,0.85*\i)}] (-1,1) -- (3,-3);
            }
        \end{tikzpicture}
    \end{minipage}

    $\displaystyle f \in L^2(\R^d) = \{f:\R^d \to \R | \int_{\R^d} \abs{f}^2 dx < \infty\}$ ist
    \begin{enumerate}[label = -]
        \item ein normierter Raum mit $+$, $\alpha \cdot$ und $\displaystyle \norm{\cdot}_2 := \sqrt{\int_{\R^d} \abs{f(x)}^2 dx}$
        \item ein Skalarproduktraum mit $\displaystyle \skprod{f}{g} := \int_{\R^d} f \bar g dx$, wobei $\norm{f}_2^2=\skprod{f}{f}$
        \item ein vollständiger Raum, also \mim{Banachraum}
    \end{enumerate}
    Ein vollständiger normierter Banachraum mit Skalarprodukt heißt \mim{Hilbertraum}.

    $\mathcal F$ kann auch als Abbildung auf $L^2(\R^d)$ betrachtet werden. Dann gilt:\\
    \[\hat f = \mathcal F f \in L^2(\R^d)\]
    und
    \begin{equation}
        \norm{\hat f}_2 = \norm{f}_2
    \end{equation}
    und sogar
    \begin{equation}
        \skprod{\hat f}{\hat g}_2 = \skprod{f}{g}_2
    \end{equation}
    für alle $f,g \in L^2(\R^d)$.

    Weitere Eigenschaften der Fouriertransformation:

    \begin{enumerate}[label = \roman *)]
        \item $f \in L^1(\R^d) \Rightarrow \hat f$ stetig und $\underset{\abs{z} \to \infty}{lim} \hat f(z) = 0$
        \item $\mathcal F: L^1(\R^d) \to C(\R^d)$ ist eine lineare Abbildung
        \item $\mathcal F: L^1(\R^d) \to C(\R^d)$ ist eine beschränkte/stetige Abbildung
        \item Verschiebung $\overset{\mathcal F}{\to}$ Modulation, d.h.
        \begin{equation*}
            g(x) = f(x+a) \Rightarrow \hat g(z) = e^{i \skprod{a}{z}} \hat f(z)
        \end{equation*}
        \item Modulation $\overset{\mathcal F}{\to}$ Verschiebung, d.h.
        \begin{equation*}
            g(x) = e^{i \skprod{x}{a}} f(x) \Rightarrow \hat g(z)= \hat f(z-a)
        \end{equation*}
        \item Skalierung $\overset{\mathcal F}{\to}$ inverse Skalierung, d.h.
        \begin{equation*}
            g(x)=f(cx) \Rightarrow \hat g(z) = \frac{1}{\abs c} \hat f(\frac{z}{\abs c})
        \end{equation*}
        \item Konjugation: $g(x) = \overline{f(x)} \Rightarrow \hat g(z) = \overline{\hat f (-z)}$\\
        Folglich: $f$ reelwertig $\Rightarrow \hat f(z) = \overline{\hat f(-z)}$
        \item
        \begin{align*}
            \text{Grundmode:} \ & \displaystyle \hat f(0) = \frac{1}{(2 \pi )^\frac{d}{2}} \int_{\R^d} f(x) dx\\
            \text{Analog:} \ & \displaystyle f(0) = \frac{1}{(2 \pi )^\frac{d}{2}} \int_{\R^d} \hat f(x) dx
        \end{align*}
        \item Differentiation $\overset{\mathcal F}{\to}$ Multiplikation mit Potenzen von z, d.h.
        \begin{equation*}
            g(x) = \frac{\partial^{\alpha_1 + \cdots + \alpha_d}}{\partial x_1^{\alpha_1} \cdots \partial x_d^{\alpha_d}} f(x) \Rightarrow \hat g(z) = i^{\alpha_1 + \cdots + \alpha_d} z_1^{\alpha_1} \cdots z_d^{\alpha_d} \hat f(z)
        \end{equation*}
        \item Umkehrung des letzten Punktes:
        \begin{equation*}
            g(x) = x_1^{\alpha_1} \cdots x_d^{\alpha_d} f(x) \Rightarrow \hat g(z) = i^{\alpha_1 + \cdots + \alpha_d}  \frac{\partial^{\alpha_1 + \cdots + \alpha_d}}{\partial x_1^{\alpha_1}} \hat f(z)
        \end{equation*}
        \item
        \begin{align*}
            \text{Faltungssatz:} \  & \mathcal F(f*g) = (2 \pi)^{\frac{d}{2}} \mathcal F(f) \cdot \mathcal F(g), \ \widehat{f*g}=(2 \pi)^{\frac{d}{2}} \hat f \cdot \hat g\\
            \text{Analog:} \ & \mathcal F (f \cdot g) = \frac{1}{(2 \pi)^{\frac{d}{2}}} \mathcal F(f) * \mathcal F(g), \ \widehat{f \cdot g} = \frac{1}{(2\pi)^{\frac{d}{2}}} \hat f * \hat g
        \end{align*}
        d.h.: Faltung $\overset{\mathcal F}{\to}$ Multiplikation und umgekehrt
    \end{enumerate}

    \Large{\textbf{Zur Erinnerung:}}
    \normalsize
    \begin{center}
        \begin{tikzpicture}[scale = 0.9]
            \draw (0,0) rectangle (3,3);
            \draw (1.5,1.5) node[] {\LARGE $f_0$};
            \draw[->] (3.1,1.5) -- node[above] {$\mathcal F$} (3.9,1.5);
            \draw (4,0) rectangle (7,3);
            \draw (5.5,1.5) node[] {\LARGE $\hat f$};
            \draw[->] (7.1,1.5) -- node[above] {\footnotesize \begin{tabular}{c}
                Hohe Frequenzen\\abschneiden
            \end{tabular}} (9.4,1.5);
            \draw (9.5,0) rectangle (12.5,3);
            \draw (11,1.5) node[] {\LARGE $\hat u$};
            \draw[->] (12.6,1.5) -- node[above] {$\mathcal F^{-1}$} (13.4,1.5);
            \draw (13.5,0) rectangle (16.5,3);
            \draw (15,1.5) node[] {\LARGE $u$};

            \draw[thin] (4,-2) -- node[below] {\small 0} (7,-2);
            \draw[thick, name path = P2,shift = {(-2,-3)}]plot [smooth, tension = 0.8]coordinates {(6,1) (6.1,1.3) (6.4,1.1) (6.6,1.6) (7,1.1) (7.3,2.0) (7.5,1.1) (7.7,1.2) (7.8,1.9) (8.1,1.5) (8.5,1.9) (9,1)};
            \draw[decorate,decoration={brace,amplitude=2pt,mirror}] (4.8,-2.5) -- node[below] {\small tiefe Frequenzen} (6.2,-2.5);

            \draw[thick, name path = P2,shift = {(3.5,-3)}]plot [smooth, tension = 0.8]coordinates {(6,1) (6.1,1.3) (6.4,1.1) (6.6,1.6) (7,1.1) (7.3,2.0) (7.5,1.1) (7.7,1.2) (7.8,1.9) (8.1,1.5) (8.5,1.9) (9,1)};
            \draw[fill, color = white] (9.5,-2) rectangle (10.3,-0.5);
            \draw[fill, color = white] (12.5,-2) rectangle (11.7,-0.5);
            \draw[] (11.7,-0.7) -- (11.7,-2.1) node[below] {\small $r$};
            \draw[] (10.3,-0.7) -- (10.3,-2.1) node[below] {\small $-r$};
            \draw[thin] (9.5,-2) -- (12.5,-2);
            \draw (11,-2.1) node[below] {\small 0};

            \draw[->] (5.5,-3) -- (5.5,-3.5) -- (11,-3.5) -- (11,-2.7);

            \draw[shift = {(0.25,-0.3)}] (6.25,-3.8) node[] {\small \textbullet} (6.5,-4) -- (7,-4) node[below] {\small $-r$} -- (7,-3.6) -- (9,-3.6) -- (9,-4) node[below] {\small $r$} -- (9.5,-4) (8,-4) node[below] {\small 0};

        \end{tikzpicture}
    \end{center}

    \begin{center}
        \begin{tikzpicture}
            \draw[dotted] (-0.3,0) rectangle (-6,-7);
            \draw (-2.5,0) node[above] {\mim{Zeitbereich}};
            \draw[dotted] (0.3,0) rectangle (6,-7);
            \draw (2.5,0) node[above] {\mim{Frequenzbereich}};

            \draw (-3,-1) node[] {$f$};
            \draw (-2.8,-1) edge[bend right=-15,->] node[above, pos = 0.5] {$\mathcal F$} node[below, pos = 0.5] {\small $n \C log(n)$} (2.8,-1);
            \draw (3,-1) node[] {$\hat f$};
            \draw[] (3,-1.5) edge[bend right=-15,->] node[right, pos = 0.5] {\small Mult. mit:} node[left, pos = 0.5] {\small $n$} (3,-5.5);
            \draw (4.05,-4) node[left] {\footnotesize $\hat g:=$};
            \draw (4,-4.1) -- (4.3,-4.1) -- (4.3,-3.8) -- (5,-3.8) -- (5,-4.1) -- (5.3,-4.1) (4.65,-4.1) node[] {\tiny $0$};
            \draw[dotted] (4,-3.8) -- (5.3,-3.8) node[xshift=10] {\footnotesize $\frac{1}{(2 \pi)^\frac{d}{2}}$};
            \draw (3,-6) node[] {$\hat u$} (3,-6.25) node[right] {\small \begin{tabular}{c}
                $= \hat f \cdot \hat g \cdot (2 \pi)^{\frac{d}{2}}$\\
                $= \widehat{f*g}$
            \end{tabular}};
            \draw[] (2.5,-6) edge[bend right=-15,->] node[above, pos = 0.5] {\small $n \cdot log(n)$} node[below, pos = 0.5] {$\mathcal F^{-1}$} (-2.5,-6);
            \draw (-3,-6) node[] {$u$};
            \draw (-3,-6) node[left] {\small $f*g=$};
            \draw (-3,-1.5) edge[bend left=-15,->] node[right,pos = 0.5] {\small $n^2$} node[left, pos = 0.5] {\small Falltung, $*g$} (-3,-5.5);
            \draw [thick, name path = P2,->]plot [smooth, tension = 0.9]coordinates {(-2.5,-1.5) (1.5,-1.5) (1.5,-5) (-2.5,-5.5)};
            \draw (2,-3.5) node[left] {\small $n \cdot log(n)$};
        \end{tikzpicture}
    \end{center}

    Wie sieht $g$ aus?
    \[g = \mathcal F^{-1} \left(\frac{1}{( 2 \pi)^{\frac{d}{2}}} \chi_{[-r,r]}\right)\]

    \small
    \[\left(
    \chi_M(z) =
    \begin{cases}
         0, & z \not \in M\\
         1, & z \in M
    \end{cases}\right)\]

    \normalsize
    \begin{align*}
        g(x) = & \frac{1}{( 2 \pi)^{\frac{d}{2}}}(\mathcal F^{-1} \chi_{[-r,r]^d})(x)\\
        = & \frac{1}{( 2 \pi)^{\frac{d}{2}}} \frac{1}{( 2 \pi)^{\frac{d}{2}}} \int_{\R^d} \chi_{[-r,r]^d}(z) e^{i \skprod{z}{x}} dz\\
        (d=1) \to = & \frac{1}{2 \pi} \int_{-\infty}^{\infty} \chi_{[-r,r]}(z) e^{izx} dz\\
        = & \frac{1}{2 \pi} \int_{r}^{-r} e^{izx} dz\\
        = & \frac{1}{2 \pi}  \frac{e^{izx}}{ix} \Big|_{z=-r}^{r}\\
        = & \frac{1}{2 \pi i x}\left( e^{irx} - e^{-irx} \right)\\
        = & \frac{1}{\pi x} sin(rx)\\
        = & sinc\left(\frac{rx}{\pi}\right) \cdot \frac{r}{\pi}
    \end{align*}

    \[\text{Wobei:\ } sinc(\varphi)=
    \begin{cases}
        \frac{sin(\pi \varphi)}{\pi \varphi} &, \varphi \neq 0\\
        1 &, \varphi = 0
    \end{cases}\]

    $g$ hat auch Masse $1$, denn mit den Eigenschaften der Fouriertransformation folgt:

    \[\frac{1}{( 2 \pi)^{\frac{d}{2}}} = \hat g(0) = (\mathcal F g)(0) = \frac{1}{( 2 \pi)^{\frac{d}{2}}} \int_{R^d} g(x) \underbrace{e^{\underbrace{-\skprod{x}{0}}_0}}_1 dx = \frac{1}{( 2 \pi)^{\frac{d}{2}}} \int_{R^d} g(x) dx\]

    \[\Rightarrow \int_{\R^d} g(x) dx = 1\]

    Für $d=2$ gilt:

    \begin{align*}
        g(x) = & \frac{1}{(2 \pi)^{1}}(\mathcal F^{-1} \chi_{[-r,r]^2})(x)\\
        = & \ \cdots \ (\text{Analog zu oben})\\
        = & \frac{1}{(2 \pi)^{2}} \int_{-\infty}^\infty  \int_{-\infty}^\infty \chi_{[-r,r]^2}\left( \srmatrix{x_1\\x_2} \right) e^{i (z_1 x_1 + z_2 x_2)} dz_1 dz_2\\
        = & \frac{1}{(2 \pi)^{2}} \int_{-r}^r \left( \int_{-r}^r e^{i z_1 x_1} e^{iz_2 x_2} dz_1 \right)dz_2\\
        = & \underbrace{ \left(\frac{1}{2 \pi} \int_{-r}^r e^{i z_1 x_1} dz_1\right) }_{\frac{1}{\pi x_1} sin(\pi x_1)} \underbrace{ \left( \frac{1}{2 \pi} \int_{-r}^r e^{iz_2 x_2} dz_2 \right)}_{\frac{1}{\pi x_2} sin(\pi x_2)}\\
    \end{align*}

    Es ist zu bemerken, dass $g$ eine Art Tensor Struktur besitzt, was in etwa bedeutet das sich die Funktion in beliebigen Dimensionen als Produkt der Funktion in einer Dimensionen darstellen lässt.

    \mim{Gauß-Kern}:
    \begin{align*}
        G(x) =& \frac{1}{(2 \pi)^{\frac{d}{2}}} e^{\frac{-\abs{x}^2}{2}} \Rightarrow G\left( \srmatrix{x_1\\\vdots\\x_d} \right) = \frac{1}{(2 \pi)^{\frac{d}{2}}} e^{\frac{-x_1^2-x_2^2 + \cdots + x_d^2}{2}}\\
        =& \left( \frac{1}{(2 \pi)^\frac{1}{2}} e^{\frac{-x_1^2}{2}}\right) \cdot \ \cdots \ \cdot \left( \frac{1}{(2 \pi)^\frac{1}{2}} e^{\frac{-x_d^2}{2}}\right) = G(x_1) \cdot \ \cdots \ \cdot G(x_d)
    \end{align*}

    \subsection{Filterbreite und Glättung}

    \begin{center}
        \begin{tikzpicture}
            \foreach \i in {0,...,4}
                \foreach \j in {0,...,4}
                    \draw (0.5*\i + 0.25,0.5*\j + 0.25) node[] {\small 1};
            \draw[step = 0.5] (0,0) grid (2.5,2.5);
            \draw (0,1.25) node[left] {klar ist: $\frac{1}{25}$};
            \draw (3,1.25) node[right] {'glättet mehr als': $\frac{1}{9}$};
            \draw[step = 0.5, shift = {(6.5,0.5)}] (0,0) grid (1.5,1.5);
            \foreach \i in {0,...,2}
                \foreach \j in {0,...,2}
                    \draw (0.5*\i + 6.75,0.5*\j + 0.75) node[] {\small 1};
        \end{tikzpicture}
    \end{center}

    Im Kontinuierlichen: Sei $m \in L^1(\R^d)$ und $s > 0$.
    Setze
        $$ m_s(x) := \frac{1}{s^d} m (\frac{x}{s}), \quad x\in \R^d$$

    Bsp (in $d =1 $):
    \begin{center}
        \begin{tikzpicture}
            \draw[->] (-2.2,0) -- (2.2,0) node[right] {\small x};
            \draw[->] (0,-0.1) -- (0,2.2) node[above] {\small y};
            \draw (-2,0.1) -- (-2,-0.1) node[below] {\small $-1$};
            \draw (2,0.1) -- (2,-0.1) node[below] {\small $-1$};
            \draw (0.1,2) -- (-0.1,2) node[left] {\small $1$};
            \draw[thick] (-2,0) -- (0,2) -- (2,0);
            \draw (1.2,1.2) node[rotate = -45] {\small $y=m(x)$};
            \draw[->] (3.8,0) -- (12.2,0) node[right] {\small x};
            \draw[->] (8,-0.1) -- (8,1.2) node[above] {\small y};
            \draw (4,0.1) -- (4,-0.1) node[below] {\small $-1$};
            \draw (2,0.1) -- (2,-0.1) node[below] {\small $-1$};
            \draw (8.1,1) -- (7.9,1) node[left, xshift = -9, yshift = 4] {\small $\frac{1}{2}$};
            \draw[thick] (4,0) -- (8,1) --(12,0);
            \draw (10,0.8) node[rotate = -12.5] {\small $y=m(x)$};
            \draw[->,double] (2,1.5) -- node[above] {\small $s=2$} (4,1.5);
        \end{tikzpicture}
    \end{center}

    Bsp: Gauß-Kern $G(x) = \frac{1}{(2 \pi)^{\frac{d}{2}}} e^{\frac{-\abs{x}^2}{2}}$\\
    Skalierung mit Faktor $s > 0$
    $$ \Rightarrow G_s(x) = \frac{1}{s^d} G\left( \frac {x} {s} \right) = \frac{1}{s^d} \frac{1}{(2 \pi)^{\frac{d}{2}}} e^{\frac{-\abs{x}}{2}} = \frac{1}{(2 \pi s^2)^{\frac{d}{2}}} e^{\frac{-\abs{x}^2}{2s^2}}$$

    Skalierung $s \hat = $ Standardabweichung $\sigma$:

    \begin{center}
        \begin{tikzpicture}
            \draw[scale=1,domain=-2.5:2.5,smooth,variable=\x]  plot ({\x},{(e^((-(\x)^2)/2)});
            \draw[scale=1,domain=-2.5:2.5,smooth,variable=\x,shift={(6,0)}]  plot ({\x},{(1/(2*pi)^(1/2))*e^((-(\x)^2)/2)});
            \draw[->,double] (2.5,0.5) -- (3.5,0.5);
        \end{tikzpicture}
    \end{center}

    \subsection{Differenzenfilter}

    Bisher: Glättung $\widehat =$ Mittelwert bilden $\widehat =$ Summe/Integrale\\
    Jetzt: Schärfen $\widehat =$ Differenzen/Kontraste hervorheben $\widehat =$ Differenzen/Ableitungen\\

    \mtitle{Diskretisierung von Ableitungen durch Differenzenquotienten}
    \ \\
    \begin{minipage}[c]{0.25\linewidth}
        \begin{center}
            \begin{tikzpicture}

                \draw[scale=1,domain=-0.75:1.5,smooth,variable=\x]  plot ({\x},{(\x * \x)});
                \draw[] (-1,-0.3) -- (2,-0.3);
                \draw[dotted] (-0.5,-0.3) node[] {\small \textbullet} node[below] {\small $x_{k-1}$} -- (-0.5,0.25) node[] {\small \textbullet};
                \draw[dotted] (0.25,-0.3) node[] {\small \textbullet} node[below] {\small $x_{k}$} -- (0.25,0.0125) node[] {\small \textbullet};
                \draw[dotted] (1,-0.3) node[] {\small \textbullet} node[below] {\small $x_{k+1}$} -- (1,1) node[] {\small \textbullet};
                \draw[decorate,decoration={brace,amplitude=2pt,mirror}] (-0.5,-0.7) -- node[below] {\small $h$} (0.25,-0.7);
                \draw[dotted] (1,-0.3) node[] {\small \textbullet} node[below] {\small $x_{k+1}$} -- (1,1) node[] {\small \textbullet};
                \draw[decorate,decoration={brace,amplitude=2pt,mirror}] (0.25,-0.7) -- node[below] {\small $h$} (1,-0.7);
            \end{tikzpicture}
        \end{center}
    \end{minipage}
    \hfill\vrule\hfill
    \begin{minipage}[c]{0.4\linewidth}
        (hier bedeutet $f(k) = f(x_k)$)\\
        Vorwärts: $\displaystyle u(h)= \frac{f(k+1) - f(k)}{h}$
        Rückwärts: $\displaystyle u(h)= \frac{f(k) - f(k-1)}{h}$
        Zentral: $\displaystyle u(h)= \frac{f(k+1) - f(k-1)}{2h}$
    \end{minipage}
    \hfill
    \begin{minipage}[c]{0.3\linewidth}
        \ \\
        $u=\displaystyle \frac{1}{h}\begin{tabular}{|c|c|c|}
            \hline
            0 & -1 & 1\\
            \hline
        \end{tabular} \boxast f$\\
        \ \\
        $u=\displaystyle \frac{1}{h}\begin{tabular}{|c|c|c|}
            \hline
            0 & -1 & 1\\
            \hline
        \end{tabular} \boxast f$\\
        \ \\
        $u=\displaystyle \frac{1}{2h}\begin{tabular}{|c|c|c|}
            \hline
            0 & -1 & 1\\
            \hline
            \end{tabular} \boxast f$\\
    \end{minipage}
    \ \\

    \mtitle{2. Abbleitung:}

    \begin{align*}
        u(h) \approx & \frac{f'(k+1) - f'(k)}{h} \text{(vorwärts)}\\
        \approx & \frac{\frac{f(k+1) - f(k)}{h} - \frac{f(k) - f(k-1)}{h}}{h} \text{(rückwärts)} \\
        = & \frac{f(k+1) -2 f(k) + f(k+1)}{h^2}
    \end{align*}

    Also folgt $\displaystyle u:=\begin{tabular}{|c|c|c|}
        \hline
        1 & -2 & 1\\
        \hline
        \end{tabular} \boxast f$ und $\displaystyle \frac{1}{h^2}\begin{tabular}{|c|c|c|}
            \hline
            1 & -2 & 1\\
            \hline
            \end{tabular} = \frac{1}{h}\begin{tabular}{|c|c|c|}
                \hline
                0 & -1 & 1\\
                \hline
                \end{tabular} * \frac{1}{h}\begin{tabular}{|c|c|c|}
                    \hline
                    -1 & 1 & 0\\
                    \hline
                    \end{tabular}$\\
    Denn:

    \begin{align*}
         & \frac{1}{h} \filter{-1 & 1 & 0}\\
        =& \frac{1}{h} \filter{0 & 1 & -1} * \left(\frac{1}{h} \filter{1 & -1 & 0} * f\right)\\
        =& \left(\frac{1}{h} \filter{ 0 & 1 & -1} * \frac{1}{h} \filter{1 & -1 & 0}\right)*f\\
        =& \left(\frac{1}{h} \filter{ -1 & 1 & 0} \boxast \frac{1}{h} \filter{1 & -1 & 0}\right)*f\\
        =& \frac{1}{h^2} \filter{1 & -2 & 1} * f \\
        =&\frac{1}{h^2} \filter{1 & -2 & 1} \boxast f
    \end{align*}

    In 2D: $\displaystyle \frac{\partial}{\partial x} \widehat = \ \begin{tabular}{|c|c|c|}
        \hline
        0 & -1 & 1\\
        \hline
    \end{tabular}, \ \frac{\partial}{\partial y} \widehat = \begin{tabular}{|c|}
        \hline
        0\\
        \hline
        -1\\
        \hline
        1\\
        \hline
    \end{tabular}, \ \frac{\partial^2}{\partial x^2} \widehat = \begin{tabular}{|c|c|c|}
        \hline
        1 & -2 & 1\\
        \hline
    \end{tabular}, \ \frac{\partial^2}{\partial y^2} \widehat = \begin{tabular}{|c|}
        \hline
        1\\
        \hline
        -2\\
        \hline
        1\\
        \hline
    \end{tabular}$.\\
    \ \\
    \mim{Diskreter Laplace Operator}:
    \[\Delta = \frac{\partial^2}{\partial x^2} + \frac{\partial^2}{\partial y^2} \widehat = \ \begin{tabular}{|c|c|c|}
        \hline
        1 & -2 & 1\\
        \hline
    \end{tabular} + \begin{tabular}{|c|}
        \hline
        1\\
        \hline
        -2\\
        \hline
        1\\
        \hline
    \end{tabular} = \begin{tabular}{|c|c|c|}
        \hline
         0 & 1 & 0\\
        \hline
        1 & -4 & 1\\
        \hline
         0 & 1 & 0\\
        \hline
    \end{tabular}\]

    \subsection{Glättungsfilter und partielle Differentialgleichungen}

    Wir haben gesehen: $\displaystyle m = \frac{1}{5}\begin{tabular}{|c|c|c|}
        \hline
        0 & 1 & 0\\
        \hline
        1 & 1 & 1\\
        \hline
        0 & 1 & 0\\
        \hline
    \end{tabular}$ ist unter allen 5-Punkt Filtern der am besten glättende.\\
    Idee: Rauschen weiter verringern indem man $m \boxast$ wiederholt anwendet $\Rightarrow$ Folge von Bildern:\\

    \begin{center}
        \begin{tikzpicture}
            \draw (0,0) rectangle (1,1);
            \draw (0.5,0.5) node {\small $f$};
            \draw (0.5,0.2) node {\tiny $:= u ^{(0)}$};
            \draw[->] (1.1,0.5) -- node[above] {\small $m \boxast$}(1.9,0.5);
            \draw (2,0) rectangle (3,1);
            \draw (2.5,0.5) node {\small $u^{(1)}$};
            \draw[->] (3.1,0.5) -- node[above] {\small $m \boxast$}(3.9,0.5);
            \draw (4,0) rectangle (5,1);
            \draw (4.5,0.5) node {\small $u^{(2)}$};
            \draw (5,0.5) node[right] {\LARGE ...};
        \end{tikzpicture}
    \end{center}

    \begin{align*}
        \Rightarrow u^{(n+1)} - u^{(n)} & = \text{(Unterschied zwischen 'Zeit' Punkt $n$ und $n+1$)}\\
        &= \underbrace{m \boxast u^{(n)}}_{u^{n+1}} - \underbrace{\delta \boxast u^{(n)}}_{u^{(n)}} \text{mit } \delta = \begin{tabular}{|c|c|c|}
            \hline
            0 & 0 & 0\\
            \hline
            0 & 1 & 0\\
            \hline
            0 & 0 & 0\\
            \hline
        \end{tabular}\\
        &=(m - \delta) \boxast u^{(n)}\\
        &=\left( \frac{1}{5}\begin{tabular}{|c|c|c|}
            \hline
            0 & 1 & 0\\
            \hline
            1 & 1 & 1\\
            \hline
            0 & 1 & 0\\
            \hline
        \end{tabular} - \frac{1}{5} \begin{tabular}{|c|c|c|}
            \hline
            0 & 0 & 0\\
            \hline
            0 & 5 & 0\\
            \hline
            0 & 0 & 0\\
            \hline
        \end{tabular}\right) \boxast u^{(n)}\\
        &= \frac{1}{5} \begin{tabular}{|c|c|c|}
            \hline
            0 & 1 & 0\\
            \hline
            1 & -4 & 1\\
            \hline
            0 & 1 & 0\\
            \hline
        \end{tabular} u^{(n)}
    \end{align*}

    Somit gilt insgesamt:

    \begin{equation}\label{eq:5.11}
        \underbrace{u^{(n+1)} - u^{(n)}}_{\widehat = \frac{\partial u}{\partial t}} = \underbrace{\frac{1}{5} \begin{tabular}{|c|c|c|}
            \hline
            0 & 1 & 0\\
            \hline
            1 & -4 & 1\\
            \hline
            0 & 1 & 0\\
            \hline
        \end{tabular}}_{\widehat = \Delta u}
    \end{equation}

    Kontinuierlich: Funktion $u$
    \[u(x,t) \quad x \in \R^2, \ t \text{ Zeit} \]

    \eqref{eq:5.11} ist eine Diskretisierung (1 Zeitschritt im Eulerverfahren) der partiellen Differentialgleichungen
    \begin{equation}\label{eq:5.12}
        \frac{\partial u}{\partial t} = \Delta u
    \end{equation}
    Bekannt als \mim{Wärmegleichung} oder \mim{Diffusionsgleichung}.\\
    Zum Zeitpunkt $t=0$ möge die Anfangsbedingung
    \begin{equation}\label{eq:5.13}
        u(x,0)=u^{(0)}=f(x)
    \end{equation}
    gelten. Vorranschreiten der Zeit $t$ repräsentiert Diffusion.\\
    Für einen stationären Zustand, also keine Änderung $\frac{\partial u}{\partial t}$ dann muss auch $\Delta u =0$ gelten.\\
    Diese wird unteranderem von konstanten Funktionen oder linearen Funktionen $u(x_1,x_2) = ax_1 + bx_2$ erfüllt.\\
    \ \\
    Es existiert auch einen explizite Formel für die Lösung der Diffusionsgleichung \eqref{eq:5.12} mit Anfangsbedingung \eqref{eq:5.13}:
    \[u(x,t) = \left( G_{\sqrt{2t}} * u^{(0)} \right)(x)\]
    Wobei $\sqrt{2t}$ für eine Skalierung um diesen Wert steht.\\
    Zu zeigen ist: $\displaystyle \frac{\partial u}{\partial t} = \Delta u$
    \[\frac{\partial}{\partial t}  \left( G_{\sqrt{2t}} * u^{(0)} \right) = \Delta  \left( G_{\sqrt{2t}} * u^{(0)} \right)\]
    \[\overset{\text{mit Satz}}{\Longrightarrow}  \left( \frac{\partial}{\partial t} G_{\sqrt{2t}} \right)* u^{(0)} =  \left( \Delta G_{\sqrt{2t}} \right) * u^{(0)}\]
    Es bleibt somit z.z.: $\frac{\partial}{\partial t} G_{\sqrt{2t}} = \Delta G_{\sqrt{2t}}$.

    \begin{center}
        \begin{tikzpicture}
            \draw (0,1.5) node[left] {$t=0$:};
            \draw (0,0) -- (3,0) node[below] {\small a} -- (3,1) -- (4,1) -- (4,0) node[below] {\small b} -- (7,0);
            \draw (0,-2) node[left] {$t>0$:};
            \draw[shift={(0,-3.5)}] plot [smooth, tension = 0.1] coordinates {(0,0) (2.95,0.05) (3.05,1) (3.95,1) (4.05,0.05) (7,0)};
        \end{tikzpicture}
    \end{center}
    Bemerkenswert ist das, für $t=0$ die Funktion nicht stetig ist, aber für alle $t>0$ die Funktion beliebig oft differenzierbar ist.\\
    \ \\
    Insgesamt lässt sich die Idee darstellen als:

    \begin{center}
        \begin{tikzpicture}
            \draw[->] (0,0) node[left] {\small kontinueirlich:} -- (10,0) node[right] {\small t};

            \draw (1,-0.5) rectangle (2,-1.5);
            \draw (1.5,-1) node {\large $u^{(0)}$};
            \draw (1.5,-1.5) node[below] {\small $u(\cdot,0)$};
            \draw[->] (1.5,-2) -- (1.5,-2.5) -- node[above] {\small $G_{\sqrt{2t}}*$} (8.5,-2.5) -- (8.5,-2);
            \draw (8,-0.5) rectangle (9,-1.5);
            \draw (8.5,-1.5) node[below] {\small $u(\cdot,t)$};

            \draw (0,-4) node[left] {\small diskret:};
            \draw (1,-4) rectangle (2,-5);
            \draw (1.5,-4.5) node[] {\large $u^{(0)}$};
            \draw[->] (2.1,-4.5) -- node[above] {\small $m \boxast$} (2.9,-4.5);
            \draw[shift={(2,0)}] (1,-4) rectangle (2,-5);
            \draw[shift={(2,0)}] (1.5,-4.5) node[] {\large $u^{(1)}$};
            \draw[->,shift={(2,0)}] (2.1,-4.5) -- node[above] {\small $m \boxast$} (2.9,-4.5);
            \draw (6,-4.5) node[] {\LARGE ...};

            \draw[shift={(7,0)}] (1,-4) rectangle (2,-5);
            \draw[shift={(7,0)}] (1.5,-4.5) node[] {\large $u^{(n)}$};
            \draw[->,shift={(5,0)}] (2.1,-4.5) -- node[above] {\small $m \boxast$} (2.9,-4.5);
        \end{tikzpicture}
      \end{center}


      % TODO ???

      \subsection{Isotrope und anisoptrope Diffusion}


      Wir haben gesehen: Glättung/Diffusion verringert Rauschen.\\
      Aber: Auch Kanten/Details werden verwischt.\\
      Ausweg: Diffusion steuern, so dass sie an Kanten (also Stellen mit großer Änderungsrate) weniger stark glättet.\\

      Der Plan lautet also:
      \[\nabla u = \norm{\srmatrix{\frac{\partial u}{\partial x}\\ \frac{\partial u}{\partial y}}}^2 = \begin{cases}
          \text{groß} & \Rightarrow \text{wenig Diffusion}\\
          \text{klein} & \Rightarrow \text{Diffusion normal}
      \end{cases}\]

      Diffusionsgleichung:

      \begin{equation}
          \frac{\partial u}{\partial t} = \Delta u = \frac{\partial}{\partial x} \frac{\partial}{\partial x} u + \frac{\partial}{\partial y} \frac{\partial}{\partial y} u = \underbrace{\srmatrix{ \frac{\partial }{\partial x } \frac{\partial}{\partial y}}}_{div} \srmatrix{\frac{\partial}{\partial x} u \\ \frac{\partial}{\partial y} u} = div(\nabla u)
      \end{equation}

      Um diese Gleichung zu regulieren setzen wir einen \mim{Diffusionstensor} $M$ in die Gleichung in.

      \[\Delta u = div(M \nabla u) = div(\srmatrix{* & * \\ * & *} \nabla u)\]

      Ansätze für $M$:

      \begin{enumerate}[label=\alph*)]
          \item $M=I=\srmatrix{1 & 0 \\ 0 & 1} \Rightarrow$ übliche Diffusion
          \item $M=g(\norm{\nabla u(x,y)}) * I$
          \begin{center}
              \begin{tikzpicture}
                  \draw[shift={(-0.75,0.75)}] (0,0) node[left] {$\displaystyle g_\kappa(s) = \frac{1}{1+\left(\frac{s}{\kappa}\right)^2}$};
                  \draw[scale=1.5,domain=0:3,smooth,variable=\x] plot ({\x},{(1/(1+\x*\x)});
                  \draw (0,1.5) node[left] {\small 1};
                  \draw[->] (0,-0.5) -- (0,2);
                  \draw[->] (0,0) -- (5,0) node[right] {\small $s$};
                  \draw[dotted] (1.5,0) node [below] {\small $\kappa$} -- (1.5,0.75);
                  \draw[dotted] (0,0.75) node[left] {\small $\frac{1}{2}$} -- (1.5,0.75);
              \end{tikzpicture}
          \end{center}
          Diese Methode geht zurück auf Perona \& Malik.
          \begin{enumerate}[label=\textbullet]
              \item Kanten mit $\norm{\nabla u} < \kappa$ werden mehr geglättet
              \item Kanten mit $\norm{\nabla u} \geq \kappa$ werden weniger geglättet
          \end{enumerate}
          Diese Art der Glättung ist \mim{Isotrop} $\widehat = $ in alle Richtungen gleich starker Fluss.
          \item $M= \begin{pmatrix}
              g(\abs{\frac{\partial u}{\partial x}(x,y)}) & 0 \\
              0 & g(\abs{\frac{\partial u}{\partial y}(x,y)})
          \end{pmatrix}$\\
          Diese art der Diffusionstensoren ist \mim{anisoptrop} also richtungsabhängig.\\
          \ \\
          Für $\x \in \Z^2$ und $\x_W = \x + \srmatrix{-1 \\ 0}$ usw.\\
          \begin{center}
              \begin{tikzpicture}
                  \draw (0,0) rectangle node[] {$\x$} (0.5,0.5);
                  \draw[->] (0.25,0.5) -- ++(0,1);
                  \draw (0,1.5) rectangle node[] {$\x_N$} (0.5,2);
                  \draw[->] (0.25,0) -- ++(0,-1);
                  \draw (0,-1) rectangle node[] {$\x_S$} (0.5,-1.5);
                  \draw[->] (0.5,0.25) -- ++(1,0);
                  \draw (1.5,0) rectangle node[] {$\x_O$} (2,0.5);
                  \draw[->] (0,0.25) -- ++(-1,0);
                  \draw (-1,0) rectangle node[] {$\x_W$} (-1.5,0.5);
                  \draw[->] (-1,-0.25) -- ++(0.75,0) node[right] {\small $x$};
                  \draw[->] (-1,-0.25) -- ++(0,-0.75) node[below] {\small $y$};
              \end{tikzpicture}
        \end{center}
        \[\text{Für} \ M=\begin{pmatrix}
            c_1(\x) & 0\\
            0 & c_2(\x)
        \end{pmatrix} \ \text{gilt:}\]

        \begin{align*}div(M \cdot \nabla u(\x)) =& \begin{pmatrix}
            \frac{\partial}{\partial x} & \frac{\partial}{\partial y}
        \end{pmatrix}
        \left[ \begin{pmatrix}
        c_1(\x) & 0\\
        0 & c_2(\x)
        \end{pmatrix}
        \begin{pmatrix}
            \frac{\partial u}{\partial x} (\x)\\
            \ \\
            \frac{\partial u}{\partial y} (\x)
        \end{pmatrix}
        \right]
        = \begin{pmatrix}
        \frac{\partial}{\partial x} & \frac{\partial}{\partial y}
        \end{pmatrix}
        \begin{pmatrix}
            c_1(\x) \frac{\partial u}{\partial x} (\x)\\
            \ \\
            c_2(\x) \frac{\partial u}{\partial y} (\x)
        \end{pmatrix}\\
        \approx & \begin{pmatrix}
        \frac{\partial}{\partial x} & \frac{\partial}{\partial y}
        \end{pmatrix}
        \begin{pmatrix}
            c_1(\x) (u(\x_O) - u(\x)) \\
            \ \\
            c_2(\x) (u(\x_S) - u(\x))
        \end{pmatrix}\\
        \approx & c_1(\x) (u(\x_O) - u(\x)) - c_1(\x_W)(u(\x_N) - u(\x_W))\\
        +&c_2(\x) (u(\x_S) - u(\x)) - c_2(\x_N)(u(\x) - u(\x_N))
        \end{align*}
      \end{enumerate}

      \subsection{Bilaterale Filter}
      Es existiert auch ein anderer Ansatz für das selbe Problem.
      \[u(\x) = \text{ gewichtetes Mittel aus allen } f(\y) \text{ mit}\]

      \begin{enumerate}[label= \alph*)]
          \item $\y$ ist nahe bei $\x$ \underline{und}
          \item $f(\y)$ ist nahe bei $f(\x)$
      \end{enumerate}

      \[u(\x) = \frac{1}{w(\x)} \int_\Omega \underbrace{g(\x - \y)}_{a)} \underbrace{h(f(\x) - f(\y))}_{\text{neu }b)} f(\y) d\y\]
      Heißt \mim{Bilateraler Filter}, wobei

      \[w(\x) = \int_\Omega g(\x - \y) h(f(\x) - f(\y)) d\y\]

      \begin{enumerate}
          \item[Oft:] $g,h$ Gauß-Kerne ($\Rightarrow$ nichtlineare Gaußfilter)
          \item[Manchmal:] $g,h$ charakteristische Funktionen ($\Rightarrow$ SUSAN-Filter)
          \begin{enumerate}
              \item[Effekt] Falls Höhe der Kante $>$ Filterradius $\Rightarrow$ Kante bleibt
          \end{enumerate}
          \item[Manchmal:] $f \overset{log}{\mapsto} log \ f \overset{\small \text{Bil. Filter}}{\mapsto} log \ u \overset{exp}{\mapsto} u$
      \end{enumerate}
      Diese Verfahren ist jedoch sehr aufwendig, denn
      \begin{enumerate}[label=\textbullet]
          \item keine Reine Filterung ($\Rightarrow$ keine FFT-Implementierung möglich)
          \item Normalisierung $w(\x)$ in jedem Punkt berechnen
      \end{enumerate}

      \subsection{Entrauschen mittels Variationsrechnung}
      Erinnerung:
      \begin{center}
          \begin{tikzpicture}
              \draw (0,0) rectangle node[] {\large $f$} (1,1);
              \draw (0.5,0) node[below] {\small entrauschtes Bild};
              \draw[->] (1,0.5) -- node[above] {\Large ?} (2,0.5);
              \draw (2,0) rectangle node[] {\large $u$} (3,1);
              \draw (2.5,0) node[below] {\small Resultat};
          \end{tikzpicture}
      \end{center}

      Wünsche an $u$:
      \begin{enumerate}
          \item $u \approx f$ (Datenkonsistenz)
          \item $u$ ist 'glatt'. (Regularitätsbedingung)
      \end{enumerate}

      Mathematische Umsetzung der Wünsche:
      \begin{enumerate}
          \item $\displaystyle \norm{u-f}_2 = \sqrt{\int_\Omega \abs{u(\x) - f(\x)}^2 dx}$ sei klein
          \item $\displaystyle \norm{\nabla u}_2 = \sqrt{\int_\Omega \abs{\nabla u(\x)}^2 dx} = \sqrt{\int_\Omega \left( \frac{\partial u}{\partial x} (\x) \right)^2 + \left( \frac{\partial u}{\partial x} (\x) \right)^2 d\x} $ sei klein
      \end{enumerate}

      Kombination:
      \begin{equation}\label{eq:5.15}
          J(u) := \norm{u-f}_2^2 + \lambda \norm{\nabla u}_2^2 \overset{u \in U}{\rightarrow} \text{min}
      \end{equation}
      Für einen geeigneten Funktionen Raum $U$ und \mim{Kopplungskonstante} $\lambda > 0$.\\
      In diesem Beispiel empfiehlt sich als Suchraum:

      \[U=\{ u : \norm{u} < \infty, \ \nabla \text{ existiert }, \ \norm{\nabla}_2<\infty  \}=W^{1,2}\]
      ein so genannter \mim{Sobolev-Räume}. Diese Suchproblem is jedoch $\infty$-dimensional und somit schwer zu lösen.\\

      Im obigen Ansatz \eqref{eq:5.15} stellt man fest, dass der Regularitätsterm
      \[\norm{\nabla u}_2^2 = \int_\Omega \abs{\nabla u(\x)}^2 d\x =
      \int_\Omega \left( \frac{\partial u}{\partial x} (\x) \right)^2 + \left( \frac{\partial u}{\partial x} (\x) \right)^2 d\x \]
      die großen Gradienten an (gewolleten) Kanten zu stark bestraft. ($\Rightarrow$ optimales $u$ glättet Kanten)\\

      Ausweg: Wähle $\norm{\nabla u}_2$ oder $\displaystyle \norm{\nabla u}_1 =\int_\Omega \abs{\nabla u(\x)} d\x = \int_\Omega \abs{\frac{\partial u}{\partial x} (\x)} + \abs{\frac{\partial u}{\partial y} (\x)} d\x$ als Regularitätsterme.

      \begin{equation}\label{eq:5.16}
        J(u):= \norm{u-f}_2^2 + \lambda \norm{\nabla u}_1 \to \ \text{min}
      \end{equation}
      Genannt \mim{Rudin–Osher–Fatemi-Funktional} (ROF)

      \textbf{Allgemeiner Ansatz bei Variationsproblemen:}

      \[J(u):=\underbrace{D(u,f)}_{\text{Datenkern}} + \lambda \underbrace{R(u)}_{\text{Regularitätsterm}} \overset{u \in U}{\to} \text{min}\]

      Notwendiges Kriterium:\\
      Falls $J:U \to \R$ in $u \in U$ ein lokales Minimum besutzt, dann gilt für jede Richtung $v \in U$:

      \begin{equation}\label{eq:5.17}
        \underset{\epsilon \nearrow 0}{lim} \frac{J(u+\epsilon v) - J(u)}{\epsilon}=0
      \end{equation}
      Dies ist die Verallgemeinerte Richtungsableitung (Gateux-Ableitung).\\

      Häufig ist $J$ in Integralform gegeben, z.b.:

      \[J(u)= \int_\Omega g(x,u(x),\nabla(x)) dx\]

      Dann führt Bedingung \eqref{eq:5.17} auf Gleichungen für bestimmte partielle Ableitungen von $g$ und $u$, die sogenannte \mim{Euler-Lagrange-Gleichung} für \eqref{eq:5.17}.\\
      $\Rightarrow$ partielle Differentialgleichung $u$.
      Fazit:

      \begin{center}
          \begin{tikzpicture}
              \draw (0,0) rectangle node[] {Filter} (2,1);
              \draw[->] (2.1,0.5) -- node[above, text width=2cm] {\small \begin{center}
                wiederholte Anwendung
              \end{center}} (3.9,0.5);
              \draw (4,0) rectangle node[yshift=-12,above,text width=3cm] {\begin{center}
                partielle Differentialgleichung
              \end{center}} (7,1);
              \draw[->] (8.9,0.5) -- node[above,text width=2cm] {\small \begin{center}
                Euler-Lagrange
              \end{center}} (7.1,0.5);
              \draw (9,0) rectangle node[] {\small Variationsrechnung} (12,1);
          \end{tikzpicture}
      \end{center}

    \section{Kantenerkennung}
        \subsection{\mim{Gradientenfilter}}
            Wir suchen Stellen $\x$ mit großem Gradienten:
            \[\nabla u(\x) = \srmatrix{\frac{\partial u}{\partial x}(\x)\\\frac{\partial u}{\partial y}(\x)}\]
            Approximation der Gradienten über zentrale Differenzen:
            \begin{equation}\label{eq:6.1}
                \frac{\partial}{\partial x} \approx \frac{1}{2} \begin{tabular}{|c|c|c|}\hline
                    -1 & 0 & 1\\
                    \hline
                \end{tabular} \text{ bzw. } \frac{\partial}{\partial y} \approx \frac{1}{2} \begin{tabular}{|c|c|c|}\hline
                    -1\\
                    \hline
                    0\\
                    \hline
                    1\\
                    \hline
                \end{tabular}
            \end{equation}

            Um Rauschen zu verringern wird auch ein entrauschen Filter simultan angewendet:

            \begin{center}
                \begin{tikzpicture}
                    \draw[left color=black!20!white, right color=black!60,color=white] (0,0) rectangle (2,5);
                    \draw (1.8,2.5) node[] {$\frac{1}{2} \begin{tabular}{|c|c|c|}\hline
                        -1 & 0 & 1\\
                        \hline
                    \end{tabular}$};
                    \draw[->,double] (3,2.5) -- node[above] {\small gegen Rauschen}(5.4,2.5);
                    \draw (5.5,2.5) node[right] {$\frac{\partial }{\partial x} \approx  \frac{1}{6} \begin{tabular}{|c|c|c|}\hline
                        -1 & 0 & 1\\
                        \hline
                        -1 & 0 & 1\\
                        \hline
                        -1 & 0 & 1\\
                        \hline
                    \end{tabular}$};
                    \draw[->] (6.7,1.7) -- node[below] {Differenzeiren} (8.7,1.7);
                    \draw[<->] (8.8,3.2) -- node[right] {Glättung} (8.8,1.7);
                    \draw (7.7,3.2) node[above] {\mim{Prewitt-Filter}:};
                \end{tikzpicture}
            \end{center}

            Alternative: $\frac{\partial }{\partial x} \approx \frac{1}{2} \begin{tabular}{|c|c|c|}\hline
                -1 & 0 & 1\\
                \hline
            \end{tabular} \boxast \frac{1}{4}\begin{tabular}{|c|c|c|}\hline
                -1\\
                \hline
                0\\
                \hline
                1\\
                \hline
            \end{tabular} = \frac{1}{8} \begin{tabular}{|c|c|c|}\hline
            -1 & 0 & 1\\
            \hline
            -2 & 0 & 2\\
            \hline
            -1 & 0 & 1\\
            \hline
            \end{tabular}=:D_x$, genannt \mim{Sobel-Filter}.
            Eine stärkere Glättung kann mittels anderer vertikaler Filter mit Binomialkoeffizienten erzeilt werden.\\

            Entsprechen wird $\frac{\partial }{\partial y} D_y:=D_x^T$ definiert.\\

            \begin{equation}\label{eq:6.2}
                \nabla u(\x) = \srmatrix{\frac{\partial u}{\partial x}(\x)\\\frac{\partial u}{\partial y}(\x)} \approx \srmatrix{(D_x \boxast u)(\x)\\ (D_y \boxast u)(\x)}
            \end{equation}

            Zur Erinnerung der Gradienten steht senkrecht auf Kanten und zeigt in Richtung heller (hoher) Werte, die Intensität wird beschrieben von $\abs{\nabla u(\x)}$, also dem Betrag des Gradienten.

            Ein typischer Algorithmus kann etwa folgende Form annehmen:
            \begin{enumerate}
                \item Gradienten mittels Prewitt oder Sobel approximieren und Richtung auf Vielfache von $45^\circ$ runden.
                \item \mim{Non-maximum suppression} (edge thinning). Da es potentiell viele Punkte mit hoher Steigung gibt kann es dazu kommen, dass Kanten sehr breit werden, dieses wird durch das edge thinning verhindert.

            \begin{center}
                \begin{tikzpicture}
                    \draw[left color=white, right color=black] (0,0) rectangle (1,1);
                    \draw[fill] (1,0) rectangle (4,1);
                    \draw[shift={(0,-1)}] plot [smooth, tension = 0.2] coordinates {(0,0.5) (0.65,0.5) (1,0) (4,0)};
                    \draw (0.7,1.5) rectangle (0.9,-1.5);
                    \draw (0.9,1.5) node[right] {\small Kante};
                    \draw (0.8,1.7) -- (0.8,-1.7) node[below] {\small dünne Kante};
                \end{tikzpicture}
            \end{center}

            Mathematisch: $\x$ wird Kantenpunkt falls:
            \[\abs{\nabla u(\x)} \leq max(\abs{\nabla u(\x_+)},\abs{\nabla u(\x_-)})\]
            wobei $\x_+$ und $\x_-$ Vorgänger und Nachfolger von $\x$ in Gradientenrichtung sind.
            \item Kandidat $\x$ wird Kantenpunkt, falls:
            \begin{center}
                \begin{tikzpicture}
                    \draw (0,0) -- (11,0);
                    \draw[decorate,decoration={brace,amplitude=5pt}] (0.1,0.1) -- node[above] {\small Kein Kantenpunkt} (2.95,0.1);
                    \draw[decorate,decoration={brace,amplitude=5pt}] (3.05,0.1) -- node[above] {\small Ja, falls Nachbar Kantenpunkt} (7.95,0.1);
                    \draw[decorate,decoration={brace,amplitude=5pt}] (8.05,0.1) -- node[above] {\small Kantenpunkt} (10.95,0.1);
                    \draw (3,0.2) -- (3,-0.2) node[below] {\small $t_1$};
                    \draw (8,0.2) -- (8,-0.2) node[below] {\small $t_2$};
                \end{tikzpicture}
            \end{center}
            wobei $t_1, \ t_2$ thresholds sind.\\
            $\x$ ist also ein Kantenpunkt, falls $\abs{\nabla u(\x)} \geq t_2$ oder $\left( \abs{\nabla u(\x)} \in [t_1,t_2] \text{ und $\x$ ist Nachbar eines Kantenpunktes}  \right)$.
            Dieses wird \mim{hysteresis thresholding} genannt und verhindert \mim{Abreißen} von Kantenzügen.
            \end{enumerate}

            Die am häufigsten verbreitete Version von 1) -3) ist der \mim{Canny-Algorithmus} (1986).

            Matlab:
            \begin{lstlisting}
                BWimg=edge(u,'canny',[t_1, t_2],sigma);
            \end{lstlisting}

            \begin{enumerate}
                \item[BWimg:] Binärbild
                \item[u:] Graustufenbild
                \item[canny:]Algorithmus
                \item[$t_1, \ t_2$:] Sind gewählt wie oben
                \item[sigma:] Parameter für den Gaußkern aus 1)
            \end{enumerate}

        \subsection{Die zweite Ableitung}

            Zunächst in 1D:
            %Wenn jemand eine bessere Idee hat, kann er das gerne aufräumen.
            \begin{center}
                \begin{tikzpicture}
                    \draw[dotted] (-0.5,0) node[left] {\small $u$:}-- (8.5,0);
                    \draw[scale=1,domain=-2:0,smooth,variable=\x,shift ={(2,0)}] plot ({\x},{(e^((-(\x)^2)/(0.2))}) -- (4,1) [scale=1,domain=0:2,smooth,variable=\x,shift ={(4,0)}] plot({\x},{(e^((-(\x)^2)/(0.2))});

                    \draw[dotted] (-0.5,-2) node[left] {\small $u'$:}-- (8.5,-2);
                    \draw[scale=1,domain=-1.5:1,smooth,variable=\x,shift ={(1.5,-2)}] plot ({\x},{(e^((-(\x)^2)/(0.05))}) -- (4,0) [scale=1,domain=-1:1.5,smooth,variable=\x,shift ={(5,0)}] plot({\x},{-(e^((-(\x)^2)/(0.05))});

                    \draw[dotted] (-0.5,-4) node[left] {\small $u''$:}-- (8.5,-4);
                    \draw[scale=1,domain=-1.5:1,smooth,variable=\x,shift ={(1.5,-4)}] plot ({\x},{-5*\x*(e^((-(\x)^2)/(0.05))}) -- (4,0) [scale=1,domain=-1:1.5,smooth,variable=\x,shift ={(5,0)}] plot({\x},{5*\x*(e^((-(\x)^2)/(0.05))});

                    \draw(1.5,1.2)--(1.5,-4.5);
                    \draw(6.5,1.2)--(6.5,-4.5);
                    \draw[] (1.5,0.2865) node[circle,fill,inner sep=1pt] {};
                    \draw (6.5,0.2865) node[circle,fill,inner sep=1pt] {};
                    \draw (1.5,-1.02) node[circle,fill,inner sep=1pt] {};
                    \draw (6.5,-2.98) node[circle,fill,inner sep=1pt] {};
                    \draw (1.5,-4) node[circle,fill,inner sep=1pt] {};
                    \draw (6.5,-4) node[circle,fill,inner sep=1pt] {};
                \end{tikzpicture}
            \end{center}

            Test für kantenpunkte $u''(\x) = 0$ und $\abs{u'(\x)}>$ threshold.\\
            Wichtig: Vorglätten!, da die 2. Ableitung noch anfälliger gegenüber Rauschen als die 1. Ableitung ist.\\
            \ \\
            In 2D. Laplace Operator $\Delta u = \frac{\partial^2 u}{\partial^2 x} + \frac{\partial^2 u}{\partial^2 y}$ (Richtungsunabhängige Messung der 2. Ableitung)\\
            Vorglätten: $\Delta (G * u) = (\Delta G)*u$, wobei $\Delta G$ vorher berechnet werden kann.

            In 1D:

            \begin{center}
                \begin{tikzpicture}
                    \draw (0,0.2) node[left] {$G:$};
                    \draw[scale=1,domain=-1.5:1.5,smooth,variable=\x,shift ={(2,0)}] plot ({\x},{(e^((-(\x)^2)/(0.2))});
                    \draw (4.5,0.2) node[left] {$\Delta G:$};
                    \draw[scale=0.5,domain=-4:4,smooth,variable=\x,shift ={(13,0)}] plot ({\x},{(\x*\x*(e^((-(\x)^2)/(2))) - (e^((-(\x)^2)/(2))});
                \end{tikzpicture}
            \end{center}

            In 2D:
            \def\centerx{2}
            \def\centery{-1}
            \begin{center}
                \begin{tikzpicture}
                \pgfplotsset{
                    colormap name=viridis,
                }
                    \begin{axis}[hide axis,scale=0.75,name=plot1,samples=60]
                        \addplot3[surf,domain=-2:6,domain y=-3.7:3.7,samples=40]
                            {exp(-( (x-\centerx)^2 + (y-\centery)^2)/3 )};
                        \end{axis}
                        \draw[->] (6,1.5)--(7,1.5);
                        \draw (1.5,0) node[left] {\Large $G$};
                        \draw (7.5,0) node[left] {\Large $\Delta G$};
                    \begin{axis}[hide axis,scale=0.75,at={(8cm,1cm)}, anchor=center]
                        \addplot3[surf,domain=-3:3,domain y=-5:5,samples=60]
                        {x^2*exp(- x^2/2 - y^2/2) - 2*exp(- x^2/2 - y^2/2) + y^2*exp(- x^2/2 - y^2/2)};
                    \end{axis}
                \end{tikzpicture}
            \end{center}

            Dieses wird \mim{Laplacian of Gaußian method} genannt.\\
            Matlab:
            \begin{lstlisting}
                BWimg=edge(u,'log',thresh,sigma);
            \end{lstlisting}
            $\Rightarrow$ alle $\x \in \Omega$ mit:
            \begin{enumerate}
                \item[] $\Delta (G_{sigma} * u)(\x) \approx u$, nicht auf Gleichheit sondern auf Vorzeichenwechsel testen.
                \item[und:] $\abs{\nabla(G_{sigma}) * u}>\text{thresh}$
            \end{enumerate} 

            \section{Schärfen und Entfalten}
            (Gegenteil von Kapitel 5)
            \begin{enumerate}
                \item[Gegeben:] unscharfes Bild
                \item[Gesucht:] Version mit vielen erkennbaren Details 
            \end{enumerate}
        
            \subsection{\mim{Laplace-Schärfen}}
                Idee:
        
                \begin{center}
                \begin{tikzpicture}
                    \draw[dotted] (-0.5,0) node[left] {\small $u$:}-- (8.5,0);
                    \draw[scale=1,domain=-2:0,smooth,variable=\x,shift ={(2,0)}] plot ({\x},{(e^((-(\x)^2)/(0.2))}) -- (4,1) [scale=1,domain=0:2,smooth,variable=\x,shift ={(4,0)}] plot({\x},{(e^((-(\x)^2)/(0.2))});
        
                    \draw[dotted] (-0.5,-2) node[left] {\small $u''$:}-- (8.5,-2);
                    \draw[scale=1,domain=-1.5:1,smooth,variable=\x,shift ={(1.5,-2)}] plot ({\x},{-5*\x*(e^((-(\x)^2)/(0.05))}) -- (4,0) [scale=1,domain=-1:1.5,smooth,variable=\x,shift ={(5,0)}] plot({\x},{5*\x*(e^((-(\x)^2)/(0.05))});
        
                    \draw[dotted] (-0.5,-4) node[left] {\small $u - \tau u''$:}-- (8.5,-4);
                    \draw[scale=1,domain=-2:0,smooth,variable=\x,shift ={(2,-4)}] plot ({\x},{(e^((-(\x)^2)/(0.2))+4*(\x+0.5)*(e^((-(\x+0.5)^2)/(0.07))}) -- (4,1.056) [scale=1,domain=0:2,smooth,variable=\x,shift ={(4,0)}] plot({\x},{(e^((-(\x)^2)/(0.2))-4*(\x-0.5)*(e^((-(\x-0.5)^2)/(0.07))});
        
                \end{tikzpicture}
            \end{center}
            Zu sehen ist, dass durch die Subtraktion von $u''$, skaliert mit einem Faktor $\tau > 0$ die Kanten hervorgehoben werden.\\
            
            Hinweise zur Umsetzung:
            \begin{enumerate}[label=-]
                \item $u - \tau u''$ reskalieren (Kontrast-stretching) falls der Farbraum verlassen wird.
                \item $\tau$ kann auch sehr klein gewählt werden und der Vorgang dafür wiederholt iteriert werden.
                \item In 2D $\Delta statt$ 2. Ableitung
                \item Vorglätten: $u - \tau \C \Delta (G*u)$
            \end{enumerate}
        
            \subsection{Kantenverstärkende Diffusion}
                Verallgemeinerte Diffusionsgleichung: $\displaystyle \frac{\partial u}{\partial t} = div(M \nabla u)$.
                Idee: $M$ so wählen, so dass der Fluss:
                \begin{enumerate}[label=-]
                    \item Parallel zum Gradienten (d.h. durch die Kante verläuft): $\displaystyle \lambda_1=\frac{1}{1+\frac{\abs{\nabla u (\x)}^2}{\kappa^2}}$
                    \item senkrecht zu $\nabla u$ (entlang der Kante): $\Lambda_2=1$
                \end{enumerate}
        
                $\Rightarrow M$ hat EW $\lambda_1$ zum EV $\displaystyle v_1= \frac{\nabla u}{\abs{\nabla u}}\srmatrix{\frac{\partial u}{\partial x}(\x)\\\frac{\partial u}{\partial y}(\x)}$ und EW $\lambda_2$ zum EV $\displaystyle v_2 = \frac{1}{\abs{\nabla u}} \srmatrix{-\frac{\partial u}{\partial y}(\x)\\\frac{\partial u}{\partial x}(\x)} \perp v_1$.\\
                $\displaystyle \Rightarrow M \C \underbrace{\mat{v_1 & v_2}}_{\small \text{orthogonale Matrix}} = \mat{\lambda_1 v_1 & \lambda_2 v_2} = \mat{v_1 & v_2} \mat{\lambda_1 & \ \\ \ & \lambda_2} \Rightarrow M^{-1} = M^T$\\
                $\Rightarrow M= \mat{v_1 & v_2} \mat{\lambda_1 & \ \\ \ & \lambda_2} \underbrace{\mat{v_1 & v_2}^T}_{=\mat{v_1^T \\ v_2^T}} = \frac{1}{\abs{\nabla u}^2} \mat{\lambda_1 \frac{\partial u}{\partial x}(\x)^2 + \lambda_2 \frac{\partial u}{\partial y}(\x)^2  & (\lambda_1 - \lambda_2)\frac{\partial u}{\partial x} \frac{\partial u}{\partial y}(\x)\\ (\lambda_1 - \lambda_2)\frac{\partial u}{\partial x} \frac{\partial u}{\partial y}(\x) & \lambda_2 \frac{\partial u}{\partial x}(\x)^2 + \lambda_1 \frac{\partial u}{\partial y}(\x)^2}$\\
                falls $\nabla u(\x) \neq 0$, sonst $M= \mat{1 & 0 \\ 0 & 1}$.
        
            \subsection{\mim{Entfaltung}}
                \begin{center}
                    \begin{tikzpicture}
                        \draw (0,0) rectangle node[] {$f$} node[above,yshift=27] {original} (2,2);
                        \draw (4,0) rectangle node[] {$u$} node[above,yshift=27] {geglättet} (6,2);
                        \draw[] (2.1,1) edge[bend left,->]  node[above] {$g*$} node[below] {\small Glättung} (3.9,1);
                        \draw[] (1,-0.1) edge[bend right=40,<-]  node[above] {?} node[below] {\small Entfaltung} (5,-0.1);
                    \end{tikzpicture}
                \end{center}
                Das heißt: $u=f * g$, wobei $u, \ g$ gegeben sind und $f$ gesucht ist.\\
                Alternativ kann dies als die Invertierung des Faltungsoperator $f \mapsto g * f$ betrachtet werden.
                \begin{enumerate}[label = \alph*)]
                    \item Diskreter Fall: \begin{align*}
                                            & g*f=u\\
                                            &(g*f)(j)=u(j), \ j \in \Omega\\
                                            &\sum_k g(j-k)f(k)=u(j), \ j \in \Omega\\
                                            &\Rightarrow \Omega \times \Omega \text{ Gleichungsystem}
                                            \end{align*}
                        \begin{center}
                            \begin{tikzpicture}
                                \matrix [matrix of nodes, column sep=1mm, row sep=1mm,name=M,left delimiter={(},right delimiter={)}]
                                {
                                    {$g(0)$}& {$g(-1)$}& {}& {}& {}& {}& {}& {$g(-n)$}\\
                                    {$g(1)$}& {$g(0)$}& {$g(-1)$}& {}& {}& {}& {}& {}\\
                                    {}& {}& {}& {}& {}& {}& {}& {}\\
                                    {}& {}& {}& {}& {}& {}& {}& {}\\
                                    {}& {}& {}& {}& {}& {}& {}& {}\\
                                    {}& {}& {}& {}& {}& {}& {}& {}\\
                                    {}& {}& {}& {}& {}& {}& {}& {\phantom{g(-0)}}\\
                                    {}& {}& {}& {}& {}& {}& {}& {\phantom{g(0)}}\\
                                    {}& {}& {}& {}& {}& {}& {}& {\phantom{g(0)}}\\
                                    {$g(n)$}& {}& {}& {}& {}& {}& {}& {}\\ 
                                    };
                                    \draw[dotted] (M-2-2)--(M-8-8);
                                    \draw[dotted] (M-2-1)--(M-9-8);
                                    \draw[dotted] (M-2-3)--(M-7-8);
                                    \draw (M-1-1) node[left,xshift =-23] {$j=0$};
                                    \draw (M-2-1) node[left,xshift =-23] {$j=1$};
                                    \draw (M-10-1) node[left,xshift =-23] {$j=n$};
                                    \draw (M-1-1) node[yshift =23] {$k=0$};
                                    \draw (M-1-2) node[yshift =23] {$k=1$};
                                    \draw (M-1-8) node[yshift =23] {$k=n$};
                                    \draw[decorate,decoration={brace,amplitude=4pt,mirror}] (-3.3,-3) -- node[below] {\mim{Toeplitz-Matrix}} (3.5,-3);
                                    \matrix [matrix of nodes, column sep=1mm, row sep=1mm,name=N,left delimiter={(},right delimiter={)},shift={(5,0)}]
                                    {
                                    {$f(0)$}\\
                                    {$f(1)$}\\
                                    {}\\
                                    {}\\
                                    {}\\
                                    {}\\
                                    {}\\
                                    {}\\
                                    {}\\
                                    {}\\
                                    {}\\
                                    {}\\
                                    {$f(n)$}\\ 
                                    };
                                \draw[dotted] (N-2-1) -- (N-13-1);
                                \draw (6.9,0) node[] {\Large $=$};
                                \matrix [matrix of nodes, column sep=1mm, row sep=1mm,name=O,left delimiter={(},right delimiter={)},shift={(8.5,0)}]
                                {
                                    {$u(0)$}\\
                                    {$u(1)$}\\
                                    {}\\
                                    {}\\
                                    {}\\
                                    {}\\
                                    {}\\
                                    {}\\
                                    {}\\
                                    {}\\
                                    {}\\
                                    {}\\
                                    {$u(n)$}\\ 
                                    };
                                \draw[dotted] (O-2-1) -- (O-13-1);
                            \end{tikzpicture}
                        \end{center}
                        \item Kontinuierlicher Fall: \begin{align*}
                                                        &(g*f)(x)=u(x),\ x \in \Omega\\
                                                        &\int g(x-y) f(y) dy = u(x), \ x \in \Omega\\
                                                        &\text{Integralgleichung für die gesuchte Funktion $f$}
                                                        \end{align*}
                        $\Rightarrow$ Kontinuierliche Matrix:
    
                        \begin{center}
                            \begin{tikzpicture}
                                \draw (0,0) rectangle (5,5);
                                \draw[->] (-0.2,5.1) -- (-0.2,-0.1) node[below] {$x$};
                                \draw[->] (-0.1,5.2) -- (5.1,5.2) node[right] {$y$};
                                \draw[] (2.5,5.5) node {\Large $g$};
                                \draw[dotted] (0.2,4.8) -- node[sloped,above] {$g(x-y)$} (4.8,0.2);
                                \draw (5.5,5) node[above,xshift=7] {\Large $f$} rectangle (6,0);
                                \draw[<->] (6.2,4.9) node[above] {\small $a$} -- (6.2,0.1)  node[below] {\small $b$};
                                \draw (6.5,2.5) node[] {\Large $=$};
                                \draw (7,5) node[above,xshift=7] {\Large $u$} rectangle (7.5,0);
                                \draw[->] (7.7,4.9) -- (7.7,0.1) node[above] {\small $x$};
                            \end{tikzpicture}
                        \end{center}
                        Wobei $[a,b]$ die das Definitionsgebiet von $f$ ist.
                        Diese Problem is jedoch schlecht gestellt, da der Operator kompakt ist. ($\nearrow$ Datei im Studip)
                \end{enumerate}

                Wir versuchen es trotzdem zu lösen:
                \begin{align*}
                    g*f&=u & &|\cdot \F\\
                    \F(g * f)&= \F u \\
                    (2 \pi)^\frac{d}{2}(\F g)\cdot (\F f)&= \F u & &| \div (2 \pi)^\frac{d}{2}(\F g)\\
                    \F f &= \frac{\F u}{(2 \pi)^\frac{d}{2} \F g} & &| \F^{-1}
                \end{align*}
                Und erhalten:
                \begin{equation}
                    f= (2 \pi)^\frac{d}{2} \F^{-1}\left(\frac{\F u}{\F g}\right)
                \end{equation}
                Dieses kann jedoch zu Problemen führen, da etwa $g \approx 0$ werden kann.
                Je glatter $g$ ist, desto stärker klingt $(\F g)(z)$ ab für $z \to \infty$.\\
                \ \\
                Anders betrachtet:\\
                Wenn $\abs{\hat g(z)}$ für hohe Frequenzen klein ist, dann ist:
                \[A:f \mapsto g +f\]
                ein Tiefpassfilter. Nimmt man nun eine Funktion $h$ mit hoher Frequenz und großer Amplitude, dann gilt:
                \[A(f + h) = Af + \underbrace{Ah}_{\approx 0} \approx Af\]
                
                \underline{Problembehebung:}
                \begin{enumerate}[label = \arabic*. Ansatz:]
                    \item \ \\
                    \begin{minipage}[c]{0.55\textwidth}
                        \begin{center}
                            Approximiere die Funktion $\frac{1}{x}$ durch\\
                            \[R_\alpha = \begin{cases}
                                \frac{1}{x}, & \abs{x} > \alpha\\
                                \frac{1}{\alpha}, & x \in [0,\alpha]\\
                                \frac{1}{-\alpha}, & x \in [-\alpha,0]
                            \end{cases}\]
                            wobei $\alpha > 0$.
                        \end{center}
                        und ersetze $\displaystyle f=(2 \pi)^\frac{d}{2} \F^{-1}\left(\frac{\hat u (z)}{\hat g(z)}\right)$ durch:\\
                        \[f=(2 \pi)^\frac{d}{2} \F^{-1}\left(\hat u (z) R_\alpha(\hat g(z)) \right)\]
                        und lasse $\alpha \to 0$.
                    \end{minipage}
                    %\hfill\vrule\hfill
                    \begin{minipage}[c]{0.4\textwidth}
                        \begin{center}
                            \begin{tikzpicture}
                                \draw[->] (-2,0) -- (2,0);
                                \draw[<-] (0,2) -- (0,-2);
                                \draw[scale=0.4,domain=0.22:4.7,smooth,variable=\x,shift ={(0,0)}] plot ({\x},{1/\x});
                                \draw[scale=0.4,domain=-0.22:-4.7,smooth,variable=\x,shift ={(0,0)}] plot ({\x},{1/\x});
                                \draw[scale=0.4,domain=0.5:4.7,smooth,variable=\x,shift ={(0,0)},dashed,red,thick] plot ({\x},{1/\x});
                                \draw[scale=0.4,domain=-0.5:-4.7,smooth,variable=\x,shift ={(0,0)},dashed,red,thick] plot ({\x},{1/\x});
                                \draw (0.2,0.1) -- (0.2,-0.1) node[below] {\tiny $\alpha$};
                                \draw (-0.2,0.1) -- (-0.2,-0.1) node[below] {\tiny $-\alpha$};
                                \draw[color=red,thick] (0,0.81818) -- (0.2,0.81818);
                                \draw[color=red,thick] (0,-0.81818) -- (-0.2,-0.81818);
                                \draw (-1.9,1.9) rectangle node[align=right,left,xshift = 2] {\small $\frac{1}{x}$:\\$R_\alpha$:} (-0.6,1);
                                \draw (-1.2,1.62)--(-0.7,1.62);
                                \draw[thick,red] (-1.2,1.22)--(-0.7,1.22);
                            \end{tikzpicture}
                        \end{center}
                    \end{minipage}
                    \item Variationsrechnung:
                        \begin{enumerate}[label = \arabic*. Wunsch:]
                            \item $g *f \approx u$
                            \item $\norm{f}_2$ klein
                        \end{enumerate}
                        Minimiere nun:
                        \[\Rightarrow J(f) := \norm{g * f - u}_2^2 + \lambda \norm{f}_2^2 \to min\]
                        \[\iff \int_{\R^d} ((g*f)(x) - u(x))^2 + \lambda f(x)^2 dx \to min\]
                        über die Wahl von $f \in U:= L^2(\R^d)$.\\
                        Idee: $\F$ anwenden $\Rightarrow *$ wird zu $\C$ \underline{und} $\norm{\C}_2$ bleibt unverändert.
                        \begin{align*}
                            \Rightarrow J(f)&=\norm{g*f-u}_2^2 + \lambda\norm{f}_2^2\\
                            &=||\widehat{g*f-u}||_2^2 + \lambda ||\hat f||_2^2\\
                            &=||(2 \pi)^\frac{d}{2} \hat g \hat f -\hat u||_2^2 + \lambda||\hat f||_2^2\\
                            &=\int_{\R^d}\left[ \left((2 \pi)^\frac{d}{2} \hat g(z) \hat f(z) - \hat u(z)\right)^2 + \lambda |\hat f(z)|^2 \right] dz \overset{f \in U}{\longrightarrow} min
                        \end{align*}
                    Strategie: Integral für jedes einzelne $z$ minimieren. Daraus erhalten wir ein optimales $\hat f$ und somit auch ein optimales $f$.\\
                    Also minimiere für jedes $z \in \R^d$
                    \[I(t):=|(2 \pi)^\frac{d}{2} \hat g(z)t - \hat u(z)|^2 + \lambda |t|^2 \overset{t \in \mathbb C}{\longrightarrow}min\]
                    Später setzen wir $\hat f(z):=t_{min}$, nun zur minimierung:
                    \begin{align*}
                        I(t)&=((2 \pi)^\frac{d}{2} \hat g(z) t - \hat u(z))((2 \pi)^\frac{d}{2} \ \overline{\hat g(z)} \ \overline t -\overline{\hat u (z)}) + \lambda t \overline t\\
                        &=(2 \pi)^\frac{d}{2} \hat g(z) \ \overline{\hat g(z)} \ t \ \overline t + \lambda t \overline t - (2 \pi)^\frac{d}{2} (\hat g(z) \ \overline{\hat u(z)} \ t + \overline{\hat g(z)} \ \hat u(z) \ \overline{t}) + \hat u(z) \ \overline{\hat u(z)}\\
                        &=((2 \pi)^\frac{d}{2} |\hat g(z)|^2 + \lambda)|t|^2 - (2 \pi)^\frac{d}{2} (2 \C Re(\underbrace{\hat g(z) \ \overline{\hat u(z)}t}_{\circledast})) + |\hat u(z)|^2 \overset{t \in \mathbb C}{\longrightarrow} min\\
                    \end{align*}
                    Das Argument (Winkel) taucht nur in $\circledast$ auf\\
                    \begin{align*}
                        &\Rightarrow \text{So wählen, das $\circledast$ auf die positive reele Achse fällt}\\
                        &\Rightarrow 0=arg(\circledast)= arg(\hat g(z) \ \overline{\hat u(z)}) + arg(t)\\
                        &\Rightarrow arg(t)=-arg(\hat g(z) \ \overline{\hat u(z)})=arg(\overline{\hat g(z)} \ \hat u(z))\\
                        &\Rightarrow I(t) = ((2\pi)^\frac{d}{2} |\hat g(z)|^2 + \lambda)|t|^2 - (2\pi)^\frac{d}{2} 2 \C |\overline{\hat g(z)} \ \hat u(z)| \ |t| + |\hat u (z)|^2 \overset{|t| \in \R}{\longrightarrow}min
                    \end{align*}
                    Dieses ist nun ein Polynom in $|t|$, sodass das minimum einfach bestimmt werden kann.\\
                    \begin{align*}
                        &0=\frac{d}{d|t|}...=2\C((2\pi)^\frac{d}{2}|\hat g(z)|^2 + \lambda)|t| - (2\pi)^\frac{d}{2} \C 2 \C |\overline{\hat g(z)} \ \hat u(z)|\\
                        &\Rightarrow |t| = \frac{(2\pi)^\frac{d}{2} \C 2 \C |\overline{\hat g(z)} \ \hat u(z)|}{2\C((2\pi)^\frac{d}{2}|\hat g(z)|^2 + \lambda)} = \frac{(2\pi)^\frac{d}{2}\C |\overline{\hat g(z)} \ \hat u(z)|}{(2\pi)^\frac{d}{2}|\hat g(z)|^2 + \lambda} \text{ und } arg(u)=arg(\overline{\hat g(z)} \ \hat u(z))\\
                        &\Rightarrow t =  \frac{(2\pi)^\frac{d}{2} \ \overline{\hat g(z)} \ \hat u(z)}{(2\pi)^\frac{d}{2}|\hat g(z)|^2 + \lambda}=:\hat f(z)
                    \end{align*}
                    Wegen
                    \[\hat f(z) = (2\pi)^\frac{d}{2} \frac{\overline{\hat g(z)}}{(2\pi)^\frac{d}{2}|\hat g(z)|^2 + \lambda} \hat u(z)\]
                    gilt
                    \begin{equation}
                        f(z) = \F^{-1}\left( \frac{\overline{\hat g(z)}}{(2\pi)^\frac{d}{2}|\hat g(z)|^2 + \lambda} \right) * u
                    \end{equation}
                    Dieses Verfahren wird \mim{$L^2$ deblurring} gennant.
                    Es gibt auch einen alternativen, algebraischen Zugang:
                
                    \[I(f) = \norm{g*f-u}_2^2 + \lambda \norm{f}_2^2 \overset{f}{\to} min\]
                    \[\iff \norm{\mat{g*f-u\\ \sqrt{\lambda} f}} \overset{f}{\to} min\]
                    \[\iff \norm{\mat{Af \\ \sqrt{\lambda}f} - \mat{u\\0}} = \norm{\mat{A \\ \sqrt{\lambda}}f - \mat{u\\0}} \overset{f}{\to} min \quad (A= f \mapsto g*f)\]
                    $\Rightarrow$ lineares Ausgleichsproblem.
                    \[\Rightarrow \mat{A^* &\sqrt{\lambda} I^*} \mat{A \\ \sqrt{\lambda}I}f = \mat{A^* & \sqrt{\lambda}I} \mat{u \\ 0} \quad \text{(Normalengleichung)}\]
                    \[\Rightarrow \mat{A^*A + \abs{\lambda}I}f=A^*u\]
                    \[\Rightarrow f= \mat{A^*A + \abs{\lambda}I}^{-1}A^*u\]
                    Die Inverse existiert, da $-\abs{\lambda}$ nicht im Spektrum von $A^*A$ sein kann, denn das Spektrum von $A^*A$ ist positiv und reel.
                    \item noch einmal Variationsrechnung, diesmal mit anderen Wünschen
                    \item Variationsrechnung:
                    \begin{enumerate}[label = \arabic*. Wunsch:]
                        \item $g *f \approx u$
                        \item $\norm{\nabla f}$ klein
                    \end{enumerate}
                    Nach analoger Rechnung wie oben erhält man: 
                    \begin{equation}
                        f=\F \left( \frac{\overline{\hat g(z)}}{(2\pi)^\frac{d}{2} \ |\hat g(z)|^2 + \lambda |z|^2} \right) * u
                    \end{equation}
                    $\Rightarrow$ Dämpfung höher wenn Frequenz höher.\\
                    Dieses Verfahren nennt sich \mim{$H^1$ deblurring}
                \end{enumerate}

    \printindex

\end{document}
