%\documentclass{report}
\documentclass{article}

\usepackage{amsmath,amsfonts,amssymb,amsthm}
\usepackage{adjustbox}
\usepackage{enumitem}
\usepackage{a4wide}
\usepackage[document]{ragged2e}
\usepackage{tikz}
\usepackage[utf8]{inputenc}
\usepackage[T1]{fontenc}
\usepackage[german]{babel}
\usepackage{hyphenat}
\usepackage{listings}
\usepackage{caption}
%\usepackage{imakeidx}
\usepackage{makeidx}
\usepackage{wasysym}
\usepackage{color}
\usepackage{graphicx}
\usepackage{accents}
\usepackage{stmaryrd}
\usepackage{pgfplots}
\usepackage{varwidth}
\usepackage{mathtools}
\usepackage{hyperref}

\usepackage{fancyhdr}
\pagestyle{fancy}

\renewcommand{\subsectionmark}[1]{\markright{\normalfont \thesubsection \ #1 }}
\renewcommand{\headrulewidth}{0.5pt}
%\renewcommand{\footrulewidth}{0.5pt}

\renewcommand{\sectionmark}[1]{%
\markboth{\normalfont \textbf{#1}}{}}

\fancyhead[LE,RO]{\textsl{\leftmark}}
\fancyhead[LO,RE]{\textsl{\rightmark}}
\pagestyle{fancy}

\definecolor{lcolor}{rgb}{0.0, 0.18, 0.39}
\hypersetup{
    colorlinks=true,
    linktoc=all,
    linkcolor=lcolor,
}

\usetikzlibrary{positioning, calc}
\usetikzlibrary{decorations.pathreplacing,angles,quotes}
\usetikzlibrary{decorations.pathmorphing,snakes}
\usetikzlibrary{shapes.misc}
\usetikzlibrary{intersections,through,backgrounds}
\usetikzlibrary{matrix}
\usetikzlibrary{arrows}



\theoremstyle{plain}
\newtheorem{theorem}{Theorem}
\newtheorem{lemma}[theorem]{Lemma}
\newtheorem*{lemma*}{Lemma}
\newtheorem{cor}[theorem]{Korollar}
\newtheorem*{cor*}{Korollar}
\newtheorem{prop}[theorem]{Proposition}
\setlength{\parskip}{1em}

\theoremstyle{definition}
\newtheorem{definition}[theorem]{Definition}
\newtheorem*{definition*}{Definition}
\newtheorem{notation}[theorem]{Notation}
\newtheorem{bemerkung}[theorem]{Bemerkung}
\newtheorem*{bemerkung*}{Bemerkung}
\newtheorem{bsp}[theorem]{Beispiel}
\newtheorem*{remark*}{remark}
\newtheorem{remark}{remark}

\numberwithin{equation}{section}

\newcommand{\norm}[1] {
\left|\left| #1 \right|\right|
}

\newcommand{\hnorm}[1] {
\left|\left|\left| #1 \right|\right|\right|
}

\newcommand{\colvec}[1]{
\begin{pmatrix}#1\end{pmatrix}
}

\newcommand{\vect}[1]{
\begin{pmatrix}#1\end{pmatrix}
}

\newcommand{\skprod}[2]{
\left \langle #1,#2 \right \rangle
}
\newcommand{\abs}[1] {
\left| #1 \right|
}

\newcommand{\br}[1] {
\left( #1 \right)
}

\newcommand{\R}[0] {
\mathbb R
}

\newcommand{\Z}[0] {
    \mathbb Z
}

\newcommand{\N}[0] {
    \mathbb N
}

\newcommand{\F}[0]{
    \mathcal F
}

\newcommand{\srmatrix}[1] {
\left( \begin{smallmatrix} #1 \end{smallmatrix} \right)
}

\newcommand{\mtitle}[1] {
    \begin{center}
        \large{\textbf{#1}}
    \end{center}
}

\newcommand{\Index}[1]{#1\index{#1}}

\newcommand{\mim}[1] {
\underline{\textbf{#1\index{#1}}}
}

\newcommand{\C}[0]{
    \cdot
}

\newcommand{\pa}[1] {
    \par{\textbf{#1}}
}

\newcommand{\tvmark}[3] {
    \draw[] (#1,0.1+#2) -- (#1,-0.1+#2) node[anchor =north] {\small #3};
}
\newcommand{\thmark}[3] {
    \draw[] (#1+0.1,#2) -- (#1-0.1,#2) node[left] {\small #3};
}

\newcommand{\tvnmark}[3] {
    \draw[] (#1,#2-0.1) node[below] {\small #3};
}

\newcommand{\tlbar}[5]{
    \draw[thick] (#1,#2-0.1) node[below] {\small #4} -- (#1,#3) node[above] {#5};
    \draw[thick] (#1-0.1,#3) -- (#1+0.1,#3);
}

\newcommand{\filter}[1] {
\begin{tabular}{|c|c|c|}
    \hline
    #1\\
    \hline
\end{tabular}
}
\newcommand{\quo}[1] {
\glqq #1 \grqq
}

\newcommand{\x}[0] {
  \boldsymbol{x}
}
\newcommand{\y}[0] {
    \boldsymbol{y}
}

\newcommand{\mat}[1] {
\begin{pmatrix} #1 \end{pmatrix}
}

\definecolor{dkgreen}{rgb}{0,0.6,0}
\definecolor{gray}{rgb}{0.5,0.5,0.5}
\definecolor{mauve}{rgb}{0.58,0,0.82}

\lstset{frame=tb,
language=Matlab,
aboveskip=10mm,
belowskip=10mm,
showstringspaces=false,
columns=flexible,
basicstyle={\small\ttfamily},
numbers=left,
identifierstyle=\color{black},
numberstyle=\small\color{gray},
keywordstyle=\color{blue},
commentstyle=\color{dkgreen},
stringstyle=\color{mauve},
breaklines=true,
breakatwhitespace=true,
tabsize=3}

%Font
%\usepackage{tgadventor}
\renewcommand*\rmdefault{cmss}
\makeindex

\begin{document}
%\title{Mathematische Bildverarbeitung}
%\author{Jonas Sattler}
\pagestyle{empty}
\adjustbox{valign=t}{
    \begin{minipage}[t]{0.65\linewidth}
    {\LARGE \bf Mathematische Bildverarbeitung}\\[1em]
    {\Large \bf Vorlesungsskript}
    \end{minipage}
}
\hfill
\adjustbox{valign=t}{
    \begin{minipage}[t]{0.35\linewidth}
    \begin{flushright}
    \includegraphics[width=1.0\textwidth]{office_grau_de.pdf}
    \end{flushright}
    \end{minipage}
}

\vspace{2em}
\adjustbox{valign=t}{
    \begin{minipage}[t]{0.65\linewidth}
    {\small Institut für Mathematik}\hfill\\
    {\small Vorlesung von Prof. Dr. Marko Lindner}\\
    {\small In \LaTeX gesetzt durch Jonas Sattler}\\[1em]
    {\small Fehlermeldungen an \href{mailto:fabian.gabel@tuhh.de}{fabian.gabel@tuhh.de}}
    \end{minipage}
}
\hfill
\adjustbox{valign=t}{
    \begin{minipage}[t]{0.3\linewidth}
    \begin{flushright}
        {\small Wintersemester 2018/19}\\
    \end{flushright}
    \end{minipage}
}

\vspace{1em}
\par\noindent\rule{\textwidth}{0.4pt}


\tableofcontents
\newpage

\pagestyle{fancy}

\input{chap00}    
%\input{chap01}    
%\input{chap02}    
%\input{chap03}    
%\input{chap04}    
%\input{chap05}    
%            \section{Schärfen und Entfalten}
            (Gegenteil von Kapitel 5)
            \begin{enumerate}
                \item[Gegeben:] unscharfes Bild
                \item[Gesucht:] Version mit vielen erkennbaren Details
            \end{enumerate}

            \subsection{\mim{Laplace-Schärfen}}
                Idee:

                \begin{center}
                \begin{tikzpicture}
                    \draw[dotted] (-0.5,0) node[left] {\small $u$:}-- (8.5,0);
                    \draw[scale=1,domain=-2:0,smooth,variable=\x,shift ={(2,0)}] plot ({\x},{(e^((-(\x)^2)/(0.2))}) -- (4,1) [scale=1,domain=0:2,smooth,variable=\x,shift ={(4,0)}] plot({\x},{(e^((-(\x)^2)/(0.2))});

                    \draw[dotted] (-0.5,-2) node[left] {\small $u''$:}-- (8.5,-2);
                    \draw[scale=1,domain=-1.5:1,smooth,variable=\x,shift ={(1.5,-2)}] plot ({\x},{-5*\x*(e^((-(\x)^2)/(0.05))}) -- (4,0) [scale=1,domain=-1:1.5,smooth,variable=\x,shift ={(5,0)}] plot({\x},{5*\x*(e^((-(\x)^2)/(0.05))});

                    \draw[dotted] (-0.5,-4) node[left] {\small $u - \tau u''$:}-- (8.5,-4);
                    \draw[scale=1,domain=-2:0,smooth,variable=\x,shift ={(2,-4)}] plot ({\x},{(e^((-(\x)^2)/(0.2))+4*(\x+0.5)*(e^((-(\x+0.5)^2)/(0.07))}) -- (4,1.056) [scale=1,domain=0:2,smooth,variable=\x,shift ={(4,0)}] plot({\x},{(e^((-(\x)^2)/(0.2))-4*(\x-0.5)*(e^((-(\x-0.5)^2)/(0.07))});

                \end{tikzpicture}
            \end{center}
            Zu sehen ist, dass durch die Subtraktion von $u''$, skaliert mit einem Faktor $\tau > 0$ die Kanten hervorgehoben werden.\\

            Hinweise zur Umsetzung:
            \begin{enumerate}[label=-]
                \item $u - \tau u''$ reskalieren (Kontrast-stretching) falls der Farbraum verlassen wird.
                \item $\tau$ kann auch sehr klein gewählt werden und der Vorgang dafür wiederholt iteriert werden.
                \item In 2D $\Delta$ statt 2. Ableitung
                \item Vorglätten: $u - \tau \C \Delta (G*u)$
            \end{enumerate}

            \subsection{Kantenverstärkende Diffusion}
                Verallgemeinerte Diffusionsgleichung: $\displaystyle \frac{\partial u}{\partial t} = div(M \nabla u)$.
                Idee: $M$ so wählen, so dass der Fluss:
                \begin{enumerate}[label=-]
                    \item Parallel zum Gradienten (d.h. durch die Kante verläuft): $\displaystyle \lambda_1=\frac{1}{1+\frac{\abs{\nabla u (\x)}^2}{\kappa^2}}$
                    \item senkrecht zu $\nabla u$ (entlang der Kante): $\lambda_2=1$
                \end{enumerate}

                $\Rightarrow M$ hat EW $\lambda_1$ zum EV $\displaystyle v_1= \frac{\nabla u}{\abs{\nabla u}}\srmatrix{\frac{\partial u}{\partial x}(\x)\\\frac{\partial u}{\partial y}(\x)}$ und EW $\lambda_2$ zum EV $\displaystyle v_2 = \frac{1}{\abs{\nabla u}} \srmatrix{-\frac{\partial u}{\partial y}(\x)\\\frac{\partial u}{\partial x}(\x)} \perp v_1$.\\
                $\displaystyle \Rightarrow M \C \underbrace{\mat{v_1 & v_2}}_{\small \text{orthogonale Matrix}} = \mat{\lambda_1 v_1 & \lambda_2 v_2} = \mat{v_1 & v_2} \mat{\lambda_1 & \ \\ \ & \lambda_2} \Rightarrow M^{-1} = M^T$\\
                $\Rightarrow M= \mat{v_1 & v_2} \mat{\lambda_1 & \ \\ \ & \lambda_2} \underbrace{\mat{v_1 & v_2}^T}_{=\mat{v_1^T \\ v_2^T}} = \frac{1}{\abs{\nabla u}^2} \mat{\lambda_1 \frac{\partial u}{\partial x}(\x)^2 + \lambda_2 \frac{\partial u}{\partial y}(\x)^2  & (\lambda_1 - \lambda_2)\frac{\partial u}{\partial x} \frac{\partial u}{\partial y}(\x)\\ (\lambda_1 - \lambda_2)\frac{\partial u}{\partial x} \frac{\partial u}{\partial y}(\x) & \lambda_2 \frac{\partial u}{\partial x}(\x)^2 + \lambda_1 \frac{\partial u}{\partial y}(\x)^2}$\\
                falls $\nabla u(\x) \neq 0$, sonst $M= \mat{1 & 0 \\ 0 & 1}$.

            \subsection{Entfaltung}\index{Entfaltung}
                \begin{center}
                    \begin{tikzpicture}
                        \draw (0,0) rectangle node[] {\Large $f$} node[above,yshift=27] {original} (2,2);
                        \draw (4,0) rectangle node[] {\Large $u$} node[above,yshift=27] {geglättet} (6,2);
                        \draw[] (2.1,1) edge[bend left,->]  node[above] {$g*$} node[below] {\small Glättung} (3.9,1);
                        \draw[] (1,-0.1) edge[bend right=40,<-]  node[above] {?} node[below] {\small Entfaltung} (5,-0.1);
                    \end{tikzpicture}
                \end{center}
                Das heißt: $u=f * g$, wobei $u, \ g$ gegeben sind und $f$ gesucht ist.\\
                Alternativ kann dies als die Invertierung des Faltungsoperator $f \mapsto g * f$ betrachtet werden.
                \begin{enumerate}[label = \alph*)]
                    \item Diskreter Fall: \begin{align*}
                                            & g*f=u\\
                                            &(g*f)(j)=u(j), \ j \in \Omega\\
                                            &\sum_k g(j-k)f(k)=u(j), \ j \in \Omega\\
                                            &\Rightarrow \Omega \times \Omega \text{ Gleichungsystem}
                                            \end{align*}
                        \begin{center}
                            \begin{tikzpicture}
                                \matrix [matrix of nodes, column sep=1mm, row sep=1mm,name=M,left delimiter={(},right delimiter={)}]
                                {
                                    {$g(0)$}& {$g(-1)$}& {}& {}& {}& {}& {}& {$g(-n)$}\\
                                    {$g(1)$}& {$g(0)$}& {$g(-1)$}& {}& {}& {}& {}& {}\\
                                    {}& {}& {}& {}& {}& {}& {}& {}\\
                                    {}& {}& {}& {}& {}& {}& {}& {}\\
                                    {}& {}& {}& {}& {}& {}& {}& {}\\
                                    {}& {}& {}& {}& {}& {}& {}& {}\\
                                    {}& {}& {}& {}& {}& {}& {}& {\phantom{g(-0)}}\\
                                    {}& {}& {}& {}& {}& {}& {}& {\phantom{g(0)}}\\
                                    {}& {}& {}& {}& {}& {}& {}& {\phantom{g(0)}}\\
                                    {$g(n)$}& {}& {}& {}& {}& {}& {}& {}\\
                                    };
                                    \draw[dotted] (M-2-2)--(M-8-8);
                                    \draw[dotted] (M-2-1)--(M-9-8);
                                    \draw[dotted] (M-2-3)--(M-7-8);
                                    \draw (M-1-1) node[left,xshift =-23] {$j=0$};
                                    \draw (M-2-1) node[left,xshift =-23] {$j=1$};
                                    \draw (M-10-1) node[left,xshift =-23] {$j=n$};
                                    \draw (M-1-1) node[yshift =23] {$k=0$};
                                    \draw (M-1-2) node[yshift =23] {$k=1$};
                                    \draw (M-1-8) node[yshift =23] {$k=n$};
                                    \draw[decorate,decoration={brace,amplitude=4pt,mirror}] (-3.3,-3) -- node[below] {\mim{Toeplitz-Matrix}} (3.5,-3);
                                    \matrix [matrix of nodes, column sep=1mm, row sep=1mm,name=N,left delimiter={(},right delimiter={)},shift={(5,0)}]
                                    {
                                    {$f(0)$}\\
                                    {$f(1)$}\\
                                    {}\\
                                    {}\\
                                    {}\\
                                    {}\\
                                    {}\\
                                    {}\\
                                    {}\\
                                    {}\\
                                    {}\\
                                    {}\\
                                    {$f(n)$}\\
                                    };
                                \draw[dotted] (N-2-1) -- (N-13-1);
                                \draw (6.9,0) node[] {\Large $=$};
                                \matrix [matrix of nodes, column sep=1mm, row sep=1mm,name=O,left delimiter={(},right delimiter={)},shift={(8.5,0)}]
                                {
                                    {$u(0)$}\\
                                    {$u(1)$}\\
                                    {}\\
                                    {}\\
                                    {}\\
                                    {}\\
                                    {}\\
                                    {}\\
                                    {}\\
                                    {}\\
                                    {}\\
                                    {}\\
                                    {$u(n)$}\\
                                    };
                                \draw[dotted] (O-2-1) -- (O-13-1);
                            \end{tikzpicture}
                        \end{center}
                        \item Kontinuierlicher Fall: \begin{align*}
                                                        &(g*f)(x)=u(x),\ x \in \Omega\\
                                                        &\int_\R g(x-y) f(y) dy = u(x), \ x \in \Omega\\
                                                        &\text{Integralgleichung für die gesuchte Funktion $f$}
                                                        \end{align*}
                        $\Rightarrow$ Kontinuierliche Matrix:

                        \begin{center}
                            \begin{tikzpicture}
                                \draw (0,0) rectangle (5,5);
                                \draw[->] (-0.2,5.1) -- (-0.2,-0.1) node[below] {$x$};
                                \draw[->] (-0.1,5.2) -- (5.1,5.2) node[right] {$y$};
                                \draw[] (2.5,5.5) node {\Large $g$};
                                \draw[dotted] (0.2,4.8) -- node[sloped,above] {$g(x-y)$} (4.8,0.2);
                                \draw (5.5,5) node[above,xshift=7] {\Large $f$} rectangle (6,0);
                                \draw[<->] (6.2,4.9) node[above] {\small $a$} -- (6.2,0.1)  node[below] {\small $b$};
                                \draw (6.5,2.5) node[] {\Large $=$};
                                \draw (7,5) node[above,xshift=7] {\Large $u$} rectangle (7.5,0);
                                \draw[->] (7.7,4.9) -- (7.7,0.1) node[below] {\small $x$};
                            \end{tikzpicture}
                        \end{center}
                        Wobei $[a,b]$ die das Definitionsgebiet von $f$ ist.
                        Diese Problem is jedoch schlecht gestellt, da der Operator kompakt ist. ($\nearrow$ Datei im Studip)
                \end{enumerate}

                Wir versuchen es trotzdem zu lösen:
                \begin{align*}
                    g*f&=u & &|\cdot \F\\
                    \F(g * f)&= \F u \\
                    (2 \pi)^\frac{d}{2}(\F g)\cdot (\F f)&= \F u & &| \div (2 \pi)^\frac{d}{2}(\F g)\\
                    \F f &= \frac{\F u}{(2 \pi)^\frac{d}{2} \F g} & &| \F^{-1}
                \end{align*}
                Und erhalten:
                \begin{equation}
                    f= (2 \pi)^\frac{d}{2} \F^{-1}\left(\frac{\F u}{\F g}\right)
                \end{equation}
                Dieses kann jedoch zu Problemen führen, da etwa $g \approx 0$ werden kann.
                Je glatter $g$ ist, desto stärker klingt $(\F g)(z)$ ab für $z \to \infty$.\\
                \ \\
                Anders betrachtet:\\
                Wenn $\abs{\hat g(z)}$ für hohe Frequenzen klein ist, dann ist:
                \[A:f \mapsto g +f\]
                ein Tiefpassfilter. Nimmt man nun eine Funktion $h$ mit hoher Frequenz und großer Amplitude, dann gilt:
                \[A(f + h) = Af + \underbrace{Ah}_{\approx 0} \approx Af\]

                \underline{Problembehebung:}
                \begin{enumerate}[label = \arabic*. Ansatz:]
                    \item \ \\
                    \begin{minipage}[c]{0.55\textwidth}
                        \begin{center}
                            Approximiere die Funktion $\frac{1}{x}$ durch\\
                            \[R_\alpha = \begin{cases}
                                \frac{1}{x}, & \abs{x} > \alpha\\
                                \frac{1}{\alpha}, & x \in [0,\alpha]\\
                                \frac{1}{-\alpha}, & x \in [-\alpha,0]
                            \end{cases}\]
                            wobei $\alpha > 0$.
                        \end{center}
                        und ersetze $\displaystyle f=(2 \pi)^\frac{d}{2} \F^{-1}\left(\frac{\hat u (z)}{\hat g(z)}\right)$ durch:\\
                        \[f=(2 \pi)^\frac{d}{2} \F^{-1}\left(\hat u (z) R_\alpha(\hat g(z)) \right)\]
                        und lasse $\alpha \to 0$.
                    \end{minipage}
                    %\hfill\vrule\hfill
                    \begin{minipage}[c]{0.4\textwidth}
                        \begin{center}
                            \begin{tikzpicture}
                                \draw[->] (-2,0) -- (2,0);
                                \draw[<-] (0,2) -- (0,-2);
                                \draw[scale=0.4,domain=0.22:4.7,smooth,variable=\x,shift ={(0,0)}] plot ({\x},{1/\x});
                                \draw[scale=0.4,domain=-0.22:-4.7,smooth,variable=\x,shift ={(0,0)}] plot ({\x},{1/\x});
                                \draw[scale=0.4,domain=0.5:4.7,smooth,variable=\x,shift ={(0,0)},dashed,red,thick] plot ({\x},{1/\x});
                                \draw[scale=0.4,domain=-0.5:-4.7,smooth,variable=\x,shift ={(0,0)},dashed,red,thick] plot ({\x},{1/\x});
                                \draw (0.2,0.1) -- (0.2,-0.1) node[below] {\tiny $\alpha$};
                                \draw (-0.2,0.1) -- (-0.2,-0.1) node[below] {\tiny $-\alpha$};
                                \draw[color=red,thick] (0,0.81818) -- (0.2,0.81818);
                                \draw[color=red,thick] (0,-0.81818) -- (-0.2,-0.81818);
                                \draw (-1.9,1.9) rectangle node[align=right,left,xshift = 2] {\small $\frac{1}{x}$:\\$R_\alpha$:} (-0.6,1);
                                \draw (-1.2,1.62)--(-0.7,1.62);
                                \draw[thick,red] (-1.2,1.22)--(-0.7,1.22);
                            \end{tikzpicture}
                        \end{center}
                    \end{minipage}
                    \item Variationsrechnung:
                        \begin{enumerate}[label = \arabic*. Wunsch:]
                            \item $g *f \approx u$
                            \item $\norm{f}_2$ klein
                        \end{enumerate}
                        Minimiere nun:
                        \[\Rightarrow J(f)  \coloneqq  \norm{g * f - u}_2^2 + \lambda \norm{f}_2^2 \to min\]
                        \[\iff \int_{\R^d} ((g*f)(x) - u(x))^2 + \lambda f(x)^2 dx \to min\]
                        über die Wahl von $f \in U \coloneqq  L^2(\R^d)$.\\
                        Idee: $\F$ anwenden $\Rightarrow *$ wird zu $\C$ \underline{und} $\norm{\C}_2$ bleibt unverändert.
                        \begin{align*}
                            \Rightarrow J(f)&=\norm{g*f-u}_2^2 + \lambda\norm{f}_2^2\\
                            &=||\widehat{g*f-u}||_2^2 + \lambda ||\hat f||_2^2\\
                            &=||(2 \pi)^\frac{d}{2} \hat g \hat f -\hat u||_2^2 + \lambda||\hat f||_2^2\\
                            &=\int_{\R^d}\left[ \left|(2 \pi)^\frac{d}{2} \hat g(z) \hat f(z) - \hat u(z)\right|^2 + \lambda |\hat f(z)|^2 \right] dz \overset{f \in U}{\longrightarrow} min
                        \end{align*}
                    Strategie: Integral für jedes einzelne $z$ minimieren. Daraus erhalten wir ein optimales $\hat f$ und somit auch ein optimales $f$.\\
                    Also minimiere für jedes $z \in \R^d$
                    \[I(t) \coloneqq |(2 \pi)^\frac{d}{2} \hat g(z)t - \hat u(z)|^2 + \lambda |t|^2 \overset{t \in \mathbb C}{\longrightarrow}min\]
                    Später setzen wir $\hat f(z) \coloneqq t_{min}$, nun zur Minimierung:
                    \begin{align*}
                        I(t)&=((2 \pi)^\frac{d}{2} \hat g(z) t - \hat u(z))((2 \pi)^\frac{d}{2} \ \overline{\hat g(z)} \ \overline t -\overline{\hat u (z)}) + \lambda t \overline t\\
                        &=(2 \pi)^\frac{d}{2} \hat g(z) \ \overline{\hat g(z)} \ t \ \overline t + \lambda t \overline t - (2 \pi)^\frac{d}{2} (\hat g(z) \ \overline{\hat u(z)} \ t + \overline{\hat g(z)} \ \hat u(z) \ \overline{t}) + \hat u(z) \ \overline{\hat u(z)}\\
                        &=((2 \pi)^\frac{d}{2} |\hat g(z)|^2 + \lambda)|t|^2 - (2 \pi)^\frac{d}{2} (2 \C Re(\underbrace{\hat g(z) \ \overline{\hat u(z)}t}_{\circledast})) + |\hat u(z)|^2 \overset{t \in \mathbb C}{\longrightarrow} min\\
                    \end{align*}
                    Das Argument (Winkel) taucht nur in $\circledast$ auf\\
                    \begin{align*}
                        &\Rightarrow \text{So wählen, das $\circledast$ auf die positive reele Achse fällt}\\
                        &\Rightarrow 0=arg(\circledast)= arg(\hat g(z) \ \overline{\hat u(z)}) + arg(t)\\
                        &\Rightarrow arg(t)=-arg(\hat g(z) \ \overline{\hat u(z)})=arg(\overline{\hat g(z)} \ \hat u(z))\\
                        &\Rightarrow I(t) = ((2\pi)^\frac{d}{2} |\hat g(z)|^2 + \lambda)|t|^2 - (2\pi)^\frac{d}{2} 2 \C |\overline{\hat g(z)} \ \hat u(z)| \ |t| + |\hat u (z)|^2 \overset{|t| \in \R}{\longrightarrow}min
                    \end{align*}
                    Dieses ist nun ein Polynom in $|t|$, sodass das Minimum einfach bestimmt werden kann.\\
                    \begin{align*}
                        &0=\frac{d}{d|t|}...=2\C((2\pi)^\frac{d}{2}|\hat g(z)|^2 + \lambda)|t| - (2\pi)^\frac{d}{2} \C 2 \C |\overline{\hat g(z)} \ \hat u(z)|\\
                        &\Rightarrow |t| = \frac{(2\pi)^\frac{d}{2} \C 2 \C |\overline{\hat g(z)} \ \hat u(z)|}{2\C((2\pi)^\frac{d}{2}|\hat g(z)|^2 + \lambda)} = \frac{(2\pi)^\frac{d}{2}\C |\overline{\hat g(z)} \ \hat u(z)|}{(2\pi)^\frac{d}{2}|\hat g(z)|^2 + \lambda} \text{ und } arg(u)=arg(\overline{\hat g(z)} \ \hat u(z))\\
                        &\Rightarrow t =  \frac{(2\pi)^\frac{d}{2} \ \overline{\hat g(z)} \ \hat u(z)}{(2\pi)^\frac{d}{2}|\hat g(z)|^2 + \lambda}=:\hat f(z)
                    \end{align*}
                    Wegen
                    \[\hat f(z) = (2\pi)^\frac{d}{2} \frac{\overline{\hat g(z)}}{(2\pi)^\frac{d}{2}|\hat g(z)|^2 + \lambda} \hat u(z)\]
                    gilt
                    \begin{equation}
                        f(z) = \F^{-1}\left( \frac{\overline{\hat g(z)}}{(2\pi)^\frac{d}{2}|\hat g(z)|^2 + \lambda} \right) * u
                    \end{equation}
                    Dieses Verfahren wird \mim{$L^2$ deblurring} genannt.
                    Es gibt auch einen alternativen, algebraischen Zugang:

                    \[I(f) = \norm{g*f-u}_2^2 + \lambda \norm{f}_2^2 \overset{f}{\to} min\]
                    \[\iff \norm{\mat{g*f-u\\ \sqrt{\lambda} f}} \overset{f}{\to} min\]
                    \[\iff \norm{\mat{Af \\ \sqrt{\lambda}f} - \mat{u\\0}} = \norm{\mat{A \\ \sqrt{\lambda}}f - \mat{u\\0}} \overset{f}{\to} min \quad (A= f \mapsto g*f)\]
                    $\Rightarrow$ lineares Ausgleichsproblem.
                    \[\Rightarrow \mat{A^* &\sqrt{\lambda} I^*} \mat{A \\ \sqrt{\lambda}I}f = \mat{A^* & \sqrt{\lambda}I} \mat{u \\ 0} \quad \text{(Normalengleichung)}\]
                    \[\Rightarrow \mat{A^*A + \abs{\lambda}I}f=A^*u\]
                    \[\Rightarrow f= \mat{A^*A + \abs{\lambda}I}^{-1}A^*u\]
                    Die Inverse existiert, da $-\abs{\lambda}$ nicht im Spektrum von $A^*A$ sein kann, denn das Spektrum von $A^*A$ ist positiv und reel.
                    \item noch einmal Variationsrechnung, diesmal mit anderen Wünschen
                    \begin{enumerate}[label = \arabic*. Wunsch:]
                        \item $g *f \approx u$
                        \item $\norm{\nabla f}$ klein
                    \end{enumerate}
                    Nach analoger Rechnung wie oben erhält man:
                    \begin{equation}
                        f=\F \left( \frac{\overline{\hat g(z)}}{(2\pi)^\frac{d}{2} \ |\hat g(z)|^2 + \lambda |z|^2} \right) * u
                    \end{equation}
                    $\Rightarrow$ Dämpfung höher wenn Frequenz höher.\\
                    Dieses Verfahren nennt sich \mim{$H^1$ deblurring}.
                \end{enumerate}

    
%                \section{Restauration (Inpainting)}
  Problem: Lücken im Bild, etwa

  \begin{enumerate}
    \item Kratzer
    \item Scannerzeile kaputt
    \item Defekt in der Kamera
    \item Bewusst entferntes Objekt
  \end{enumerate}
  sollen sinnvoll und unauffällig geschlossen werden.

  Sei $f: \Omega \to F$ unser Bild jedoch mit Defekt, d.h. fehlenden Funktionswerten in $D \subset \Omega$.

  \begin{enumerate}
    \item[1. Fall:] Jeder Punkt aus $D$ hat Nachbarn in $\Omega \backslash D$.\\
    $\Rightarrow$ Lücken mittels Interpolation aus benachbarten Werten in $\Omega \backslash D$ schließen.
    \item[2. Fall:] $D$ hat innere Punkte. Diesen Fall werden wir im folgenden näher betrachten.
  \end{enumerate}

  \subsection{Frequenzraum-Ansatz}
    Zur Illustration in 1D:
    \begin{center}
      \begin{tikzpicture}[scale = 1.3]
        \draw (0,0) -- (8,0);
        \draw[thick] plot [smooth, tension = 0.8] coordinates {(1,1) (2,1.2) (3,1) (4,1.3) (5,1.7) (6,1.5) (7,2)};
        \draw[fill,color=white] (3,0.5) rectangle (5,2);
        \draw (1,-0.1) -- (1,0.1);
        \draw (3,-0.1) -- (3,0.1);
        \draw (5,-0.1) -- (5,0.1);
        \draw (7,-0.1) -- (7,0.1);
        \draw (5.3,-0.1) -- (5.3,0.1);
        \draw (2.7,-0.1) -- (2.7,0.1);
        \draw[decorate,decoration={brace,amplitude=2pt,mirror}] (1,-0.5) -- node[below] {\small $\Omega$} (7,-0.5);
        \draw[thick, line width =0.6mm] (3,0) -- node[below] {\small $D$} (5,0);
        \draw[decorate,decoration={brace,amplitude=2pt,mirror}] (2.7,-0.15) -- node[below] {\small $u$} (3,-0.15);
        \draw[decorate,decoration={brace,amplitude=2pt,mirror}] (5,-0.15) -- node[below] {\small $u$} (5.3,-0.15);
      \end{tikzpicture}
    \end{center}
    Betrachte Umgebung $u \subset \Omega \backslash D$ und errechne den Mittelwert:
    \[ m \coloneqq  \frac{1}{\abs{u}} \int_u f(x) dx\]
    von $f$ auf $u$.

    \textbf{Algorithmus:}

    \begin{tikzpicture}
        \node at (5, 10) {
    {\begin{varwidth}{\linewidth}\begin{enumerate}[label=]
        \item[\texttt{[1]}] Initialisiere $f$ auf $D$ mittels konstanter Funktion $m$:
          \begin{center}
            \begin{tikzpicture}
              \draw (0,0) -- (8,0);
              \draw[thick] plot [smooth, tension = 0.8] coordinates {(1,1) (2,1.2) (3,1) (4,1.3) (5,1.7) (6,1.5) (7,2)};
              \draw[fill,color=white] (3,0.5) rectangle (5,2);
              \draw (1,-0.1) -- (1,0.1);
              \draw (3,-0.1) -- (3,0.1);
              \draw (5,-0.1) -- (5,0.1);
              \draw (7,-0.1) -- (7,0.1);
              \draw[dashed,thick,red,line width=0.4mm] (3,1) -- (3,1.4) -- (5,1.4) -- (5,1.7);
              \draw[thick, line width =0.6mm] (3,0) -- node[below] {\small $D$} (5,0);
              \draw[<-] (3.05,1.2) -- (6,0.5);
              \draw[<-] (5.05,1.6) -- (6,0.5) node[right] {\small Sprünge};
            \end{tikzpicture}
          \end{center}
          $\Rightarrow$ Sprünge am Rand von $D$.\\
          Idee: Sprünge $\widehat =$ hochfrequente Anteile $\Rightarrow$ wende Tiefpass filter an.
        \item[\texttt{[2]}] $ f \xrightarrow{\F} \hat f \xrightarrow{\text{Lösche hohe Frequenzen}} \hat g \xrightarrow{\F^{-1}}g$
          \begin{center}
            \begin{tikzpicture}
              \draw (0,0) -- (8,0);
              \draw[thick] plot [smooth, tension = 0.8] coordinates {(1,1) (2,1.2) (3,1) (4,1.3) (5,1.7) (6,1.5) (7,2)};
              \draw[fill,color=white] (3,0.5) rectangle (5,2);
              \draw (1,-0.1) -- (1,0.1);
              \draw (3,-0.1) -- (3,0.1);
              \draw (5,-0.1) -- (5,0.1);
              \draw (7,-0.1) -- (7,0.1);
              \draw[] (3,1) -- (3,1.4) -- (5,1.4) -- (5,1.7);
              \draw[thick, line width =0.9mm] (3,0) -- node[below] {\small $D$} (5,0);
              \draw[thick,red] plot [smooth, tension = 0.8] coordinates {(1,1.03) (2,1.18) (2.9,1) (3.3,1.4) (4.7,1.4) (5.1,1.7) (6,1.53) (7,1.98)};
            \end{tikzpicture}
          \end{center}
        \item[\texttt{[3]}] Ersetze $f$ \underline{innerhalb} von $D$ durch $g$.
          \begin{center}
            \begin{tikzpicture}
              \draw (0,0) -- (8,0);
              \draw[thick] plot [smooth, tension = 0.8] coordinates {(1,1) (2,1.2) (3,1) (4,1.3) (5,1.7) (6,1.5) (7,2)};
              \draw[fill,color=white] (3,0.5) rectangle (5,2);
              \draw (1,-0.1) -- (1,0.1);
              \draw (3,-0.1) -- (3,0.1);
              \draw (5,-0.1) -- (5,0.1);
              \draw (7,-0.1) -- (7,0.1);
              \draw[thick, line width =0.9mm] (3,0) -- node[below] {\small $D$} (5,0);
              \draw[thick,red] plot [smooth, tension = 0.8] coordinates {(3,1) (3.3,1.4) (4.7,1.4) (5,1.7)};
            \end{tikzpicture}
          \end{center}
      \end{enumerate}\end{varwidth}}};
      \draw[->,thick] (0,5.5) -- (-1,5.5) -- node[sloped,above]  {Iteriere} node[sloped,below] {\footnotesize Abbruch falls $\norm{f-g}<$ threshold} (-1,10.9) -- (-0.5,10.9);
    \end{tikzpicture}

  \subsection{PDE-Transport-Diffusions-Ansatz}
    \begin{minipage}[c]{0.25\linewidth}
      \begin{center}
        \begin{tikzpicture}[scale=0.7]
          \draw[thick] (0,0) rectangle (5,5);
          \draw plot [smooth cycle,tension=1] coordinates {(2.5,1.5) (2,2.5) (3.5,2.9) (4,1.5)};
          \draw (3,2) node[] {\large $D$};
          \draw (1,4) node[] {\large $\Omega$};
          \draw[<-] (3.2,2.5) to[bend left] (4,3.5);
          \draw[<-] (3.4,2.3) to[bend left] (4.2,3.3);
          \draw[<-] (3.6,2.1) to[bend left] (4.4,3.1);
        \end{tikzpicture}
      \end{center}
    \end{minipage}
    \hfill
    \begin{minipage}[c]{0.65\linewidth}
        \begin{enumerate}
          \item[Idee:] Informationen aus $\Omega \backslash D$ nach $D$ ''hineinragen''.
          \item[Referenzen:] \begin{enumerate}
            \item[\textbullet] Weichert 1998
            \item[\textbullet] Bornemann \& März 2007
          \end{enumerate}
        \end{enumerate}
    \end{minipage}
    \ \\
    \ \\
    \hfill\\
    \textbf{Skalierung der Diffusion:}
    \[\frac{\partial u}{\partial t} = div(M \nabla u)\]
    \[\text{mit } M = \srmatrix{ | & |\\ v_1 & v_2\\ | & |} \mat{\lambda_1 & \ \\ & \lambda_2} \srmatrix{- & v_1^T & -\\ \\- & v_2^T & -} \in \R^{2 \times 2}\]
    Wobei $v_1 \perp v_2$ die Eigenvektoren des sogenannten \underline{doppelt geglätteten Strukturtensors} \index{doppelt geglätteter Strukturtensor}
    \[J= G_\rho * \biggl[ \underbrace{\srmatrix{| \\ \nabla (G_\sigma * u)\\ |}}_{2 \times 1} \cdot \underbrace{\srmatrix{- & \nabla^T(G_\sigma * u) & -}}_{1 \times 2} \biggr] \]
    sind.\\
    \begin{enumerate}
      \item[$\Rightarrow$] $v_1$ Richtung mit maximalem Kontrast mit EW $\mu_1$
      \item[] $v_2$ Richtung mit minimalem Kontrast mit EW $\mu_2$, genannt \mim{Kohärentsrichtung}.
      \item[] Hierbei ist $\mu_1 \geq \mu_2$.
    \end{enumerate}

    \begin{center}
        \begin{tikzpicture}[scale=0.7]
          \draw[thick] (0,0) rectangle (5,5);
          \draw plot [smooth cycle,tension=1] coordinates {(2.5,1.5) (2,2.5) (3.5,2.9) (4,1.5)};
          \draw (3,2) node[] {\large $D$};
          \draw (1,4) node[] {\large $\Omega$};
          \draw[<-] (3.2,2.5) to[bend left] (4,3.5);
          \draw[<-] (3.4,2.3) to[bend left] (4.2,3.3);
          \draw[<-] (3.6,2.1) to[bend left] (4.4,3.1);
          \draw[->,red!85!black,thick] (3.82,2.66) -- node[sloped,above] {\small $v_1$} ++(-215:1.5);
          \draw[->,green!70!black,thick] (3.82,2.66) -- node[sloped,below] {\small $v_2$} ++(-125:1.5);
        \end{tikzpicture}
      \end{center}

      Fälle:
      \begin{enumerate}
        \item[]
        \begin{minipage}[c]{0.25\linewidth}
          $\mu_1 \gg \mu_2 \approx 0$
        \end{minipage}
        \begin{minipage}[c]{0.25\linewidth}
          \begin{tikzpicture}[scale =0.6]
            \draw[<-] (3.2,2.5) to[bend left] (4,3.5);
            \draw[<-] (3.4,2.3) to[bend left] (4.2,3.3);
            \draw[<-] (3.6,2.1) to[bend left] (4.4,3.1);
            \draw[->,red!85!black,thick] (3.82,2.66) -- node[sloped,above] {\small $v_1$} ++(-215:1.5);
          \draw[->,green!70!black,thick] (3.82,2.66) -- node[sloped,below] {\small $v_2$} ++(-125:1.5);
          \end{tikzpicture}
        \end{minipage}
                \item[]
        \begin{minipage}[c]{0.25\linewidth}
          $\mu_1 \approx \mu_2 \approx 0$
        \end{minipage}
        \begin{minipage}[c]{0.12\linewidth}
          \begin{tikzpicture}[scale =0.6]
            \definecolor{left} {HTML}{001528}
            \draw[shading = axis,rectangle, left color=left!70!white, right color=left!60!white,shading angle=135, anchor=north,fill,draw=white] (0,0) rectangle (2,2);
          \end{tikzpicture}
        \end{minipage}
        \begin{minipage}[c]{0.3\linewidth}
          Lokal keine Struktur.
        \end{minipage}
        \item[]
        \begin{minipage}[c]{0.25\linewidth}
          $\mu_1 \approx \mu_2 \gg 0$
        \end{minipage}
        \begin{minipage}[c]{0.12\linewidth}
          \begin{tikzpicture}[scale =0.6]
            \definecolor{left} {HTML}{001528}
            \draw[shading = axis,rectangle, left color=left!20!white, right color=left!30!white,shading angle=135, anchor=north,fill,draw=white] (0,0) rectangle (2,2);
            \draw[line width = 0.8mm] (0,0) -- (2,0) -- (2,2);
          \end{tikzpicture}
        \end{minipage}
        \begin{minipage}[c]{0.3\linewidth}
          Kanten.
        \end{minipage}
      \end{enumerate}

      Die Werte $\lambda_1$ und $\lambda_2$ werden wie folgt gewählt:\\
      $\alpha \in (0,1)$ wird festgehalten.

      \begin{center}
        \begin{tikzpicture}[scale = 1]
            \draw[] (0,0) -- (8,0);
            \draw (0.25,-0.1) -- (0.25,0.1) node[above] {\small $0$};
            \draw (7.75,-0.1) -- (7.75,0.1) node[above] {\small $1$};
            \draw (1.25,-0.1) node[below] {\small $\lambda_1$} -- (1.25,0.1) node[above] {\small $\alpha$};
            \draw[->] (4.2,-0.7) to[out=180, in =-40] node[sloped,below] {\footnotesize falls $\mu_1 \approx \mu_2$} (1.75,-0.1);
            \draw[<-] (7.25,-0.1) to[in=0,out =-140] node[sloped,below] {\footnotesize falls $\mu_1 \gg \mu_2$} (4.8,-0.7) node[left] {\small $\lambda_2$};
          \end{tikzpicture}
      \end{center}

      \[\lambda_1  \coloneqq  \alpha, \ \lambda_2  \coloneqq  \alpha + (1 - \alpha)(1-g(\mu_1 - \mu_2))\]

      wobei $g$ wie bei Perona Malik gewählt wird, also:

      \[g(x)=\frac{1}{1 + \frac{s^2}{\kappa^2}}\]

      Dieses wird \mim{Kohärenz verstärkende Diffusion} genannt.

      \subsection{Variationsansatz}


      \begin{minipage}[c]{0.5\linewidth}
        \begin{enumerate}
          \item[geg.:] $f$ auf $\Omega \backslash D$
          \item[ges.:] $u$ auf $\Omega$
          \item[Wunsch 1:] $u=f$ auf $\Omega \backslash D$
          \item[Wunsch 2:] $\norm{\nabla u}$ klein auf $\Omega$
        \end{enumerate}
    \end{minipage}
    \hfill
    \begin{minipage}[c]{0.35\linewidth}
      \begin{center}
        \begin{tikzpicture}[scale=0.7]
          \draw[thick] (0,0) rectangle (5,5);
          \draw plot [smooth cycle,tension=1] coordinates {(2.5,1.5) (2,2.5) (3.5,2.9) (4,1.5)};
          \draw (3,2) node[] {\large $D$};
          \draw (1,4) node[] {\large $\Omega$};
        \end{tikzpicture}
      \end{center}
    \end{minipage}\\
    \hfill\\ \hfill\\
    Daraus folgt:

    \[ J(u) \coloneqq  \norm{\nabla u}_2^2 \to \text{min} \]

    auf

    \[U \coloneqq \{u \in W^{1,2}(\Omega) : u|_{\Omega \backslash D} = f\} \]

    Angenommen, $u \in U$ minimiert $J$, dann folgt für beliebige $v \in W^{1,2} (\Omega)$ mit $v|_{\Omega \backslash D}=0$:

    \begin{align*}
      0 &= \underset{t \to 0}{lim} \frac{J(u+tv) - J(u)}{t} = \underset{t \to 0}{lim} \frac{1}{t} \int_\Omega ||\underbrace{\nabla(u+tv)(x)}_{\nabla u(x) + t \nabla v(x)}||^2 - \norm{\nabla u(x)}^2 dx\\
      &=\underset{t \to 0}{lim} \frac{1}{t} \int_\Omega \skprod{\nabla u(x) + t \nabla v(x)}{\nabla u(x) + t \nabla v(x)} - \skprod{\nabla u (x)}{\nabla u (x)} dx\\
      &=\underset{t \to 0}{lim} \frac{1}{t} \int_\Omega t^2 \norm{\nabla v(x)}^2 + 2 t \skprod{\nabla u(x)}{\nabla v(x)} dx =  \underset{t \to 0}{lim} \int_\Omega t \norm{\nabla v(x)}^2 + 2 \skprod{\nabla u(x)}{\nabla v(x)} dx\\
      &=2 \int_D \skprod{\nabla u(x)}{\nabla v(x)} dx \overset{\text{Greensche Formel}}{=} 2 \biggl( \overbrace{\int_{\delta D} \frac{\partial u}{\partial n} \underbrace{v(x)}_{0} ds(x)}^{0} - \int_D \Delta u(x) v(x) dx \biggr)\\
      &=2 \int_D \Delta u(x) v(x) dx \Rightarrow \Delta u =0 \text{ in } D
    \end{align*}

    
%    \section{Segmentierung}

Dieses ist die Zerlegung eines Bildes in verschiedene Objekt.\\
Eine einfache Methode hierfür ist das \mim{Histogramm thresholding}:
\begin{center}
  \begin{tikzpicture}
    \draw (1,0) -- (2,0) -- (2,1) -- (3,1) -- (3,0) -- (4,0) -- (4,1) -- (5.5,1) -- (5.5,0) -- (7,0);
    \draw[dotted] (0.5,0.5) -- (7.5,0.5) node[right] {threshold};
    \draw[line width=0.5mm,red,dashed] (2,1) -- (3,1);
    \draw[line width=0.5mm,red,dashed] (4,1) -- (5.5,1);
    \draw[line width=0.5mm,green!60!black,dashed] (1,0) -- (2,0);
    \draw[line width=0.5mm,green!60!black,dashed] (3,0) -- (4,0);
    \draw[line width=0.5mm,green!60!black,dashed] (5.5,0) -- (7,0);
  \end{tikzpicture}
\end{center}
So kann ein Bild in mehrere Objekte zerlegt werden.\\
Hierbei können jedoch diverse Probleme auftreten, die vor der Zerlegung durch preprocessing behoben werden sollten. Einige der preprocessing Methoden sind:
\begin{enumerate}
  \item[-] Entrauschen $\nearrow$ 5.7
  \item[-] Farbraum optimal ausnutzen $\nearrow$ 3.2
  \item[-] \mim{Beleuchtungsausgleich}
\end{enumerate}
Dieser Beleuchtungsausgleich wurde noch nicht vorher besprochen, das Problem:

\begin{center}
  \begin{tikzpicture}
    \draw (0,0) -- ++(20:1) -- ++(0,1) -- ++(20:1) -- ++(0,-1) -- ++(20:1) -- ++(0,1) -- ++(20:1.5) -- ++(0,-1) -- ++(20:1.5);
    \draw[dotted] (-0.5,1.5) -- (6,1.5) node[right] {threshold??};
  \end{tikzpicture}
\end{center}

Anstatt eines \quo{geraden} Bildes ist das Bild, etwa durch Beleuchtung, \quo{gekippt} und der Ansatz mittels Histogramthresholding würde nicht das gewünschte Ergebnis erzielen.\\

In 2D könnte dies so aus sehen.

\begin{center}
  \begin{tikzpicture}
    \draw[white] (0,0) rectangle node[draw,black] {\includegraphics[scale = 0.2]{Bild3plusgrad.png}} (3,3);
    \draw (1.5,3) node[above] {\LARGE $f$};
    \draw (1.5,0) node[below] {gegeben};
    \draw (3.5,1.5) node[] {\LARGE $=$};
    \draw[white] (4,0) rectangle node[draw,black] {\includegraphics[scale = 0.2]{Bild3.png}} (7,3);
    \draw (5.5,3) node[above] {\LARGE $u$};
    \draw (5.5,0) node[below] {erwünscht};
    \draw (7.5,1.5) node[] {\LARGE $+$};
    \draw[white] (8,0) rectangle node[draw,black] {\includegraphics[scale = 0.2]{Bild3grad.png}} (11,3);
    \draw (9.5,3) node[above] {\LARGE $v$};
    \draw (9.5,0) node[below] {gesucht};
    \draw (11.75,1.5) node[] {\LARGE $\Longleftrightarrow$};
    \draw[white] (12.5,0) rectangle node[draw,black] {\includegraphics[scale = 0.2]{Bild3thresh.png}} (15.5,3);
    \draw (14,3) node[above] {};
    \draw (14,0) node[below] {Ergebnis durch thresholding};
  \end{tikzpicture}
\end{center}

Rechts ist das Bild, das durch das Beschriebene Histogramm thresholding dargestellt wurde zu sehen. Methoden um durch preprocessing den gesuchten Gradienten zu entfernen werden im folgenden beschrieben.

\subsection{Beleuchtungsausgleich}

\begin{enumerate}
  \item[Einfachster Fall:] $v$ konstant. In diesem Fall kann man etwa eine \quo{Leeraufnahme} machen, $v=f$ setzen und dieses $v$ in allen folgenden Aufnahmen subtrahieren.
  \item[Normalfall:] $v$ ändert sich bei Jeder Aufnahme. Hierbei gibt es mehrere Ansätze:
\end{enumerate}

\begin{enumerate}[label = \alph*)]
  \item \mim{Lineare Regression}: \\
  \begin{minipage}[c]{0.6\linewidth}
          \item[] Wir Unterstellen das der Verlauf \mim{affin-linear} ist, d.h.:
          \[v(x,y) = a x + b y + c\]
          Diese Parameter $a$, $b$, $c$ gilt es nun zu schätzen.
    \end{minipage}
    \hfill
    \begin{minipage}[c]{0.35\linewidth}
              \begin{center}
        \begin{tikzpicture}[scale=0.8]
          \draw[thick] (0,0) rectangle node[] {\large $\Omega$} (3,3);
          \draw[->] (-0.2,3.2) -- (3.2,3.2) node[right] {\small $x$};
          \draw[->] (-0.2,3.2) -- (-0.2,-0.2) node[below] {\small $y$};
          \foreach \x in {1,...,5}
            \foreach \y in {1,...,5}{
              \draw (\x / 2,\y / 2) node[circle,fill=black,inner sep=0pt] {};
          }
          \draw (2.05,2.05) -- (3.5,2.3) node[right] {\small $(x_i,y_j)$};
        \end{tikzpicture}
      \end{center}
    \end{minipage}
    Dazu soll gelten
    \[\forall (x,y) \in \Omega : ax+by+c \approx f(x,y)\]
    um dieses zu erfüllen wird eine Stichprobe von endlich vielen Punkten $(x_i,y_i)$ aus $\Omega$ gewählt und durch diese ein Gleichungssystem gebildet.

    \begin{gather*}
    ax_1+by_1+c \approx f(x_1,y_1)\\
    \vdots \\
    ax_n+by_n+c \approx f(x_n,y_n)
    \end{gather*}
    In Matrix Form ergibt sich:

    \[\underbrace{\mat{x_1 & y_1 & 1 \\ \vdots & \vdots & \vdots \\ \vdots & \vdots & \vdots \\ \vdots & \vdots & \vdots \\ x_n & y_n & 1}}_{A} \underbrace{\mat{a \\ b \\ c}}_{w} \approx \underbrace{\mat{f(x_1,y_1) \\ \vdots \\ \vdots \\ \vdots \\ f(x_n,y_n)}}_{z}\]
    Die optimale Lösung dieses Problems kann über die \mim{Normalengleichung} berechnet werden.

    \[A^T A w = A^T z\]

    Daraus erhält man $w$, somit auch $a$, $b$, $c$ und schlussendlich $v \Rightarrow u = f-v$.\\
    Anschließend wird noch ein Histogramm stretching durchgeführt um das finale Bild zu erhalten.

    \item \mim{Polynomiale Regression}:
    Ähnlich zur linearen Regression nun wird hierbei keine affin-lineare Funktion, sondern ein Polynom genutzt. Für ein Polynom zweiten gerades kann etwa die Funktion
    \[v(x,y) = ax^2 by^2 cxy+ dx +ey +f\]
    gewählt werden. Wieder entsteht ein Gleichungssystem:

    \[\underbrace{\mat{x_1^2 & y_1^2 & x_1y_1 & x_1 & y_1 & 1 \\ \vdots & \vdots & \vdots & \vdots & \vdots & \vdots \\ \vdots & \vdots & \vdots & \vdots & \vdots & \vdots \\ \vdots & \vdots & \vdots & \vdots & \vdots & \vdots \\ x_n^2 & y_n^2 & x_ny_n & x_n & y_n & 1}}_{A} \underbrace{\mat{a \\ b \\ c \\ d \\e \\ f}}_{w} \approx \underbrace{\mat{f(x_1,y_1 \\ \vdots \\ \vdots \\ \vdots \\ f(x_n,y_n)}}_{z}\]

  \item \mim{Trigonometrisches Polynom}
  Hierbei wird $v$ in den niedrigfrequenten Anteilen von $f$ gesucht.

  \begin{center}
    \begin{tikzpicture}
      \draw (0,0.25) node[left] {$f:$};
      \draw[step = 0.5] (0,0) grid (1.75,0.5);
      \draw (2,0.25) node[] {$...$};
      \draw[step = 0.5] (2.25,0) grid (4,0.5);
      \draw (0.25,0) node[below] {$0$};
      \draw (0.75,0) node[below] {$1$};
      \draw (1.25,0) node[below] {$...$};
      \draw (3.25,0) node[below] {$...$};
      \draw (3.75,0) node[below] {\footnotesize $n-1$};
    \end{tikzpicture}
  \end{center}

  Es ergibt sich $\hat f$:

  \[\hat f_k = \sum_{m=0}^{n-1} f_k \bigl(\underbrace{e^{- i 2 \pi \frac{k}{n}}}_{w_k}\bigr)^m \quad , \ k= 0,1,...,n-1\]

  Mittels der FFT (Fast Fourier Transformation) ergibt sich $\hat f$ zu:

  \begin{center}
    \begin{tikzpicture}[scale=1.5]
      \draw (0,0.25) node[left] {$\hat f:$};
      \draw[step = 0.5] (0,0) grid (1.75,0.5);
      \draw (2,0.25) node[] {$...$};
      \draw[step = 0.5] (2.25,0) grid (4,0.5);
      \draw (0.25,0) node[below] {$0$};
      \draw (0.75,0) node[below] {$1$};
      \draw (0.25,0.25) node[] {\footnotesize $w_0$};
      \draw (0.75,0.25) node[] {\footnotesize$w_1$};
      \draw (1.25,0) node[below] {$...$};
      \draw (3.25,0) node[below] {$...$};
      \draw (1.25,0.25) node[] {\footnotesize$\cdots$};
      \draw (3.75,0) node[below] {\footnotesize $n-1$};
      \draw[<->] (0.75,-0.5) -- (0.75,-0.75) -- node[below] {\footnotesize komplex konjugiert} (3.75,-0.75) -- (3.75,-0.5);
      \draw[<->] (0.5, 0.75) to[bend left] node [above] {\footnotesize niedrige Frequenzen} (3.5,0.75);
      \draw[<-] (2,0.75) to[bend left=10] (3.5,1) node[right] {\footnotesize hohe Frequenzen};
    \end{tikzpicture}
  \end{center}
  Durch entfernen dieser niedrigen Frequenzen ergibt sich $u$.
Ähnliches funktioniert auch in 2D:

    \begin{center}
    \begin{tikzpicture}[scale=1]
      \draw (0,0) rectangle (3,5);
      \draw (1.5,5) node[above] {$n$};
      \draw (0,2.5) node[left] {$m$};
      \draw (-0.25,4.25) node[left] {\large $f:$};
      \draw[->] (3.2,2.7) to[bend left] node[above] {\tt{FFT2($f$)}} (4.8,2.7);
      \draw[<-] (3.2,2.3) to[bend right] node[below] {\tt{iFFT2($f$)}} (4.8,2.3);
      \draw (5,0) rectangle (8,5);
      \draw (6.5,5) node[above] {$n$};
      \draw (8,2.5) node[right] {$m$};
      \draw (4.75,4.25) node[left] {\large $\hat f:$};
      \draw[fill = black!30] (6.5,2.5) circle (0.75);
      \draw (6.6,2.7) -- (8.5,3.5) node[right] {hohe Frequenzen};
      \draw (5.5,1) -- (8.3,1.5) node[right] {niedrige Frequenzen};
    \end{tikzpicture}
  \end{center}
\end{enumerate}

\subsection{Thresholding als Variationsproblem}

\begin{enumerate}
\item[geg:] Bild $u: \Omega \to F=[0,1]$ und Schwellenwert $t \in (0,1)$
\item[$\Rightarrow$:]   $\begin{array}{lr}
                            \Omega_0 = \{x \in \Omega : u(x) \leq t \} \to \text{schwarz}  \\
                            \Omega_1 = \{x \in \Omega : u(x) > t \} \to \text{weiß}
                        \end{array} \bigg \}$ soll in ein Variationsproblem umformuliert werden.
\end{enumerate}

Setze dazu:

\begin{equation}\label{eq.9.1}
J(v) \coloneqq -\int_\Omega (u(x) -t) \C v(x) dx \to \ min
\end{equation}

mit $v \in U \coloneqq  \bigl \{ v \in \Omega \to \{ 0,1\} \text{ \ (oder $[0,1]$)} \bigr\}$, dies ist jedoch kein Vektorraum.

Die Lösung dieses Problems ist offenbar

\[v = \chi_{\Omega_1}, \ \text{d.h.} \begin{cases}
0, & x \in \Omega_0 \\
1, & x \in \Omega_1
\end{cases} \]

illustiert hier:


\begin{center}
    \begin{tikzpicture}
        \draw[thick] (0,0) rectangle (8,5);
        \draw[fill, black!30, draw = black] plot[smooth cycle,tension=0.9] coordinates {(7,1) (7.1,2) (6.8,3) (6,4) (5,3.3) (4,2.5) (3.5,1.2) (5.5,0.7) };
        \draw (0.5,4.5) node[] {\LARGE $\Omega$};
        \draw[white] (0.3, 4.2) -- node[black,below,sloped,style={align=center},draw,fill=black!5] {\footnotesize $u(x) \leq t$ \\ \footnotesize $u(x) -t \leq 0$ \\ \footnotesize $v(x) = 0$} ++(0:3);
        \draw (1,0.8) node[] {$\Omega_0$};
        \draw[] (5.5, 2.8)  node[black,below,sloped,style={align=center},draw,fill=black!5] {\footnotesize $u(x) > t$ \\ \footnotesize $u(x) -t > 0$ \\ \footnotesize $v(x) = 1$};
        \draw (4.2,1.4) node[] {$\Omega_1$};
        \draw[fill, black!30, draw = black] (2.5,2) rectangle ++(0.11,0.08);
        \draw[fill, black!30, draw = black] (4,3.7) rectangle ++(0.08,0.11);
        \draw[fill, black!30, draw = black] (7.2,4.1) rectangle ++(0.07,0.13);
    \end{tikzpicture}
\end{center}

Diese Herangehensweise ist jedoch kompliziert, um sie zu begründen betrachten wir die Flecken die neben der großen Masse in der obing Illustration zu sehen sind und etwa durch Rauschen entstanden seien könnten. Durch Verallgemeinerung des in \ref{eq.9.1} gegebenen Funktionals können wir die Zerlegung von $\Omega$ in $\Omega_0$ und $\Omega_1$ weniger Anfällig gegenüber Rauschen und sonstigen kleinen Strukturen machen.

Dazu sei

\begin{equation}\label{eq.9.2}
TV(v) = \int_\Omega \abs{\nabla v(x)} dx
\end{equation}

die sogennate \mim{Totalvariation} einer Funktion auf $\Omega$.

Illustration in 1D:

\begin{center}
    \begin{tikzpicture}
        \draw (0,0) -- (9,0);
        \draw (1,1.5) node[] {$\Omega = [a,b]$};
        \draw plot[smooth,tension=0.9] coordinates {(1,0.25) (1.7,0.7) (2.5,1) (3.5,1.5) (5,1.9) (6.1,2.5) (7.25,2.9) (8,3)} node[right] {\small $v_1$ monoton};
        \draw[red] plot[smooth,tension=0.9] coordinates {(1,2.85) (2,2.7) (3,2.5) (4.5,1.8)  (6.5, 1.3) (7.25,0.9) (8,0.5)} node[black,right] {\small $v_2$ monoton};
        \draw[thick] (1,0.1) -- (1,-0.1) node[below] {\small $a$};
        \draw[thick] (8,0.1) -- (8,-0.1) node[below] {\small $b$};
        \draw (9,1.7) node[style={align=left},right] {\small $TV(v_1) = v_1(b) -v_1(a)$ \\ \small $TV(v_2) = \abs{v_2(b) - v_2(a)}$};
        \draw (0,-4) -- (9,-4);
        \draw plot[smooth,tension=0.8] coordinates {(1,-3.8) (2,-2.3) (3, -2.7) (4,-1.5) (5,-3.8) (6,-2) (7,-2.2) (8,-2)};
        \draw[thick] (1,-3.9) -- (1,-4.1) node[below] {\small $a$};
        \draw[thick] (8,-3.9) -- (8,-4.1) node[below] {\small $b$};
        \draw (9,-2.3) node[style={align=left},right] {\small Hier ist die Totalvariation die Summe \\ \small aller Höhenanstiege, wobei sowohl \\
        \small ab- und anstiege positiv gezählt werden.};
    \end{tikzpicture}
\end{center}

In Abschnitt 11.2 werden wir weiterhin sehen, dass die Totalvariation sich auch auf nicht stetigen Funktionen etwa $\chi_{(0,\infty)}$ berechnen lässt, obwohl für diese der Gradient nicht definiert ist. Für die eben genannte Funktion beträgt die Totalvariationetwa $1$.

Nun in 2D:

\begin{center}
    \begin{tikzpicture}
        \draw[thick] (0,0) rectangle (5,3);
        \draw[fill, black!30, draw = black, thick] plot[smooth cycle,tension=1] coordinates {(4,0.5) (4.1,1) (3,1.5) (2.5,1) (2.4,0.5) (3,0.4)};
        \draw (0.5,2.5) node[] {\large $0$};
        \draw (3.1,0.9) node[] {\large $1$};
        \draw (2.45,1.15) node[] {\small $v$};
    \end{tikzpicture}
\end{center}

$TV(v)$ ist hier die Länge der Kante die $\Omega_1$ von $\Omega_0$ trennt.\\
Die Idee ist nun $TV(v)$ als Strafterm zu \ref{eq.9.1} hinzu zu addieren,


\begin{equation}\label{eq.9.3}
\widetilde J(v) \coloneqq  \underbrace{-\int_\Omega (u(x)-t) \C v(x) dx}_{J(x)} + \lambda \C TV(v) \to \ min
\end{equation}
wobei $v \in U$ ist.
Der Effekt dieser Herangehensweise wird nun illustriert.

\begin{center}
    \begin{tikzpicture}
        \draw[thick] (0,0) rectangle (8,5);
        \draw[fill, black!30, draw = black] plot[smooth cycle,tension=0.9] coordinates {(7,1) (7.1,2) (6.8,3) (6,4) (5,3.3) (4,2.5) (3.5,1.2) (5.5,0.7) };
        \draw (0.5,4.5) node[] {\LARGE $\Omega$};
        \draw (1,0.8) node[] {$\Omega_0$};
        \draw (4.2,1.4) node[] {$\Omega_1$};
        \draw[fill, black!30, draw = black] (2.5,2) rectangle ++(0.11,0.08);
        \draw[fill, black!30, draw = black] (4,3.7) rectangle ++(0.08,0.11);
        \draw[fill, black!30, draw = black] (7.2,4.1) rectangle ++(0.07,0.13);
        \draw[fill, black!10, draw = black] (1,2.5) rectangle ++(0.4,0.4);
        \draw[->] (-0.5,3.5) node[left] {$M$} -- (0.95,2.95);
        \draw[] (-0.5,2.5) node[left,draw] {\small $u(x) \geq t, \ t(x)=0$ oder $1$?} -- (1.2,2.7);
        \draw[decorate,decoration={brace,amplitude=2pt}] (1.45,2.9) -- node[right] {\small $R$} (1.45,2.5);
    \end{tikzpicture}
\end{center}

$\begin{array}{cccccccc}
\text{Falls:} & v|_M = 0:&a & +& \lambda \C b \\
\text{Falls:} & v|_M = 1:&a-dR^2 & + &\lambda \C (b+4R)
\end{array} \bigg \}$ Also $M \to 0 \iff 4 \lambda - dR < 0 \iff R > \frac{4 \lambda}{d}$ %??

Das heißt das kleine Segment $M$ wird durch \ref{eq.9.3} \quo{erkannt} falls seine Kantenlänge $R>\frac{4 \lambda}{d}$ ist. Somit können die Abmessung der kleinsten zu segmentierenden Strukturen über $\lambda$ gesteuert werden.

\subsection{Segmentierung nach Mumford und Shah}

\begin{minipage}{.65\textwidth}
    \begin{enumerate}
        \item[Wieder:] Variationsrechnung
        \item[Diesmal:] Ohne Vorkenntnis des thresholds
        \item[Idee:] Bild zerlegen in \quo{glatte} Teile getrennt durch Sprünge an deren Rändern.
        \item[] \
        \item[geg.:] Bild $u$
        \item[ges.:] Stückweise glattes Bild $v$ mit Randkurve $\Gamma$
    \end{enumerate}
    \begin{enumerate}
        \item[1. Wunsch:] $u \approx v$ auf ganz $\Omega$
        \item[2. Wunsch:] $\nabla v$ klein auf $\Omega \backslash \Gamma$
    \end{enumerate}
\end{minipage}%
\begin{minipage}{0.3\textwidth}
\begin{center}
    \begin{tikzpicture}
        \draw[thick] (0,0) rectangle (4,5);
        \draw[fill, black!30, draw = black,thick] plot[smooth cycle,tension=0.9] coordinates {(3.6,0.5) (3.3,2) (3.4,2.8) (3.8,2.3) (3.4,3.5) (2.9,4) (3.2,4.3) (2.7,4.5) (2.3,4.3) (2.5,4)  (2,3.5) (1.6,2.3) (2,2.8) (1.9,2) (2.2,0.5) (2.4,1.9) (2.8,1.9)};
        \draw (1.65,3.2) node[] {$\Gamma$};
    \end{tikzpicture}%%(2.7,4)
\end{center}
\end{minipage}

\[\Rightarrow J(v,\Gamma)  \coloneqq  \underbrace{\norm{u-v}_{2,\Omega}^2}_{1. \text{ Wunsch}}  + \lambda \underbrace{\norm{\nabla v}_{2,\Omega \backslash \Gamma}^2}_{2. \text{ Wunsch}} \to \ min \]

Wie im letzten Abschnitt \ref{eq.9.2} soll nun noch die Segmentierung sehr kleiner Strukturen vermieden werden, indem man zu $J$ einen entsprechenden Strafterm addiert.

\begin{equation*}
\widetilde J(v,\Gamma)  \coloneqq  \norm{u-v}_{2,\Gamma}^2 + \lambda \norm{\nabla v}_{2,\Omega \backslash \Gamma}^2 + \mu \ \text{Länge}(\Gamma) \to \ min
\end{equation*}

Dieses wird \mim{Mumford-Shah-Funktional} (1989) genannt.

\begin{enumerate}
\item[$\lambda$] bestimmt die \quo{Flachheit} von $v$ auf $\Omega \backslash \Gamma$
\item[$\mu$] ist proportional zur Größe der kleinsten zu segmentierenden Struktur.
\end{enumerate}

Die numerische Lösung dieses Problemes ist jedoch sehr kompliziert, da neben $v$ auch die Kurve $\Gamma$ variiert wird, desßhalb existert eine \quo{vereinfachte} Version das dieses Problem approximiert.

Mumford-Shah (1989):

\[\widetilde J(v,\Gamma)  \coloneqq  \norm{u-v}_{2,\Gamma}^2 + \lambda \norm{\nabla v}_{2,\Omega \backslash \Gamma}^2 + \mu \ \text{Länge}(\Gamma)\]

Strekalovskiy \& Cremers (2014):

\[\widetilde J(v)  \coloneqq  \int_\Omega \left[ \abs{u(x) - v(x)}^2 + \text{min}(\lambda \abs{\nabla c(x)}^2, \mu) \right] dx \]

Die Minimierung und $\mu$ simulieren den Sprung an der Randkurve $\Gamma$.
    
%\input{chap09}    
%\section{Registrierung}

\begin{enumerate}
    \item[geg.:] \begin{align*}
        \text{Referenzbild} & & \text{und} & &  \text{Objektbild}\\
        u_0: \Omega_0 \to F & & \ & & u:\Omega \to F
    \end{align*}
    \begin{center}
        \begin{tikzpicture}
            \draw (-2,0) node[left]{\includegraphics[scale=0.15]{Bild4.png}};
            \draw[->] (-2,0) to[bend left=10] node[above] {?} (2.7,0);
            \draw (2.7,0) node[right]{\includegraphics[scale=0.15]{Bild4skew.png}};
        \end{tikzpicture}
    \end{center}
    \item[ges.:] Transformation/Deformation d.h. $d:\Omega_0 \to \Omega$, die beide Bilder bestmöglich in Einklang bringt, d.h.
        \begin{align*}
            u \circ d &\approx u_0 \\
            u(d(x)) &\approx u_0(x) \quad\text{für alle } x \in \Omega_0
        \end{align*}
    \begin{center}
        \begin{tikzpicture}
            \draw (-1.5,0) node[](A) {\Large$\Omega_0$};
            \draw (1.5,0) node[](B) {\Large$\Omega$};
            \draw (0,2) node[](C) {\Large$F$};
            \draw[->] (A) to[bend left=10] node[above,sloped]{$d$} (B);
            \draw[->] (A) to[bend left=10] node[above,sloped]{$u_0$} (C);
            \draw[->] (B) to[bend right=10] node[above,sloped]{$u$} (C);
            %\draw[] plot [smooth] coordinates {(A) (2,0) (C)};
            \draw[->] (A) to[out=270, in=270] (2,0) to[out=90,in=0] node[sloped,above]{$u \circ d$} (C);
        \end{tikzpicture}
    \end{center}
\end{enumerate}

Man unterscheidet zunächst in
\begin{enumerate}
    \item[\textbullet] Merkmalsbasierte Verfahren
    \item[\textbullet] Globale Verfahren
\end{enumerate}

\subsection{Merkmalsbasierte Verfahren}

Hierbei sollen endlich viele \mim{Landmarks}(\mim{Merkmale}) aus $u_0$ und $u$ paarweise in Einklang gebracht werden, hieraus erhält man endlich viele Gleichungen zur Schätzung von $d$. Diese Kontrollpunkte müssen jedoch vorher von Hand bestimmt werden und in den beiden gegebenen Bildern miteinander identifiziert werden, es ist nur schwer möglich dem Computer das Finden dieser Kontrollpunkte beizubringen.

Beispiel: $d:\R^2 \to \R^2$ linear + Translation, d.h.:
\[ d:\underbrace{\mat{x_1 \\ x_2}}_{p} \mapsto \mat{a & b \\ c & d} \C \mat{x_1 \\ x_2}+ \mat{e\\f} \coloneqq \underbrace{\mat{y_1 \\ y_2}}_{q} = \mat{ax_1 + bx_2 +e\\ cx_1 + dx_2 + f} = \underbrace{\mat{x_1 & x_2 & 0 & 0 & 1 & 0\\ 0 & 0 & x_1 & x_2 & 0 & 1}}_{A_p}\C \mat{a\\b\\c\\d\\e\\f}\;.\]

Die zu lösende Gleichung ergibt sich somit zu:
\[A_p \C \mat{a\\b\\c\\d\\e\\f} = q.\]

Sind $p_1,...,p_n \in \Omega_0$ und $q_1,...,q_n \in \Omega$ paarweise zusammen gehörende Kontrollpunkte, so kann man $a$, $b$, $c$, $d$, $e$, $f$ über das folgende Lineare Ausgleichsproblem schätzen:
\[ \mat{\ \boxed{A_{p_1}} \ \\ \ \boxed{A_{p_2}} \ \\ \cdots \\ \ \boxed{A_{p_n}} \ } \C \mat{a\\b\\c\\d\\e\\f} \approx \mat{\ \boxed{q_1 \vphantom{A_{p_1}}} \ \\ \ \boxed{q_2 \vphantom{A_{p_1}}} \ \\ \cdots \\ \ \boxed{q_n \vphantom{A_{p_1}}} \ }.\]

Dieses kann etwa über Normalengleichungen oder QR Zerlegungen geschehen:
\begin{align*}
    n<3&\text{: unterbestimmt}\\
    n=3&\text{: fair}\\
    n>3&\text{: überbestimmt}
\end{align*}

Im selben Stil können quadratische oder höhere Deformationen berechnet werden:
\[d:\underbrace{\mat{x_1 \\ x_2}}_{p} \mapsto \mat{ax_1^2 + b x_1 x_2 + c x_2^2 + d x_1 + e x_2 +f\\ g x_1^2 + hx_1x_2 + ix_2^2 + jx_1 + hx_2 +l} = \underbrace{\mat{y_1\\y_2}}_{q}\]

\[\left(
    \begin{array}{*{12}c}
        x_1^2 & x_1x_2 & x_2^2 & x_1 & x_2 & 1 & 0 & 0 & 0 & 0 & 0 & 0\\
        0 & 0 & 0 & 0 & 0 & 0 & x_1^2 & x_1x_2 & x_2^2 & x_1 & x_2 & 1
    \end{array}
    \right) \C \mat{a\\ \vdots \\ l}\]

Um die 12 gesuchten Parameter hier $a,b, ... ,l$ zu bestimmen werden $n \geq 6$ Kontrollpunkte auf jeder Seite benötigt.

Es gibt auch noch einen anderen Spezialfall der affin-linearen Deformationen, etwa über Dreh-Spiegelungen mit Verschiebung:
\[d: \mat{x_1\\x_2} \mapsto r \C \mat{\cos(\varphi) & -\sin(\varphi)\\ \sin(\varphi) & \cos(\varphi)} + \mat{e\\f}=\mat{y_1\\y_2}\]
\[\Rightarrow \mat{y_1\\y_2} = \mat{r \C \cos(\varphi)x_1 -r \C \sin(\varphi)x_2 + e\\ r \C \sin(\varphi)x_1 + r \C \cos(\varphi) x_2 + f} = \mat{x_1 & -x_2 & 1 & 0\\ x_2 & x_1 & 0 & 1} \mat{r \C \cos(\varphi)\\ r \C \sin(\varphi)\\e \\ f} \begin{array}{c}
     \coloneqq g\\ \coloneqq h\\ \ \\ \
\end{array}\]
Somit brauchen wir $n \geq 2$ Kontrollpunkte um $g,h,e,f$ zu bestimmen, hierraus können dann $r \coloneqq \sqrt{g^2+h^2}$ und $\varphi=\arctan(\frac{h}{g})$ bestimmt werden.

\subsection{Globale Verfahren}

Globale Verfahren benutzen keine Manuell bestimmten Kontrollpunkte und können somit komplett durch einen Computer durchgeführt werden, wiederum wird dieses Problem mittels der Variationsrechnung formuliert.
Hierbei lautet das zu minimierende Funktional:
\[J(d) \coloneqq \underbrace{\vphantom{R(d)}D(u_0,u \circ d)}_{\text{\small Datenterm}} + \lambda \underbrace{R(d)}_{\text{\small Regularitätsterm}} \to \min\]

\begin{enumerate}
    \item Wahl des Datenterms $D$
    \begin{enumerate}
        \item Punktweise Differenz in der $\Ell^2$-Norm:
            \[D(f,g)  \coloneqq  \norm{f-g}_2^2 = \int_\Omega \abs{f(x)-g(x)}^2 \d x\]
        \item Vergleich der Grauwert-Verläufe
            \[\bar{f}  \coloneqq \ \frac{1}{\abs{\Omega}} \int_\Omega f(x) \d x, \quad \bar{g}  \coloneqq  \frac{1}{\abs{\Omega}} \int_\Omega g(x) \d x\]
            und dann Vergleich von
            \[\frac{f - \bar{f}}{\norm{f-\bar{f}}_2} \quad \text{und} \quad \frac{g - \bar{g}}{\norm{g-\bar{g}}_2}\]
            über das $\Ell^2$-Skalarprodukt
            \[\skprod{\frac{f - \bar{f}}{\norm{f-\bar{f}}_2}}{\frac{g - \bar{g}}{\norm{g-\bar{g}}_2}} \in [-1,1]\]

            \begin{enumerate}
                \item[1] bedeutet vollständige Korreliertheit mit gleicher Tendenz
                \item[-1] vollständige Korreliertheit mit entgegengesetzter Tendenz
                \item[0] bedeutet Unkorreliertheit
            \end{enumerate}

            Der sich ergebende Datenterm lautet:
            \[D(f,g)  \coloneqq  1 - \skprod{\frac{f - \bar{f}}{\norm{f-\bar{f}}_2}}{\frac{g - \bar{g}}{\norm{g-\bar{g}}_2}}\]

            Diese Verfahren nennt sich \mim{Normalized Crosscorrelation} (NCC).
            \item Punktweise Differenzen der Gradienten:
                \[D(f,g)  \coloneqq  \norm{\nabla f - \nabla g}_2^2 = \int_\Omega \abs{\nabla f - \nabla g}^2 \d x\]

            \item Vergleich der Gradientenverläufe, also NCC (wie in (b)) von $\nabla f$ und $\nabla g$.
    \end{enumerate}
    Es ist zu bemerken das (a) keine Verschiebung um eine Konstante erkennt, (b) findet sogar Transformationen der Form $f=a*g+c$.

    \item Notwendigkeit und Wahl des Regularitätsterms.
    Betrachte etwa $\begin{tabular}{|c|c|c|c|}
        \hline
        1 & 2 & 3 & 4\\
        \hline
    \end{tabular} \to \begin{tabular}{|c|c|c|c|}
        \hline
        1 & 4 & 3 & 2\\
        \hline
    \end{tabular}$, diese Zerreißung muss bestraft werden.
    Ansätze:
    \[d: \mat{x_1 \\ x_2 } \mapsto \mat{y_1(x_1,x_2) \\ y_2(x_1,x_2)}\]
    \begin{enumerate}
        \item Große Streckungen/Deformationen bestrafen
            \[R(d)  \coloneqq  \int_\Omega\bigl( \abs{\nabla y_1(x)}^2 + \abs{\nabla y_2(x)}^2   \bigr) \d x \]
            Hier werden Abbleitungen 1. Ordnung benutzt.
        \item Große Krümmungen bestrafen
            \[R(d)  \coloneqq  \int_\Omega\bigl( \abs{\Delta y_1(x)}^2 + \abs{\Delta y_2(x)}^2   \bigr) \d x \]
            Hier werden Abbleitungen 2. Ordnung benutzt.

            Es können aber auch Ableitungen höherer Ordnungen sowie Kombinationen verwendet werden.
    \end{enumerate}
\end{enumerate}

    

\printindex

\end{document}
