\theoremstyle{plain}
\newtheorem{theorem}{Theorem}
\newtheorem{lemma}[theorem]{Lemma}
\newtheorem*{lemma*}{Lemma}
\newtheorem{cor}[theorem]{Korollar}
\newtheorem*{cor*}{Korollar}
\newtheorem{prop}[theorem]{Proposition}
\setlength{\parskip}{1em}

\theoremstyle{definition}
\newtheorem{definition}[theorem]{Definition}
\newtheorem*{definition*}{Definition}
\newtheorem{notation}[theorem]{Notation}
\newtheorem{bemerkung}[theorem]{Bemerkung}
\newtheorem*{bemerkung*}{Bemerkung}
\newtheorem{bsp}[theorem]{Beispiel}
\newtheorem*{remark*}{remark}
\newtheorem{remark}{remark}

\numberwithin{equation}{section}

\newcommand{\norm}[1] {
\left\| #1 \right\|
}
%\newcommand{\norm}[1] {
%\left|\left| #1 \right|\right|
%}

\newcommand{\hnorm}[1] {
\left|\left|\left| #1 \right|\right|\right|
}

\newcommand{\colvec}[1]{
\begin{pmatrix}#1\end{pmatrix}
}

\newcommand{\vect}[1]{
\begin{pmatrix}#1\end{pmatrix}
}

\newcommand{\skprod}[2]{
\left \langle #1,#2 \right \rangle
}
\newcommand{\abs}[1] {
\left| #1 \right|
}

\newcommand{\br}[1] {
\left( #1 \right)
}

\newcommand{\R}[0] {
\mathbb R
}

\newcommand{\Z}[0] {
    \mathbb Z
}

\newcommand{\N}[0] {
    \mathbb N
}

\newcommand{\F}[0]{
    \mathcal F
}

\newcommand{\srmatrix}[1] {
\left( \begin{smallmatrix} #1 \end{smallmatrix} \right)
}

\newcommand{\mtitle}[1] {
    \begin{center}
        \large{\textbf{#1}}
    \end{center}
}

\newcommand{\Index}[1]{#1\index{#1}}

\newcommand{\mim}[1] {
\underline{\textbf{#1\index{#1}}}
}

\newcommand{\C}[0]{
    \cdot
}

\newcommand{\pa}[1] {
    \par{\textbf{#1}}
}

\newcommand{\tvmark}[3] {
    \draw[] (#1,0.1+#2) -- (#1,-0.1+#2) node[anchor =north] {\small #3};
}
\newcommand{\thmark}[3] {
    \draw[] (#1+0.1,#2) -- (#1-0.1,#2) node[left] {\small #3};
}

\newcommand{\tvnmark}[3] {
    \draw[] (#1,#2-0.1) node[below] {\small #3};
}

\newcommand{\tlbar}[5]{
    \draw[thick] (#1,#2-0.1) node[below] {\small #4} -- (#1,#3) node[above] {#5};
    \draw[thick] (#1-0.1,#3) -- (#1+0.1,#3);
}

\newcommand{\filter}[1] {
\begin{tabular}{|c|c|c|}
    \hline
    #1\\
    \hline
\end{tabular}
}
\newcommand{\quo}[1] {
\glqq #1 \grqq
}

\newcommand{\x}[0] {
  \boldsymbol{x}
}
\newcommand{\y}[0] {
    \boldsymbol{y}
}

\newcommand{\mat}[1] {
\begin{pmatrix} #1 \end{pmatrix}
}

\definecolor{dkgreen}{rgb}{0,0.6,0}
\definecolor{gray}{rgb}{0.5,0.5,0.5}
\definecolor{mauve}{rgb}{0.58,0,0.82}

\lstset{frame=tb,
language=Matlab,
aboveskip=10mm,
belowskip=10mm,
showstringspaces=false,
columns=flexible,
basicstyle={\small\ttfamily},
numbers=left,
identifierstyle=\color{black},
numberstyle=\small\color{gray},
keywordstyle=\color{blue},
commentstyle=\color{dkgreen},
stringstyle=\color{mauve},
breaklines=true,
breakatwhitespace=true,
tabsize=3}

%Font
%\usepackage{tgadventor}
\renewcommand*\rmdefault{cmss}

%Differentialoperatoren
\let\divsymb=\div % rename builtin command \div to \divsymb
\renewcommand{\div}[1]{\mathrm{div\,} #1} % for divergence
\renewcommand{\d}[1]{\ensuremath\, {\operatorname{d}\!{#1}}}
\newcommand{\DD}{\ensuremath\,{\operatorname{D}}}

%sonstige Operatoren
\newcommand{\TV}{\operatorname{TV}}% Total Variation
\newcommand{\loc}{\mathrm{loc}} %für Ell^1_\loc
\newcommand{\sign}{\operatorname{sign}} %für Ell^1_\loc
\newcommand{\supp}{\operatorname{supp}} %für Ell^1_\loc

%Konstanten 
\newcommand{\ii}{\mathrm{i}} %imaginäre Einheit
\newcommand{\GG}{\mathrm{G}} %Gauß-Kern
\newcommand{\e}{\mathrm{e}}  %Eulersche Zahl

%Funktionenräume
\newcommand{\CC}{\mathrm{C}} %Stetige Funktionen
\newcommand{\Ell}{\mathrm{L}} %Lp (vgl \ell)
\newcommand{\WW}{\mathrm{W}} %Sobolev
\newcommand{\HH}{\mathrm{H}} %Sobolev W^{k,2}
